\documentclass[lang = zh , final , oneside , openany , titlepage , zihao = -4 , linespread = 1.3 , baselineskip = false , cjk-font = windows , text-font = newtx , math-font = newtx]{sjtureport}

\usepackage{amsmath}
\usepackage{amsthm}
\usepackage[hidelinks]{hyperref}%目录超链接并隐藏
\usepackage{graphicx}%图片路径支持
\usepackage{gbt7714}
\bibliographystyle{gbt7714-numerical}
\usepackage{booktabs} % 表格支持
\usepackage{longtable} % 表格支持
%\usepackage{subcaption}%小标题支持

%\sjtusetup
%{
%  info =
%  {
%    zh/title = {上海交通大学学位论文模板示例文档},
%    en/title = {A Sample Document for SJTU Thesis Template},
%    zh/author = {某某},
%    en/author = {Mo Mo},
%  },
%
%  style =
%  {
%    float-num-sep = {-},
%  },
%
%  name =
%  {
%    achv = {攻读学位期间完成的论文},
%  },
%}

\title{物理必修二}
%\author{某某}
%\subject{XX期末课程论文}
%\keywords{上海交大, 饮水思源, 爱国荣校}

\begin{document}

\maketitle

\setcounter{page}{1}  % 将页码设置为1
\pagestyle{plain}     % 设置为普通页码样式
\tableofcontents

%\begin{abstract}
%本模板是上海交通大学本科生课程论文、学位论文、学术报告等文档的LaTeX模板,旨在帮助学生快速上手LaTeX排版。
%\end{abstract}

\newpage
\setcounter{page}{1}  % 将页码设置为1
\pagestyle{plain}     % 设置为普通页码样式
\chapter{静电场及其应用}
\section{电荷}
\subsection{电荷}

\begin{definition}
    电荷的多少叫做\textbf{电荷量(electric quantity)},用 $Q$ 或者 $q$ 表示,单位为 \textbf{库伦 (coulomb)},简称  \textbf{库},符号为 $C$。
\end{definition}

\subsection{静电感应}

\begin{definition}
    带电体靠近导体,由于电荷间的相互吸引或者排斥,导体中的自由电荷会趋向或者
\end{definition}

\nocite{*}
\bibliography{ref}
\end{document}