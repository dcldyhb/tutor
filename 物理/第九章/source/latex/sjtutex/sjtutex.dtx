% \iffalse meta-comment
%
% Copyright (C) SJTUG
%   2018--2025 Weijian Wu   <alexarawu@outlook.com>
%   2022--2025 Zilong Li    <logcreative@outlook.com>
%   2024--2025 Boshi Yuan   <nemoyuan2008@outlook.com>
%
% This work may be distributed and/or modified under the
% conditions of the LaTeX Project Public License, either
% version 1.3c of this license or (at your option) any later
% version. The latest version of this license is in:
%
%   http://www.latex-project.org/lppl.txt
%
% and version 1.3 or later is part of all distributions of
% LaTeX version 2005/12/01 or later.
%
%<*internal>
\iffalse
%</internal>
%
%<*internal>
\fi
\begingroup
  \def\NameOfLaTeXe{LaTeX2e}
\expandafter\endgroup\ifx\NameOfLaTeXe\fmtname\else
\csname fi\endcsname
%</internal>
%
%<*install>
\input ctxdocstrip.tex
\keepsilent
\askforoverwritefalse

\preamble

    Copyright (C) SJTUG
      2018--2025 Weijian Wu   <alexarawu@outlook.com>
      2022--2025 Zilong Li    <logcreative@outlook.com>
      2024--2025 Boshi Yuan   <nemoyuan2008@outlook.com>

    This work may be distributed and/or modified under the
    conditions of the LaTeX Project Public License, either
    version 1.3c of this license or (at your option) any later
    version. The latest version of this license is in:

      http://www.latex-project.org/lppl.txt

    and version 1.3 or later is part of all distributions of
    LaTeX version 2005/12/01 or later.

\endpreamble

\generate{
  \usedir{tex/latex/sjtutex}
    \file{sjtuthesis.cls}               {\from{\jobname.dtx}{class,thesis}}
    \file{sjtureport.cls}               {\from{\jobname.dtx}{class,report}}
    \file{sjtuarticle.cls}              {\from{\jobname.dtx}{class,article}}
    \file{sjtu-lang-zh.def}             {\from{\jobname.dtx}{lang,zh}}
    \file{sjtu-lang-en.def}             {\from{\jobname.dtx}{lang,en}}
    \file{sjtu-lang-de.def}             {\from{\jobname.dtx}{lang,de}}
    \file{sjtu-lang-ja.def}             {\from{\jobname.dtx}{lang,ja}}
    \file{sjtu-scheme-zh.def}           {\from{\jobname.dtx}{scheme,zh}}
    \file{sjtu-scheme-en.def}           {\from{\jobname.dtx}{scheme,en}}
    \file{sjtu-scheme-de.def}           {\from{\jobname.dtx}{scheme,de}}
    \file{sjtu-scheme-ja.def}           {\from{\jobname.dtx}{scheme,ja}}
    \file{sjtu-thesis-zh.def}           {\from{\jobname.dtx}{thesis-i18n,zh}}
    \file{sjtu-thesis-en.def}           {\from{\jobname.dtx}{thesis-i18n,en}}
    \file{sjtu-thesis-de.def}           {\from{\jobname.dtx}{thesis-i18n,de}}
    \file{sjtu-thesis-ja.def}           {\from{\jobname.dtx}{thesis-i18n,ja}}
    \file{sjtu-text-font-newtx.def}     {\from{\jobname.dtx}{font,text,type1,newtx}}
    \file{sjtu-text-font-newpx.def}     {\from{\jobname.dtx}{font,text,type1,newpx}}
    \file{sjtu-text-font-lm.def}        {\from{\jobname.dtx}{font,text,type1,lm}}
    \file{sjtu-text-font-times.def}     {\from{\jobname.dtx}{font,text,type1,times}}
    \file{sjtu-text-font-libertinus.def}{\from{\jobname.dtx}{font,text,libertinus}}
    \file{sjtu-text-font-stixtwo.def}   {\from{\jobname.dtx}{font,text,stixtwo}}
    \file{sjtu-text-font-cambria.def}   {\from{\jobname.dtx}{font,text,otf,cambria}}
    \file{sjtu-text-font-newcm.def}     {\from{\jobname.dtx}{font,text,otf,newcm}}
    \file{sjtu-text-font-xits.def}      {\from{\jobname.dtx}{font,text,otf,xits}}
    \file{sjtu-math-font-newtx.def}     {\from{\jobname.dtx}{font,math,type1,newtx}}
    \file{sjtu-math-font-newpx.def}     {\from{\jobname.dtx}{font,math,type1,newpx}}
    \file{sjtu-math-font-lm.def}        {\from{\jobname.dtx}{font,math,type1,lm}}
    \file{sjtu-math-font-times.def}     {\from{\jobname.dtx}{font,math,type1,times}}
    \file{sjtu-math-font-libertinus.def}{\from{\jobname.dtx}{font,math,libertinus}}
    \file{sjtu-math-font-stixtwo.def}   {\from{\jobname.dtx}{font,math,stixtwo}}
    \file{sjtu-math-font-cambria.def}   {\from{\jobname.dtx}{font,math,otf,cambria}}
    \file{sjtu-math-font-newcm.def}     {\from{\jobname.dtx}{font,math,otf,newcm}}
    \file{sjtu-math-font-xits.def}      {\from{\jobname.dtx}{font,math,otf,xits}}
    \file{sjtu-cjk-font-windows.def}    {\from{\jobname.dtx}{font,cjk,windows}}
    \file{sjtu-cjk-font-mac.def}        {\from{\jobname.dtx}{font,cjk,mac}}
    \file{sjtu-cjk-font-ubuntu.def}     {\from{\jobname.dtx}{font,cjk,ubuntu}}
    \file{sjtu-cjk-font-adobe.def}      {\from{\jobname.dtx}{font,cjk,adobe}}
    \file{sjtu-cjk-font-fandol.def}     {\from{\jobname.dtx}{font,cjk,fandol}}
    \file{sjtu-cjk-font-founder.def}    {\from{\jobname.dtx}{font,cjk,founder}}
    \file{sjtu-cjk-font-hanyi.def}      {\from{\jobname.dtx}{font,cjk,hanyi}}
%</install>
%<*internal>
  \usedir{source/latex/sjtutex}
    \file{\jobname.ins}                 {\from{\jobname.dtx}{install}}
%</internal>
%<*install>
}

\obeyspaces
\Msg{*************************************************************}
\Msg{*                                                           *}
\Msg{* To finish the installation you have to move the following *}
\Msg{* files into a directory searched by TeX:                   *}
\Msg{*                                                           *}
\Msg{* The recommended directory is TDS:tex/latex/sjtutex        *}
\Msg{*                                                           *}
\Msg{*     sjtuthesis.cls                                        *}
\Msg{*     sjtureport.cls                                        *}
\Msg{*     sjtuarticle.cls                                       *}
\Msg{*     sjtu-lang-de.def                                      *}
\Msg{*     sjtu-lang-en.def                                      *}
\Msg{*     sjtu-lang-ja.def                                      *}
\Msg{*     sjtu-lang-zh.def                                      *}
\Msg{*     sjtu-scheme-de.def                                    *}
\Msg{*     sjtu-scheme-en.def                                    *}
\Msg{*     sjtu-scheme-ja.def                                    *}
\Msg{*     sjtu-scheme-zh.def                                    *}
\Msg{*     sjtu-thesis-de.def                                    *}
\Msg{*     sjtu-thesis-en.def                                    *}
\Msg{*     sjtu-thesis-ja.def                                    *}
\Msg{*     sjtu-thesis-zh.def                                    *}
\Msg{*     sjtu-text-font-cambria.def                            *}
\Msg{*     sjtu-text-font-libertinus.def                         *}
\Msg{*     sjtu-text-font-lm.def                                 *}
\Msg{*     sjtu-text-font-newcm.def                              *}
\Msg{*     sjtu-text-font-newpx.def                              *}
\Msg{*     sjtu-text-font-newtx.def                              *}
\Msg{*     sjtu-text-font-stixtwo.def                            *}
\Msg{*     sjtu-text-font-times.def                              *}
\Msg{*     sjtu-text-font-xits.def                               *}
\Msg{*     sjtu-math-font-cambria.def                            *}
\Msg{*     sjtu-math-font-libertinus.def                         *}
\Msg{*     sjtu-math-font-lm.def                                 *}
\Msg{*     sjtu-math-font-newcm.def                              *}
\Msg{*     sjtu-math-font-newpx.def                              *}
\Msg{*     sjtu-math-font-newtx.def                              *}
\Msg{*     sjtu-math-font-stixtwo.def                            *}
\Msg{*     sjtu-math-font-times.def                              *}
\Msg{*     sjtu-math-font-xits.def                               *}
\Msg{*     sjtu-cjk-font-adobe.def                               *}
\Msg{*     sjtu-cjk-font-fandol.def                              *}
\Msg{*     sjtu-cjk-font-founder.def                             *}
\Msg{*     sjtu-cjk-font-hanyi.def                               *}
\Msg{*     sjtu-cjk-font-mac.def                                 *}
\Msg{*     sjtu-cjk-font-ubuntu.def                              *}
\Msg{*     sjtu-cjk-font-windows.def                             *}
\Msg{*     sjtu-vi-badge-cor-red.pdf                             *}
\Msg{*     sjtu-vi-badge-reg-red.pdf                             *}
\Msg{*     sjtu-vi-logo-eng-h-cor-red.pdf                        *}
\Msg{*     sjtu-vi-logo-eng-h-reg-red.pdf                        *}
\Msg{*     sjtu-vi-logo-std-c-cor-red.pdf                        *}
\Msg{*     sjtu-vi-logo-std-c-reg-red.pdf                        *}
\Msg{*     sjtu-vi-logo-std-h-cor-red.pdf                        *}
\Msg{*     sjtu-vi-logo-std-h-reg-red.pdf                        *}
\Msg{*     sjtu-vi-logo-std-v-cor-red.pdf                        *}
\Msg{*     sjtu-vi-logo-std-v-reg-red.pdf                        *}
\Msg{*                                                           *}
\Msg{* To produce the documentation, run the file sjtutex.dtx    *}
\Msg{* through XeLaTeX.                                          *}
\Msg{*                                                           *}
\Msg{* Happy TeXing!                                             *}
\Msg{*                                                           *}
\Msg{*************************************************************}

\endbatchfile
%</install>
%
%<*internal>
\fi
%</internal>
%
%<*class>
\NeedsTeXFormat{LaTeX2e}[2024-06-01]
\providecommand\IfFormatAtLeastTF{\@ifl@t@r\fmtversion}
\IfFormatAtLeastTF{2024-06-01}{}
 {\PackageError{sjtutex}
   {Your LaTeX format is outdated!\MessageBreak\MessageBreak
    Release '2024-06-01' or newer is required to\MessageBreak
    compile this document.\MessageBreak\MessageBreak
    Please update your TeX distribution}{}}
%</class>
%<*!(driver|install)>
%<+!driver>\GetIdInfo$Id: sjtutex.dtx 24e1ee7 2025-03-29 01:12:32 +0800 Alexara Wu <alexarawu@outlook.com> $
%<class&thesis>  {Thesis document class for Shanghai Jiao Tong University (SJTUTeX)}
%<class&thesis>\ProvidesExplClass{sjtuthesis}
%<class&report>  {Report document class for Shanghai Jiao Tong University (SJTUTeX)}
%<class&report>\ProvidesExplClass{sjtureport}
%<class&article>  {Article document class for Shanghai Jiao Tong University (SJTUTeX)}
%<class&article>\ProvidesExplClass{sjtuarticle}
%<lang&zh>  {Chinese specific definition (SJTUTeX)}
%<lang&zh>\ProvidesExplFile{sjtu-lang-zh.def}
%<lang&en>  {English specific definition (SJTUTeX)}
%<lang&en>\ProvidesExplFile{sjtu-lang-en.def}
%<lang&de>  {German specific definition (SJTUTeX)}
%<lang&de>\ProvidesExplFile{sjtu-lang-de.def}
%<lang&ja>  {Japanese specific definition (SJTUTeX)}
%<lang&ja>\ProvidesExplFile{sjtu-lang-ja.def}
%<scheme&zh>  {Chinese scheme (SJTUTeX)}
%<scheme&zh>\ProvidesExplFile{sjtu-scheme-zh.def}
%<scheme&en>  {English scheme (SJTUTeX)}
%<scheme&en>\ProvidesExplFile{sjtu-scheme-en.def}
%<scheme&de>  {German scheme (SJTUTeX)}
%<scheme&de>\ProvidesExplFile{sjtu-scheme-de.def}
%<scheme&ja>  {Japanese scheme (SJTUTeX)}
%<scheme&ja>\ProvidesExplFile{sjtu-scheme-ja.def}
%<thesis-i18n&zh>  {Chinese config for thesis (SJTUTeX)}
%<thesis-i18n&zh>\ProvidesExplFile{sjtu-thesis-zh.def}
%<thesis-i18n&en>  {English config for thesis (SJTUTeX)}
%<thesis-i18n&en>\ProvidesExplFile{sjtu-thesis-en.def}
%<thesis-i18n&de>  {German config for thesis (SJTUTeX)}
%<thesis-i18n&de>\ProvidesExplFile{sjtu-thesis-de.def}
%<thesis-i18n&ja>  {Japanese config for thesis (SJTUTeX)}
%<thesis-i18n&ja>\ProvidesExplFile{sjtu-thesis-ja.def}
%<font&text&newtx>  {New TX text fonts definition (SJTUTeX)}
%<font&text&newtx>\ProvidesExplFile{sjtu-text-font-newtx.def}
%<font&text&newpx>  {New PX text fonts definition (SJTUTeX)}
%<font&text&newpx>\ProvidesExplFile{sjtu-text-font-newpx.def}
%<font&text&lm>  {Latin Modern text fonts definition (SJTUTeX)}
%<font&text&lm>\ProvidesExplFile{sjtu-text-font-lm.def}
%<font&text&times>  {Times text fonts definition (SJTUTeX)}
%<font&text&times>\ProvidesExplFile{sjtu-text-font-times.def}
%<font&text&stixtwo>  {STIX Two text fonts definition (SJTUTeX)}
%<font&text&stixtwo>\ProvidesExplFile{sjtu-text-font-stixtwo.def}
%<font&text&libertinus>  {Libertinus text fonts definition (SJTUTeX)}
%<font&text&libertinus>\ProvidesExplFile{sjtu-text-font-libertinus.def}
%<font&text&cambria>  {Cambria text fonts definition (SJTUTeX)}
%<font&text&cambria>\ProvidesExplFile{sjtu-text-font-cambria.def}
%<font&text&newcm>  {New Computer Modern text fonts definition (SJTUTeX)}
%<font&text&newcm>\ProvidesExplFile{sjtu-text-font-newcm.def}
%<font&text&xits>  {XITS text fonts definition (SJTUTeX)}
%<font&text&xits>\ProvidesExplFile{sjtu-text-font-xits.def}
%<font&math&newtx>  {New TX math fonts definition (SJTUTeX)}
%<font&math&newtx>\ProvidesExplFile{sjtu-math-font-newtx.def}
%<font&math&newpx>  {New PX math fonts definition (SJTUTeX)}
%<font&math&newpx>\ProvidesExplFile{sjtu-math-font-newpx.def}
%<font&math&lm>  {Latin Modern math fonts definition (SJTUTeX)}
%<font&math&lm>\ProvidesExplFile{sjtu-math-font-lm.def}
%<font&math&times>  {Times math fonts definition (SJTUTeX)}
%<font&math&times>\ProvidesExplFile{sjtu-math-font-times.def}
%<font&math&stixtwo>  {STIX Two math fonts definition (SJTUTeX)}
%<font&math&stixtwo>\ProvidesExplFile{sjtu-math-font-stixtwo.def}
%<font&math&libertinus>  {Libertinus math fonts definition (SJTUTeX)}
%<font&math&libertinus>\ProvidesExplFile{sjtu-math-font-libertinus.def}
%<font&math&cambria>  {Cambria math fonts definition (SJTUTeX)}
%<font&math&cambria>\ProvidesExplFile{sjtu-math-font-cambria.def}
%<font&math&newcm>  {New Computer Modern math fonts definition (SJTUTeX)}
%<font&math&newcm>\ProvidesExplFile{sjtu-math-font-newcm.def}
%<font&math&xits>  {XITS math fonts definition (SJTUTeX)}
%<font&math&xits>\ProvidesExplFile{sjtu-math-font-xits.def}
%<font&cjk&windows>  {Windows CJK fonts definition (SJTUTeX)}
%<font&cjk&windows>\ProvidesExplFile{sjtu-cjk-font-windows.def}
%<font&cjk&mac>  {macOS CJK fonts definition (SJTUTeX)}
%<font&cjk&mac>\ProvidesExplFile{sjtu-cjk-font-mac.def}
%<font&cjk&ubuntu>  {Ubuntu CJK fonts definition (SJTUTeX)}
%<font&cjk&ubuntu>\ProvidesExplFile{sjtu-cjk-font-ubuntu.def}
%<font&cjk&adobe>  {Adobe CJK fonts definition (SJTUTeX)}
%<font&cjk&adobe>\ProvidesExplFile{sjtu-cjk-font-adobe.def}
%<font&cjk&fandol>  {Fandol CJK fonts definition (SJTUTeX)}
%<font&cjk&fandol>\ProvidesExplFile{sjtu-cjk-font-fandol.def}
%<font&cjk&founder>  {Founder CJK fonts definition (SJTUTeX)}
%<font&cjk&founder>\ProvidesExplFile{sjtu-cjk-font-founder.def}
%<font&cjk&hanyi>  {Hanyi CJK fonts definition (SJTUTeX)}
%<font&cjk&hanyi>\ProvidesExplFile{sjtu-cjk-font-hanyi.def}
%<!driver>  {\ExplFileDate}{2.2.1}{\ExplFileDescription}
%</!(driver|install)>
%
%<*driver>
\PassOptionsToPackage{fontset=ubuntu}{ctex}
\documentclass{ctxdoc}
\newcommand{\sjtutex}{SJTU\TeX}
\expandafter\def\csname ver@\jobname.dtx\endcsname
  {2025/03/29 v2.2.1 SJTUTeX}
\usepackage{float}
\usepackage{multirow}
\usepackage{longtable}
\usepackage{listings}
\AtEndOfClass{\sloppy}
\definecolor{sjtu-red}{RGB}{200,22,30}
\definecolor{sjtu-dark-red}{RGB}{167,32,56}
\definecolor{sjtu-amber}{RGB}{253,208,0}
\definecolor{sjtu-orange}{RGB}{240,131,0}
\definecolor{sjtu-blue}{RGB}{0,134,209}
\definecolor{sjtu-dark-blue}{RGB}{0,64,152}
\definecolor{sjtu-green}{RGB}{51,141,39}
\definecolor{sjtu-dark-green}{RGB}{0,81,78}
\lstdefinestyle{style@base}{^^A
  basewidth         = 0.5 em,
  gobble            = 2,
  lineskip          = 2 pt,
  frame             = l,
  framerule         = 1.2 pt,
  framesep          = 0 pt,
  escapeinside      = {(*}{*)},
  basicstyle        = \small\ttfamily,
  xleftmargin       = 0.5 em,
  xrightmargin      = 0.5 em,
  framexleftmargin  = 0.5 em,
  framexrightmargin = 0.5 em}
\lstdefinestyle{style@shell}{^^A
  style             = style@base,
  rulecolor         = \color{sjtu-dark-green},
  backgroundcolor   = \color{sjtu-dark-green!2},
  language          = bash,
  alsoletter        = {-},
  keywordstyle      = \color{sjtu-green},
  emphstyle         = \color{sjtu-amber}}
\lstdefinestyle{style@latex}{^^A
  style             = style@base,
  rulecolor         = \color{sjtu-dark-blue},
  backgroundcolor   = \color{sjtu-dark-blue!2},
  language          = [LaTeX]TeX,
  alsoletter        = {*, -},
  keywordstyle      = \color{sjtu-dark-red!80},
  texcsstyle        = *\color{sjtu-blue},
  emphstyle         = [1]\color{sjtu-orange},
  emphstyle         = [2]\color{sjtu-green},
  emphstyle         = [3]\color{sjtu-dark-blue!80}}
\lstnewenvironment{shell}[1][]{^^A
  \lstset{style=style@shell, #1}}{}
\lstnewenvironment{latex}[1][]{^^A
  \lstset{style=style@latex, #1}}{}
\newcommand\TNA[1]{\textcolor{sjtu-blue}{\tn{#1}}}
\newcommand\TNB[1]{\textcolor{sjtu-dark-red!80}{\tn{#1}}}
\newcommand\OPT[1]{\textcolor{sjtu-dark-blue!80}{#1}}
\newcommand\OPS[1]{\textcolor{sjtu-green}{#1}}
\newcommand\ENV[1]{\textcolor{sjtu-orange}{#1}}
\newcommand\BEV[1]{\TNA{begin}\{\ENV{#1}\}}
\newcommand\EEV[1]{\TNA{end}\{\ENV{#1}\}}
\newcommand\note[1]{{^^A
  \color{sjtu-dark-green}{\noindent\bfseries 说明:}~\emph{#1}}}
\newcommand\sjtutexrev[1]{^^A
  \href{https://github.com/sjtug/SJTUTeX/commit/#1}{\texttt{sjtutex} rev. #1}}
\hypersetup{^^A
  bookmarksnumbered = true,
  bookmarksopen     = true}
\makeatletter
\pdfstringdefDisableCommands{^^A
  \let\color\@gobble}
\makeatother
\begin{document}
  \DocInput{\jobname.dtx}
  \IndexLayout
  \PrintChanges
  \PrintIndex
\end{document}
%</driver>
% \fi
%
% \changes{v2.0}{2021/09/10}{应用 \LaTeXiii{} 重构代码。}
% \changes{v2.0}{2022/12/27}{新增文档类 \cls{sjtuarticle} 和 \cls{sjtureport}。}
%
% \GetFileId{\jobname.dtx}
%
% \title{\bfseries\color{sjtu-red}\sjtutex{}: 上海交通大学文档类集}
% \author{SJTUG}
% \date{\filedate\quad\fileversion\thanks{\sjtutexrev{\ExplFileVersion}.}}
%
% \hypersetup{^^A
%   pdfsubject  = {上海交通大学文档类集使用说明},
%   pdfkeywords = {模板; 上海交通大学; 使用说明}}
%
% \maketitle
%
% \begin{abstract}
% 此宏包旨在建立一个简单易用的上海交通大学文档类集,包括学位论文文档类
% \cls{sjtuthesis} 以及普通文档类 \cls{sjtuarticle} 和 \cls{sjtureport}。
% \end{abstract}
%
% \def\abstractname{免责声明}
% \begin{abstract}
% \noindent
% \begin{enumerate}
% \item 本模板的发布遵守 \href{https://www.latex-project.org/lppl/lppl-1-3c.txt}
%   {\LaTeX{} Project Public License (1.3c)},使用前请认真阅读协议内容。
% \item 学位论文模板根据 \href{https://www.gs.sjtu.edu.cn/post/detail/Z3MxNDc=}
%   {《上海交通大学博士、硕士学位论文撰写指南》} ,并参考
%   \href{https://www.jwc.sjtu.edu.cn/info/1041/117021.htm}
%   {《上海交通大学本科生毕业设计(论文)撰写规范》}
%   编写而成。旨在供上海交通大学准毕业生撰写学位论文使用。
% \item 学位论文模板仅为撰写指南的参考实现,不保证审查老师不提意见。任何由于使用
%   本模板而引起的论文格式审查问题均与本模板作者无关。
% \item 任何个人或组织以本模板为基础进行修改、扩展而生成的新的专用模板,请严格遵
%   守 \LaTeX{} Project Public License 协议。由于违犯协议而引起的任何纠纷争端均
%   与本模板作者无关。
% \end{enumerate}
% \end{abstract}
%
% \tableofcontents
%
% \begin{documentation}
%
% \section{介绍}
%
% 最早的一版 \LaTeX{} 学位论文模板由一位热心的物理系同学制作,中文字符处理采用了
% 当时最为流行的 CJK--\LaTeX{} 方案。在此基础上,weijianwen 根据交大研究生院对学
% 位论文的要求,完成了一份基本可用的交大 \LaTeX{} 学位论文模板。由于 CJK--\LaTeX{}
% 方案不易使用,weijianwen 与 William Wang 开始着手把模板向 \XeTeX{} 引擎移植。
% 之后 weijianwen又断断续续做了一些完善模板的工作,在原有硕士学位论文模板的基础
% 上完成了交大学士和博士学位论文模板。
%
% 2012 年 5 月模板开始在 GitHub^^A
% \footnote{\url{https://github.com/weijianwen/SJTUThesis},项目转移后该链接已重定向。}^^A
% 上管理和更新,2018 年 1 月项目转移至 SJTUG 名下。2019 年 6 月 Alexara Wu 重构
% 了整个宏包的代码,并使用 Doc\TeX{} 文档和\textsc{DocStrip} 工具进行代码的管理,
% 升级版本号为 1.0。2022 年 11 月,论文模板改版后,使用 \LaTeXiii{} 重构了代码,
% 添加 \cls{sjtureport} 和 \cls{sjtuarticle} 文档类,升级版本号为 2.0。
% 2024 年 2 月,\pkg{sjtutex} 文档类集在 CTAN 上发布,被 \TeX{} Live 2024 及之后
% 的版本收录。
%
% 现在,\sjtutex{} 代码在 GitHub^^A
% \footnote{\url{https://github.com/sjtug/SJTUTeX}}^^A
% 上维护,并同步发布至 CTAN\footnote{\url{https://www.ctan.org/pkg/sjtutex}} 上。
% 原 \href{https://github.com/sjtug/SJTUThesis}{SJTUThesis} 仓库则作为学位论文
% 示例文档,提供开箱即用的模板。学位论文模板用户可以在
% \href{https://github.com/sjtug/SJTUThesis/discussions}{Discussions}
% 上提问使用问题,也可以在
% \href{https://github.com/sjtug/SJTUThesis/issues}{Issues}
% 中进行 Bug 反馈与新功能提案。如果需要对文档类代码进行修改,欢迎前往
% \href{https://github.com/sjtug/SJTUTeX}{\sjtutex} 仓库进行 Pull Request。
% \sjtutex{} 模板的许多实现细节离不开
% \href{https://github.com/sjtug/SJTUThesis/graphs/contributors}{热心同学们}
% 的贡献,在此感谢所有为模板贡献过代码的同学们,以及所有测试和使用模板的各位同学!
%
% \note{模板的作用在于减少论文写作过程中格式调整的时间,前提是遵守模板的用法,否
%   则即便用了 \sjtutex{} 也难以保证输出的论文符合学校规范。}
%
% \section{简明教程}
%
% \subsection{安装 \TeX{} 发行版}
%
% 因为 \LaTeXiii{} 和相关宏包在不断更新,所以推荐使用最新的 \TeX{} 发行版。
% \sjtutex{} 支持主流的 \TeX{} 发行版,包括 \TeX{} Live、\hologo{MiKTeX}、
% Mac\TeX{},但不支持 \CTeX{} 套装。安装方法具体可以参考
% \href{https://github.com/sjtug/SJTUThesis/wiki/TeX-发行版及其安装}
% {Wiki 页面《\TeX{} 发行版及其安装》}。
%
% \sjtutex{} 需要 \LaTeXe{} 内核 2024-06-01 或以上版本,\pkg{ctex} 宏包版本应当
% 在 v2.5 及以上。版本过低将无法正常编译。
%
% \subsection{文件组成}
%
% 表~\ref{tab:files} 列出了 \sjtutex{} 的主要文件及其功能介绍。
%
% \begin{table}[!hbt]
% \centering
% \begin{threeparttable}
%   \caption{模板的文件组成}
%   \label{tab:files}
%   \begin{tabular}{l|l|p{4cm}}
%     \toprule
%     \strong{类别}         & \strong{文件}               & \strong{说明}    \\
%     \midrule
%     文档类                & \file{sjtuthesis.cls}       & 学位论文文档类   \\
%                           & \file{sjtureport.cls}       & 报告文档类       \\
%                           & \file{sjtuarticle.cls}      & 文稿文档类       \\
%                           & \file{sjtu-lang-*.def}      & 文档类语言配置   \\
%                           & \file{sjtu-scheme-*.def}    & 文档类语言方案   \\
%                           & \file{sjtu-thesis-*.def}    & 学位论文语言配置 \\
%     \midrule
%     字体配置文件          & \file{sjtu-text-font-*.def} & 西文字体配置     \\
%                           & \file{sjtu-math-font-*.def} & 数学字体配置     \\
%                           & \file{sjtu-cjk-font-*.def}  & CJK 字体配置     \\
%     \midrule
%     视觉形象系统\tnote{a} & \file{sjtu-vi-logo-*.pdf}   & 校标图片         \\
%                           & \file{sjtu-vi-badge-*.pdf}  & 校徽图片         \\
%
%     \bottomrule
%   \end{tabular}
%   \begin{tablenotes}
%     \item[a] \href{https://vi.sjtu.edu.cn}{交大视觉形象系统}
%       的相关图像资源版权归上海交通大学所有。
%   \end{tablenotes}
% \end{threeparttable}
% \end{table}
%
% \subsection{使用文档类}
%
% 共有 3 种文档类,\cls{sjtuthesis} 用于学位论文的排版,\cls{sjtureport}
% 用于课程大报告的排版,\cls{sjtuarticle} 用于课程小论文的排版。
%
% 将文档保存在下载好的 \sjtutex{} 模板根目录下,或者直接使用 \TeX{} 发行版
% 中的 \pkg{sjtutex} 文档类集^^A
% \footnote{为了使用最新版本的 \sjtutex{},请及时更新发行版至最新的版本:^^A
% \TeX{} Live 用户请使用命令 \texttt{tlmgr update --self --all} 来更新全部宏包;^^A
% \hologo{MiKTeX} 用户请在 MiKTeX Console 更新选项卡中更新所有宏包。},
% 文件以 \texttt{.tex} 后缀结尾。
% 注意在使用文档类时,需要将涉及到的所有源文件使用 UTF-8 编码保存。
% 对于不同的文档类,使用方法略有不同,这里给出这三种文档类的最小使用示例。
%
% 下面这份 \TeX{} 文档展示了 \cls{sjtuthesis} 文档类的基本用法,一般需要指定
% 中英文名称。该文档将包含中英文封面,页眉为文档主题及章节名称。
% \begin{latex}[moretexcs={\sjtusetup,\maketitle,\tableofcontents,\chapter},
%   morekeywords={\frontmatter,\mainmatter}, emph={[1]sjtuthesis,document},
%   emph={[2]info},emph={[3]type,zh,en,title,author}]
% \documentclass[type=master]{sjtuthesis}
% \sjtusetup{
%   info = {
%     zh/title  = {上海交通大学学位论文模板示例文档},
%     en/title  = {A Sample Document for SJTU Thesis Template},
%     zh/author = {某某},
%     en/author = {Mo Mo},
%   }
% }
% \begin{document}
%   \maketitle
%   \frontmatter
%   \tableofcontents
%   \mainmatter
%   \chapter{欢迎}
%   \section{欢迎使用 SJTUThesis}
%   你好,\LaTeX{}!
% \end{document}
% \end{latex}
%
% 下面这份 \TeX{} 文档展示了 \cls{sjtureport} 文档类的基本用法,建议使用标准命令
% 定义中文名称。该文档将包含标题页,页眉为校标图片、文档主题及章节名称。
% \begin{latex}[moretexcs={\subject,\keywords,\maketitle,\chapter},
%   emph={[1]sjtureport,document}]
% \documentclass{sjtureport}
% \title{上海交通大学报告模板示例文档}
% \author{某某}
% \subject{XX期末课程论文}
% \keywords{上海交大, 饮水思源, 爱国荣校}
% \begin{document}
%   \maketitle
%   \chapter{欢迎}
%   \section{欢迎使用 SJTUReport}
%   你好,\LaTeX{}!
% \end{document}
% \end{latex}
%
% 下面这份 \TeX{} 文档展示了 \cls{sjtuarticle} 文档类的基本用法,建议使用标准
% 命令定义中文名称,不能够使用 \tn{chapter} 这一级。该文档包含标题栏,页眉为
% 校标图片、文档主题及章节名称。
% \begin{latex}[moretexcs={\maketitle}, emph={[1]sjtuarticle,document}]
% \documentclass{sjtuarticle}
% \title{示例文档}
% \author{某某}
% \begin{document}
%   \maketitle
%   \section{欢迎使用 SJTUArticle}
%   你好,\LaTeX{}!
% \end{document}
% \end{latex}
%
% \subsection{编译文档}
%
% 文档类推荐使用 \XeLaTeX{} 或 \LuaLaTeX{} 编译,同时也支持 \pdfTeX{} 引擎。为了
% 生成正确的目录、脚注以及交叉引用,至少需要连续编译两次。
%
% 在实际使用中,一般推荐使用自动生成工具 \pkg{latexmk} 编译文档。\pkg{latexmk}
% 命令可以自动进行多步编译,直到交叉引用都被解决。
% 假设您的 \TeX{} 源文件名为 \file{main.tex},
% 可在命令行中执行如下命令使用 \XeLaTeX{} 编译文档:
% \begin{shell}[morekeywords={latexmk},emph={-xelatex}]
% latexmk -xelatex main
% \end{shell}
% 也可通过修改 \file{latexmkrc} 配置文件来控制 \pkg{latexmk} 的行为,
% 具体可以参考 \pkg{latexmk} 文档。
%
% \section{文档配置}
%
% 本模板中的选项、命令或环境可以分为以下三类:
% \begin{itemize}
%   \item 名字后面带有 \rexptarget\rexpstar{} 的,表示只能在 \cls{sjtuthesis}
%     文档类中使用;
%   \item 名字后面带有 \exptarget\expstar{} 的,表示只能在 \cls{sjtureport}
%     和 \cls{sjtuarticle} 文档类中使用;
%   \item 名字后面不带有特殊符号的,一般表示在 \cls{sjtuthesis}、\cls{sjtureport}
%     和 \cls{sjtuarticle} 文档类中都可以使用,特殊情况另作说明。
% \end{itemize}
%
% \subsection{文档类选项}
%
% 本节所指“文档类选项”是指需要在引入文档类的时候指定的选项:
%
% \begin{latex}[emph={[1]sjtuthesis}]
% \documentclass(*\oarg{文档类选项}*){sjtuthesis}
% \end{latex}
%
% 部分选项采用 \meta{key}|=|\meta{value} 的形式,需要使用逗号分隔各选项。
% 当 \meta{value} 省略时,将采用默认值。
% 在下文的说明中,将用\textbf{粗体}表示默认值。
%
% \subsubsection{通用选项}
%
% \begin{function}[rEXP,updated=2022-12-03]{type}
%   \begin{syntax}
%     \OPT{type} = <bachelor|(master)|doctor>
%   \end{syntax}
%   论文类型。三种选项分别代表学士学位论文、硕士学位论文、博士学位论文。
% \end{function}
%
% \begin{function}[updated=2023-03-23]{lang}
%   \begin{syntax}
%     \OPT{lang} = <(zh)|en|de|ja>
%   \end{syntax}
%   论文主要语言。可选中文、英文、德文或日文,该选项会改变文档中的一些标题的名字。
%   下文中 \meta{lang} 可以指定为这些选项中的其中之一。
% \end{function}
%
% \begin{function}{draft, final}
%   是否开启草稿模式。
%   \opt{draft} 开启草稿模式,所有的图片将不会被加载,超过边界的区域将会被涂上黑色色块。
%   \opt{final} 关闭草稿模式。
%   默认为 \opt{final}。
% \end{function}
%
% \begin{function}[rEXP,updated=2025-01-14]{review}
%   盲审模式。开启盲审模式会将作者姓名、导师姓名、学号、基金等个人信息留空,
%   并使得标题页的信息表格强制显示相关字段,
%   相关说明详见第 \ref{sec:sjtuthesiskey} 节;
%   开启盲审模式还会删去版权使用授权书、原创性声明和致谢页。默认关闭。
% \end{function}
%
% \subsubsection{页面设置}
%
% \begin{function}{oneside, twoside}
%   指明论文的单双面模式。
%   \opt{oneside} 为单面模式,
%   \opt{twoside} 为双面模式。
%   \begin{itemize}
%     \item 在 \cls{sjtuthesis} 文档类中,默认为 \opt{twoside}。
%     \item 在 \cls{sjtureport} 和 \cls{sjtuarticle} 文档类中,默认为 \opt{oneside}。
%   \end{itemize}
% \end{function}
%
% \begin{function}{openright, openany}
%   指明论文是否奇数页开章。
%   \opt{openright} 为从奇数页开始新章,
%   \opt{openany} 为从任意页开始新章。
%   \begin{itemize}
%     \item 在 \cls{sjtuthesis} 文档类中,默认为 \opt{openright}。
%     \item 在 \cls{sjtureport} 文档类中,默认为 \opt{openany}。
%     \item 在 \cls{sjtuarticle} 文档类中,该选项不可用。
%   \end{itemize}
% \end{function}
%
% \begin{function}[EXP,added=2022-12-23]{titlepage, notitlepage}
%   指明论文的标题形式。
%   \opt{titlepage} 为使用标题页。
%   \opt{notitlepage} 为使用标题块。
%   \begin{itemize}
%     \item 在 \cls{sjtuthesis} 文档类中,该选项不可用,只能使用标题页。
%     \item 在 \cls{sjtureport} 文档类中,默认为 \opt{titlepage}。
%     \item 在 \cls{sjtuarticle} 文档类中,默认为 \opt{notitlepage}。
%   \end{itemize}
% \end{function}
%
% \subsubsection{字体选项}
% \label{sec:fontsetup}
%
% \begin{function}[updated=2022-12-18]{zihao}
%   \begin{syntax}
%     \OPT{zihao} = <(-4)|5>
%   \end{syntax}
%   论文默认字号,可以设定为小四号或五号。
%   \begin{itemize}
%     \item 在 \cls{sjtuthesis} 和 \cls{sjtureport} 文档类中默认为小四号;
%     \item 在 \cls{sjtuarticle} 文档类中默认为五号。
%   \end{itemize}
% \end{function}
%
% \begin{function}[added=2023-10-24]{linespread}
%   \begin{syntax}
%     \OPT{linespread} = <数值>
%   \end{syntax}
%   设置行距倍数。
%   \begin{itemize}
%     \item 在 \cls{sjtuthesis} 文档类中默认不调整行距倍数;
%     \item 在 \cls{sjtuarticle} 和 \cls{sjtureport} 文档类中默认为 1.3。
%   \end{itemize}
% \end{function}
%
% \begin{function}[added=2023-10-24]{baselineskip}
%   \begin{syntax}
%     \OPT{baselineskip} = <长度|false>
%   \end{syntax}
%   正文基线间距。
%   \begin{itemize}
%     \item 在 \cls{sjtuthesis} 文档类中,默认为 20 磅。
%     \item 在 \cls{sjtureport} 和 \cls{sjtuarticle} 文档类中,默认为 \opt{false};
%           此时正文基线间距为字号的 1.2 倍。
%   \end{itemize}
% \end{function}
%
% \begin{function}{cjk-font}
%   \begin{syntax}
%     \OPT{cjk-font} = <(auto)|windows|mac|ubuntu|adobe|(fandol)|founder|hanyi|none>
%   \end{syntax}
%   指定 CJK 字体集。\sjtutex{} 预定义了一些 CJK 字体组合,
%   具体配置见表 \ref{tab:cjkfonts}。
%   默认情况下会根据操作系统自动配置:
%   Windows 系统默认使用 \opt{windows},
%   macOS 系统默认使用 \opt{mac},
%   Linux 系统默认使用 \opt{fandol}。
%   找不到对应定义的 CJK 字体集时的回退选项为 \opt{fandol}。
%
%   \textbf{仅} \opt{windows}、\opt{founder} 和 \opt{hanyi} 字体集支持 \pdfLaTeX{} 直接生成 PDF。
%   其他 CJK 字体集使用 \pdfTeX{} 引擎需要先通过 \LaTeX{} 生成 DVI,
%   然后再使用 DVIPDFM\textit{x} 转换为 PDF。
% \end{function}
%
% \begin{table}[ht]
% \centering\small
% \setlength\leftskip{0pt plus 1 fil minus \marginparwidth}
% \begin{threeparttable}
%   \caption{CJK 字体配置}
%   \label{tab:cjkfonts}
%   \tabcolsep=3pt
%   \begin{tabular}{ccccccc}
%     \toprule
%     	                   & \strong{宋体}     & \strong{黑体}    & \strong{仿宋} & \strong{楷体}   & \strong{明朝体}\tnote{c} & \strong{哥特体}\tnote{c}  \\
%     \midrule
%       |windows|          & (中易)宋体      & (中易)黑体     & (中易)仿宋  & (中易)楷体    & MS Mincho                & MS Gothic                 \\
%       |mac|              & (华文)宋体-简   & (华文)黑体-简  & 华文仿宋      & (华文)楷体-简 & Hiragino Mincho ProN     & Hiragino Kaku Gothic ProN \\
%       |ubuntu|           & Noto Serif CJK SC & Noto Sans CJK SC & ---           & 文鼎 PL 简中楷  & Noto Serif CJK JP        & Noto Sans CJK JP          \\
%       |adobe|            & Adobe 宋体        & Adobe 黑体       & Adobe 仿宋    & Adobe 楷体      & Kozuka Mincho Pr6N       & Kozuka Gothic Pr6N        \\
%       |fandol|\tnote{a}  & Fandol 宋体       & Fandol 黑体      & Fandol 仿宋   & Fandol 楷体     & HaranoAjiMincho          & HaranoAjiGothic           \\
%       |founder|\tnote{b} & 方正书宋          & 方正黑体         & 方正仿宋      & 方正楷体        & IPAMincho                & IPAGothic                 \\
%       |hanyi|            & 汉仪书宋二 S      & 汉仪中黑 S       & 汉仪仿宋 S    & 汉仪楷体 S      & HaranoAjiMincho          & HaranoAjiGothic           \\
%     \bottomrule
%   \end{tabular}
%   \begin{tablenotes}
%     \item[a] 发行版中自带的 Fandol 中文字库容易出现缺字的情况;
%       我们建议 Linux 用户使用 \opt{ubuntu} 选项或自行配置合适的字体;参见
%       \href{https://github.com/sjtug/SJTUThesis/wiki/在线使用说明}
%       {Wiki 页面《在线使用说明》}。
%     \item[b] 配置 \opt{founder} 选项使用方正简繁扩展版(即 GBK 版)字体。
%     \item[c] 日文模板才需要日文明朝体与哥特体;日文字体使用 \pkg{fontspec}
%       宏包设置,故日文模板不支持 \pdfTeX{} 引擎,请使用 \XeLaTeX{} 或
%       \LuaLaTeX{} 编译。
%   \end{tablenotes}
% \end{threeparttable}
% \end{table}
%
% \begin{function}{text-font}
%   \begin{syntax}
%     \OPT{text-font} = <(newtx)|times|stixtwo|xits|newpx|cambria|newcm|lm|libertinus|none>
%   \end{syntax}
%   指定西文字体集。\sjtutex{} 预定义了一些西文字体组合,
%   具体配置见表 \ref{tab:latinfonts}。
%   找不到定义的西文字体集时的回退选项为 \opt{newtx}。
% \end{function}
%
% \begin{function}{math-font}
%   \begin{syntax}
%     \OPT{math-font} = <(auto)|(newtx)|times|stixtwo|xits|newpx|cambria|newcm|lm|libertinus|none>
%   \end{syntax}
%   指定数学字体集。\sjtutex{} 预定义了一些数学字体组合,
%   具体配置见表 \ref{tab:latinfonts} 数学字体列。
%   默认跟随西文字体 \opt{text-font} 的设置。
%   找不到定义的数学字体集时的回退选项为 \opt{newtx}。
%
%   \opt{xits},\opt{newcm},\opt{cambria} 选项仅支持 \XeLaTeX/\LuaLaTeX{} 编译。
% \end{function}
%
% \begin{table}[ht]
% \centering\small
% \begin{threeparttable}
%   \caption{西文字体与数学字体配置}
%   \label{tab:latinfonts}
%   \begin{tabular}{ccccc}
%     \toprule
%       & \strong{正文字体} & \strong{无衬线字体} & \strong{等宽字体} & \strong{数学字体} \\
%     \midrule
%       |newtx|                  & TG Termes X\tnote{a}     & TG Heros        & TG Cursor   & newtx           \\
%       \multirow{2}{*}{|times|} & Times New Roman\tnote{b} & Arial           & Courier New & \multirow{2}{*}{mathptmx} \\
%                                & Times\tnote{c}           & Helvetica       & Courier     &                 \\
%       |stixtwo|                & STIX Two Text            & TG Heros        & TG Cursor   & STIX Two Math   \\
%       |xits|                   & XITS                     & TG Heros        & TG Cursor   & XITS Math       \\
%       |newpx|                  & TG Pagella X             & TG Heros        & TG Cursor   & newpx           \\
%       |cambria|                & Cambria                  & Calibri         & Consolas    & Cambria Math    \\
%       |newcm|                  & New CM\tnote{d}          & New CM Sans     & New CM Mono & New CM Math     \\
%       |lm|                     & LM Roman\tnote{e}        & LM Sans         & LM Mono     & LM Math         \\
%       |libertinus|             & Libertinus Serif         & Libertinus Sans & LM Mono     & Libertinus Math \\
%     \bottomrule
%   \end{tabular}
%   \begin{tablenotes}
%     \item[a] “TG”是 TeX Gyre 的缩写。
%     \item[b] 本行中,Times New Roman、Arial 和 Courier New 是商业字体,
%       在 Windows 和 macOS 系统上均默认安装。
%     \item[c] 使用 \pdfTeX{} 引擎时,实际使用对应字体的 Type 1 开源版本。
%     \item[d] “CM”是 Computer Modern 的缩写。
%     \item[e] “LM”是 Latin Modern 的缩写。
%   \end{tablenotes}
% \end{threeparttable}
% \end{table}
%
% \begin{function}[added=2022-12-03,updated=2023-01-05]{math-style}
%   \begin{syntax}
%     \OPT{math-style} = <(ISO)|TeX>
%   \end{syntax}
%   数学符号样式。该选项将影响 \opt{uppercase-greek}、\opt{integral}、
%   \opt{integral-limits} 选项。
%   默认遵循 ISO 80000-2 标准设置,即斜体的大写希腊字母、直立的积分号
%   以及积分号上下限置于上下方。
%   用户也可以逐项修改数学样式。
% \end{function}
%
% \begin{function}[added=2023-01-05]{uppercase-greek}
%   \begin{syntax}
%     \OPT{uppercase-greek} = <slanted|upright>
%   \end{syntax}
%   大写希腊字母的正/斜体。
% \end{function}
%
% \begin{function}[added=2023-01-05]{integral}
%   \begin{syntax}
%     \OPT{integral} = <slanted|upright>
%   \end{syntax}
%   积分号的正/斜体。
% \end{function}
%
% \begin{function}[added=2023-01-05]{integral-limits}
%   \begin{syntax}
%     \OPT{integral-limits} = <true|false>
%   \end{syntax}
%   行间公式中积分号上下限的位置,
%   \opt{true} 使得上下限在积分号上下方,
%   \opt{false} 使得上下限在积分号右侧。
%   该选项只影响行间公式,行内公式统一居右侧,不受影响。
% \end{function}
%
% \subsubsection{其他选项}
%
% 其他的文档类选项会被传递给对应的 \CTeX{} 文档类,
% 请注意 \opt{GBK}、\opt{scheme} 等选项不受 \sjtutex{} 支持。
%
% \subsection{论文信息设置}
%
% \begin{function}{\sjtusetup}
%   \begin{syntax}
%     \TNA{sjtusetup}\marg{键值列表}
%   \end{syntax}
%   本模板提供了一系列选项,可由您自行配置。载入文档类之后,以下
%   所有选项均可通过统一的命令 \tn{sjtusetup} 来设置。
% \end{function}
%
% \tn{sjtusetup} 的参数是一组由(英文)逗号隔开的选项列表,列表中的选项通常是
% \meta{key}|=|\meta{value} 的形式。对于同一项,后面的设置将会覆盖前面的设置。在
% 下文的说明中,将用\textbf{粗体}表示默认值。\tn{sjtusetup} 支持不同类型以及多种
% 层次的选项设定。键值列表中,“|=|”左右的空格不影响设置;但需注意,参数列表中不
% 可以出现空行。
%
% \begin{latex}[moretexcs={\sjtusetup}, emph={[2]info,style,name},
%   emph={[3]zh,en,title,author,float-num-sep,achv}]
% \sjtusetup{
%   info = {
%     zh/title      = {上海交通大学学位论文模板示例文档},
%     en/title      = {A Sample Document for SJTU Thesis Template},
%     zh/author     = {某某},
%     en/author     = {Mo Mo},
%   },
%   style = {
%     float-num-sep = {-},
%   },
%   name = {
%     achv          = {攻读学位期间完成的论文},
%   },
% }
% \end{latex}
%
% \subsubsection{信息域}
%
% \begin{function}[updated=2025-01-14]{info}
%   \begin{syntax}
%     \OPS{info} = \marg{键值列表}
%   \end{syntax}
%   该选项包含许多子项目,用于录入论文信息。具体内容见下。
%   \begin{itemize}
%     \item 在 \cls{sjtuthesis} 文档类中,推荐使用带语言代码前缀 \meta{lang}
%           (比如 \opt{zh} 或 \opt{en})的键来设定对应语言的论文信息,见第
%           \ref{sec:sjtuthesiskey} 节;省略语言前缀不带“*”的项目表示对应的中文
%           字段、带“*”的项目表示对应的英文字段属于老用法,仍然兼容但请及时更新
%           至新用法;各标题页中的信息表格会根据设定信息的顺序进行显示与排列,
%           未设置的信息将不会显示,但在盲审模式开启时会例外地强制显示相关字段,
%           中文标题页会强制显示 \texttt{info/zh} 中的 \texttt{author, id,
%           supervisor, assoc_supervisor, co_supervisor, department, major, degree}
%           字段,其他语言标题页会强制显示 \texttt{info/<lang>} 中的
%           \texttt{author, supervisor, assoc_supervisor, co_supervisor} 字段,
%           这种顺序也是这些信息的推荐设置顺序。
%     \item 在 \cls{sjtureport} 和 \cls{sjtuarticle} 文档类中,不需要使用语言代
%           码前缀。此时推荐直接使用标准接口来设定这些信息,这些标准接口不属于键
%           值列表,应当直接写在导言区内,见第 \ref{sec:sjtureportcmd} 节。
%   \end{itemize}
% \end{function}
%
% \paragraph{适用于 \cls{sjtuthesis} 文档类的键}
% \label{sec:sjtuthesiskey}
%
% \begin{function}[rEXP,updated=2023-03-14]{info/<lang>/title}
%   \begin{syntax}
%     \OPT{\meta{lang}}/\OPT{title} = \marg{标题}
%   \end{syntax}
%   标题。
% \end{function}
%
% \begin{function}[rEXP,updated=2023-03-14]{info/<lang>/display-title}
%   \begin{syntax}
%     \OPT{\meta{lang}}/\OPT{display-title} = \marg{标题页标题}
%   \end{syntax}
%   标题页中的题目。默认为跟随对应语言的标题。
%   如果标题过长,可以尝试使用“|\\|”手动断行。
% \end{function}
%
% \begin{function}[rEXP,added=2022-12-17,updated=2023-03-14]{info/<lang>/subject}
%   \begin{syntax}
%     \OPT{\meta{lang}}/\OPT{subject} = \marg{主题}
%   \end{syntax}
%   文档主题。一般显示在中文标题页校徽下方。
%   默认值类似于 “上海交通大学学士学位论文” 或 “A Dissertation Submitted to
%   Shanghai Jiao Tong University for the Degree of Bachelor”。
% \end{function}
%
% \begin{function}[rEXP,updated=2023-03-14]{info/<lang>/keywords}
%   \begin{syntax}
%     \OPT{\meta{lang}}/\OPT{keywords} = \marg{中文关键字}
%   \end{syntax}
%   关键字列表。各关键字之间需使用英文逗号隔开。
%   为防止歧义,可以用分组括号“|{...}|”把各字段括起来。
% \end{function}
%
% \begin{function}[rEXP,updated=2023-03-14]{info/<lang>/author}
%   \begin{syntax}
%     \OPT{\meta{lang}}/\OPT{author} = \marg{姓名}
%   \end{syntax}
%   作者姓名。
% \end{function}
%
% \begin{function}[rEXP]{info/id}
%   \begin{syntax}
%     \OPT{id} = \marg{学号}
%   \end{syntax}
%   学号。该键不需要语言前缀。
% \end{function}
%
% \begin{function}[rEXP,updated=2023-03-14]{info/<lang>/supervisor,
%   info/<lang>/assoc-supervisor,info/<lang>/co-supervisor}
%   \begin{syntax}
%     \OPT{\meta{lang}}/\OPT{supervisor}       = \marg{导师姓名}
%     \OPT{\meta{lang}}/\OPT{assoc-supervisor} = \marg{副导师姓名}
%     \OPT{\meta{lang}}/\OPT{co-supervisor}    = \marg{联合导师姓名}
%   \end{syntax}
%   导师、副导师、联合导师姓名。
% \end{function}
%
% \begin{function}[rEXP,updated=2023-03-14]{info/<lang>/department}
%   \begin{syntax}
%     \OPT{\meta{lang}}/\OPT{department} = \marg{院系名称}
%   \end{syntax}
%   院系名称。
% \end{function}
%
% \begin{function}[rEXP,updated=2023-03-14]{info/<lang>/major}
%   \begin{syntax}
%     \OPT{\meta{lang}}/\OPT{major} = \marg{专业名称}
%   \end{syntax}
%   专业名称。
% \end{function}
%
% \begin{function}[rEXP,updated=2023-03-14]{info/<lang>/degree}
%   \begin{syntax}
%     \OPT{\meta{lang}}/\OPT{degree} = \marg{学位名称}
%   \end{syntax}
%   申请学位中英文名称。
%   包括申请的学位类别和级别,如“工学硕士”、“理学博士”等。
% \end{function}
%
% \begin{function}[rEXP,updated=2023-03-14]{info/<lang>/fund}
%   \begin{syntax}
%     \OPT{\meta{lang}}/\OPT{fund} = \marg{资助基金名称}
%   \end{syntax}
%   资助基金列表。各资助基金名称之间需使用英文逗号隔开。
%   为防止歧义,可以用分组括号“|{...}|”把各字段括起来。
% \end{function}
%
% \begin{function}[rEXP,updated=2023-02-25]{info/date}
%   \begin{syntax}
%     \OPT{date} = \marg{ISO 日期}
%   \end{syntax}
%   日期。默认值为文档编译日期。也可以自己指定,要求使用 ISO 格式,
%   即 |yyyy-mm-dd| 或 |yyyy-mm|,否则设定无效。该键语言前缀不是必须的。
% \end{function}
%
% \begin{function}[rEXP,updated=2023-03-14]{info/<lang>/display-date}
%   \begin{syntax}
%     \OPT{\meta{lang}}/\OPT{display-date} = \marg{日期文字}
%   \end{syntax}
%   显示日期,可以显示不同于标准日期格式的日期,日期文字将会被原样输出。
%   设定该键时,将会覆盖 \opt{info/date} 键在对应语言下的设定。
% \end{function}
%
% \begin{function}[rEXP,added=2025-01-14]{info/<lang>/custom}
%   \begin{syntax}
%     \OPT{\meta{lang}}/\OPT{custom}/\OPT{\meta{变量名}} = \marg{第一列内容}\marg{第二列内容}
%   \end{syntax}
%   对应语言标题页信息表格中显示的额外信息,可以设置多个变量。
%   需要使用一个变量名作为索引键(变量名应当由英文字母、数字、
%   连字符 \texttt{-} 组成,且第一个位置只能为英文字母);
%   并赋予两个参数来设置信息表格中两列的内容:
%   通常第一列作为该信息的属性名称会被加粗,第二列作为该信息的内容。
%   该信息会根据在 \cmd{\sjtusetup} 中的设置相对位置显示在对应的位置上。
%   当 \opt{review} 选项开启时,该选项的设置不会生效。
% \end{function}
%
% \paragraph{适用于 \cls{sjtureport} 和 \cls{sjtuarticle} 文档类的命令}
% \label{sec:sjtureportcmd}
%
% \begin{function}[EXP,added=2022-12-17,updated=2023-03-14]{\title}
%   \begin{syntax}
%     \TNA{title}\marg{标题}
%   \end{syntax}
%   设置标题,覆盖 \opt{info/title} 键的值。
% \end{function}
%
% \begin{function}[EXP,added=2022-12-17,updated=2023-03-14]{\author}
%   \begin{syntax}
%     \TNA{author}\marg{姓名}
%   \end{syntax}
%   设置作者姓名,覆盖 \opt{info/author} 键的值。
% \end{function}
%
% \begin{function}[EXP,added=2022-12-17,updated=2023-03-14]{\date}
%   \begin{syntax}
%     \TNA{date}\marg{日期}
%   \end{syntax}
%   设置日期,覆盖 \opt{info/display-date} 键的值。
%   日期会被原样显示。
% \end{function}
%
% \begin{function}[EXP,added=2023-03-14]{\subject}
%   \begin{syntax}
%     \TNA{subject}\marg{主题}
%   \end{syntax}
%   文档主题。覆盖 \opt{info/subject} 键的值。
% \end{function}
%
% \begin{function}[EXP,added=2023-03-14]{\keywords}
%   \begin{syntax}
%     \TNA{keywords}\marg{关键词}
%   \end{syntax}
%   文档关键词,使用英文逗号隔开不同的关键词。覆盖 \opt{info/keywords} 键的值。
% \end{function}
%
% \subsubsection{样式域}
%
% \begin{function}{style}
%   \begin{syntax}
%     \OPS{style} = \marg{键值列表}
%   \end{syntax}
%   该选项包含许多子项目,用于设置论文样式。具体内容见下。
% \end{function}
%
% \begin{function}[added=2024-01-10]{style/indent-first}
%   \begin{syntax}
%     \OPT{indent-first} = <(true)|false>
%   \end{syntax}
%   章节标题后首段是否缩进。
% \end{function}
%
% \begin{function}[added=2023-11-30]{style/equation-font}
%   \begin{syntax}
%     \OPT{equation-font} = \marg{字体设置}
%   \end{syntax}
%   行间数学公式的字体设置,该选项主要用于调整行间公式的行距,
%   不建议修改字号字形。
%   \cls{sjtuthesis} 中默认数学公式的行距为字号的 1.2 倍。
% \end{function}
%
% \begin{function}[added=2022-12-03,updated=2022-12-27]{style/float-font}
%   \begin{syntax}
%     \OPT{float-font} = \marg{字体设置}
%   \end{syntax}
%   图、表等浮动体的额外字体设置。
%   默认为 |\zihao{5}|,五号字。
% \end{function}
%
% \begin{function}[added=2022-12-20]{style/caption-font}
%   \begin{syntax}
%     \OPT{caption-font} = \marg{字体设置}
%   \end{syntax}
%   题注字体。
%   默认为 |\zihao{5}\bfseries|,粗体五号字。
% \end{function}
%
% \begin{function}[added=2022-12-20]{style/subcaption-font}
%   \begin{syntax}
%     \OPT{subcaption-font} = \marg{字体设置}
%   \end{syntax}
%   子图题注字体。
%   默认为 |\zihao{5}\normalfont|,正常字重五号字。
% \end{function}
%
% \begin{function}[added=2024-03-21]{style/theorem-header-font}
%   \begin{syntax}
%     \OPT{theorem-header-font} = \marg{字体设置}
%   \end{syntax}
%   定理头(即标题)字体。
%   默认为 |\bfseries\CJKsffamily|,黑体加粗。
% \end{function}
%
% \begin{function}[added=2024-03-21]{style/theorem-body-font}
%   \begin{syntax}
%     \OPT{theorem-body-font} = \marg{字体设置}
%   \end{syntax}
%   定理内容字体。
%   默认为 |\normalfont|,和正文字体相同。
% \end{function}
%
% \begin{function}[added=2023-03-28]{style/fnmark-style}
%   \begin{syntax}
%     \OPT{fnmark-style} = <plain|circled>
%   \end{syntax}
%   脚注数字编号样式。
%   \opt{plain} 表示使用普通数字编号;
%   \opt{circled} 表示使用带圈数字编号。
%   在 \opt{zh} 和 \opt{ja} 语言设置中,默认为 \opt{circled};
%   在 \opt{en} 和 \opt{de} 语言设置中,默认为 \opt{plain}。
%
%   使用带圈数字编号时,由于超过 50 的带圈数字没有对应的 Unicode 码位,
%   所以每页脚注最好不要超过 50 个。
%   带圈数字默认使用 CJK 字体。通常情况下默认字体不一定包含所有带圈数字的字符,
%   此时可以设置 \opt{fnmark-font} 选项给带圈数字设置合适的字体。
% \end{function}
%
% \begin{function}[added=2022-12-03,updated=2023-03-28]{style/fnmark-font}
%   \begin{syntax}
%     \OPT{fnmark-font} = <haranoaji|\marg{字体设置}>
%   \end{syntax}
%   脚注编号的额外字体设置。
%   默认为空。
%   可以使用预设 \opt{haranoaji},支持在 Unicode 引擎中使用 HaranoAjiMincho 字体
%   中的带圈数字。
% \end{function}
%
% \begin{function}[added=2023-12-02]{style/num-sep}
%   \begin{syntax}
%     \OPT{num-sep} = \marg{分隔符}
%   \end{syntax}
%   图、表、公式以及定理编号中的分隔符。该选项将统一设定 \opt{float-num-sep}、
%   \opt{equation-num-sep}、\opt{theorem-num-sep} 选项。
%   用户也可以逐项修改编号分隔符。
%   默认为 \opt{.} 句点。
% \end{function}
%
% \begin{function}[updated=2023-11-29]{style/float-num-sep}
%   \begin{syntax}
%     \OPT{float-num-sep} = \marg{分隔符}
%   \end{syntax}
%   图、表等浮动体编号中的分隔符。
% \end{function}
%
% \begin{function}[updated=2023-11-29]{style/equation-num-sep}
%   \begin{syntax}
%     \OPT{equation-num-sep} = \marg{分隔符}
%   \end{syntax}
%   公式编号中的分隔符。
% \end{function}
%
% \begin{function}[added=2023-12-02]{style/theorem-num-sep}
%   \begin{syntax}
%     \OPT{theorem-num-sep} = \marg{分隔符}
%   \end{syntax}
%   定理编号中的分隔符。
% \end{function}
%
% \begin{function}[added=2022-12-20,updated=2023-03-14]{style/header-uppercase}
%   \begin{syntax}
%     \OPT{header-uppercase} = <true|(false)>
%   \end{syntax}
%   页眉英文字母是否大写。默认为 \opt{false}。
% \end{function}
%
% \begin{function}[added=2022-12-20]{style/header-font}
%   \begin{syntax}
%     \OPT{header-font} = \marg{页眉字体}
%   \end{syntax}
%   页眉字体。
%   \begin{itemize}
%     \item 在 \cls{sjtuthesis} 文档类中,默认为 |\zihao{-5}|,小五号字。
%     \item 在 \cls{sjtureport} 和 \cls{sjtuarticle} 文档类中,默认为 |\zihao{-5}\sffamily|,小五号字黑体。
%   \end{itemize}
% \end{function}
%
% \begin{function}[added=2022-12-20]{style/footer-font}
%   \begin{syntax}
%     \OPT{footer-font} = \marg{页脚字体}
%   \end{syntax}
%   页脚字体。默认为 |\zihao{-5}|,小五号字。
% \end{function}
%
% \begin{function}[added=2022-12-03,updated=2024-12-30]{style/page-number}
%   \begin{syntax}
%     \OPT{page-number} = \marg{格式定义}
%   \end{syntax}
%   设置页码的显示格式,\meta{格式定义} 有两个参数:参数 |#1| 是当前页码,参数 |#2| 是当前编码的总页码数。
%   获取总页码数需要使用 \pkg{pageslts} 宏包。
%   默认为 |{#1}|,即仅显示当前页码。
% \end{function}
%
% \begin{function}[added=2023-11-30]{style/keywords-format}
%   \begin{syntax}
%     \OPT{keywords-format} = <(plain)|hang>
%   \end{syntax}
%   设置关键词格式。默认为 \opt{plain} 无缩进的普通段落,另可选 \opt{hang} 悬挂格式。
% \end{function}
%
% \subsubsection{名称域}
%
% \begin{function}{name}
%   \begin{syntax}
%     \OPS{name} = \marg{键值列表}
%   \end{syntax}
%   选项包含许多子项目,用于设置论文中一些标题的名称。部分选项只能在 \cls{sjtuthesis} 中使用。
%   具体内容见表 \ref{tab:names}。
% \end{function}
%
% \begin{function}[updated=2023-03-18]{name/contents,name/listfigure,name/listtable,
%   name/figure,name/table,name/abstract,name/index,name/appendix,name/proof,name/bib,
%   name/figure*,name/table*,name/algorithm,name/listalgorithm,
%   name/abbr,name/nom,name/ack,name/resume,name/digest,name/achv}
% \end{function}
%
% \begingroup
% \vspace{-48ex}
% \small \tabcolsep=3pt
% \begin{longtable}{l|p{7em}|>{\raggedright\arraybackslash}p{9em}|p{11em}|p{8em}}
%   \caption{\opt{name} 选项的默认设置} \label{tab:names} \\
%   \toprule
%     \strong{选项}        & \strong{|lang = zh|}   & \strong{|lang = en|}          & \strong{|lang = de|}   & \strong{|lang = ja|} \\
%   \midrule
%   \endfirsthead
%     \multicolumn{5}{r}{续表~\thetable} \\
%   \toprule
%     \strong{选项}        & \strong{|lang = zh|}   & \strong{|lang = en|}          & \strong{|lang = de|}   & \strong{|lang = ja|} \\
%   \midrule
%   \endhead
%     \bottomrule
%     \multicolumn{5}{r}{续下页}
%   \endfoot
%     \bottomrule
%   \endlastfoot
%     |contents|           & 目录                   & Contents                      & Inhaltsverzeichnis     & 目次                 \\
%     |listfigure|         & 插图                   & List of Figures               & Abbildungsverzeichnis  & 図目次               \\
%     |listtable|          & 表格                   & List of Tables                & Tabellenverzeichnis    & 表目次               \\
%     |figure|             & 图                     & Figure                        & Abbildung              & 図                   \\
%     |table|              & 表                     & Table                         & Tabelle                & 表                   \\
%     |abstract| \expstar  & 摘要                   & Abstract                      & Zusammenfassung        & 概要                 \\
%     |index|              & 索引                   & Index                         & Index                  & 索引                 \\
%     |appendix|           & 附录                   & Appendix                      & Anhang                 & 付録                 \\
%     |proof|              & 证明                   & Proof                         & Beweis                 & 证明                 \\
%     |bib|                & 参考文献               & Bibliography                  & Literaturverzeichnis   & 参考文献             \\
%   \midrule
%     |figure*|            & Figure                 & 图                            & Figure                 & Figure               \\
%     |table*|             & Table                  & 表                            & Table                  & Table                \\
%     |algorithm|          & 算法                   & Algorithm                     & Algorithmus            & アルゴリズム         \\
%     |listalgorithm|      & 算法                   & List of Algorithms            & Algorithmenverzeichnis & アルゴリズム目次     \\
%     |abbr| \rexpstar     & 缩略语对照表           & Abbreviation                  & Abkürzungsverzeichnis  & 略語表               \\
%     |nom| \rexpstar      & 主要符号对照表         & Nomenclature                  & Symbolverzeichnis      & 記号表               \\
%     |ack| \rexpstar      & 致谢                   & Acknowledgements              & Danksagungen           & 謝辞                 \\
%     |resume| \rexpstar   & 个人简历               & Resume                        & Lebenslauf             & 履歴書               \\
%     |digest| \rexpstar   & 大摘要                 & Digest                        & Kurzfassung            & 要約                 \\
%     |achv| \rexpstar     & 学术论文和科研成果目录 & List of Research Achievements & Forschungsleistungen   & 研究業績書           \\
% \end{longtable}
% \endgroup
%
% \section{内容编写}
%
% \begin{function}{document}
%   \begin{syntax}
%     \BEV{document}
%     \  \meta{文档内容}
%     \EEV{document}
%   \end{syntax}
%   在文档开始后进行内容编写,文档内容由 |document| 环境包裹。
% \end{function}
%
% \subsection{用于本节环境或命令的参数}
%
% 本节描述的部分环境和命令提供 \meta{标题参数},用于进行局部的标题相关设置。
% 可以使用逗号分隔的选项列表作为可选参数传入该环境,例如:
% \begin{latex}
% \begin{acknowledgements}[intoc,title={自定义标题}]
% \end{acknowledgements}
% \end{latex}
%
% \begin{function}[added=2025-01-12]{intoc, notintoc}
%   指明该环境的小标题是否在目录中出现。
%   \opt{intoc} 表示出现在目录中,
%   \opt{notintoc} 表示不出现在目录中。
%   默认 \tn{mainmatter} 之前的环境为 \opt{notintoc},
%   \tn{mainmatter} 之后的环境为 \opt{intoc}。
% \end{function}
%
% \begin{function}[added=2025-01-12]{title}
%   \begin{syntax}
%     \OPT{title} = \marg{小标题}
%   \end{syntax}
%   设置该环境的小标题,默认是全局语言下的小标题。
% \end{function}
%
% \begin{function}[added=2025-01-12]{marking}
%   \begin{syntax}
%     \OPT{marking} = \marg{目录中的小标题}
%   \end{syntax}
%   设置该环境在目录中的小标题,默认与小标题相同。
%   只有在设置 \opt{intoc} 小标题出现在目录中才会显示。
% \end{function}
%
% 部分环境还提供了参数 \opt{lang},用于设置环境的语言。
% \begin{function}[added=2025-01-12,label=]{lang}
%   \begin{syntax}
%     \OPT{lang} = <(zh)|en|de|ja>
%   \end{syntax}
%   设置该环境的局部语言选项,默认与全局语言选项相同。
%   该选项会设置该环境的局部语言排版设置。
% \end{function}
%
% 对于同时提供了 \meta{标题参数} 和 \opt{lang} 参数的环境,二者可以一同使用,例如:
% \begin{latex}
% \begin{abstract}[lang=en,intoc,title={自定义标题},marking={目录中的标题}]
% \end{abstract}
% \end{latex}
%
% \subsection{标题页和声明页}
%
% \begin{function}[updated=2022-12-03]{\maketitle}
%   \begin{syntax}
%     \TNA{maketitle}
%   \end{syntax}
%   生成标题。
%   \begin{itemize}
%     \item 在 \cls{sjtuthesis} 文档类中,生成标题页。
%     \item 在 \cls{sjtureport} 和 \cls{sjtuarticle} 文档类中,
%           \begin{itemize}
%             \item 若处于 \opt{titlepage} 文档类选项中,生成标题页。\cls{sjtureport} 文档类默认。
%             \item 若处于 \opt{notitlepage} 文档类选项中,生成标题块。\cls{sjtuarticle} 文档类默认。
%           \end{itemize}
%   \end{itemize}
% \end{function}
%
% \begin{function}[rEXP,updated=2025-03-22]{\copyrightpage}
%   \begin{syntax}
%     \TNA{copyrightpage}
%     \TNA{copyrightpage}\oarg{file=授权书扫描件,variant=<a|b>,额外选项}
%   \end{syntax}
%   生成空白版权使用授权书。
%   接受可选参数 \opt{file} 用于插入版权使用授权书扫描件,插入外置 PDF 页时需要手动加载 \pkg{pdfpages} 宏包。
%   接受可选参数 \opt{variant} 用于选择版权使用授权书的文本变种,\opt{a} 为硕士及博士学位论文的默认变种,\opt{b} 为学士学位论文的默认变种。
%   额外选项将传递给 \tn{includepdf} 命令。
% \end{function}
%
% \subsection{前文部分}
%
% \begin{function}[rEXP]{\frontmatter}
%   \begin{syntax}
%     \TNB{frontmatter}
%   \end{syntax}
%   声明前文部分开始。
% \end{function}
%
% \begin{function}[updated=2024-12-20]{abstract}
%   \begin{syntax}
%     \BEV{abstract}\oarg{lang=<(zh)|en|de|ja>,标题参数}
%     \  \meta{\textup{\cls{sjtuthesis}} 中的摘要}
%     \EEV{abstract}
%     \BEV{abstract}
%     \  \meta{\textup{\cls{sjtuarticle}} 和 \textup{\cls{sjtureport}} 中的摘要}
%     \EEV{abstract}
%   \end{syntax}
%   摘要环境。会在结尾添加关键词。
%   \begin{itemize}
%     \item \cls{sjtuthesis} 文档类中,可以设置可选参数,
%           指定摘要的局部语言会同步更改小标题为对应语言设置。
%     \item \cls{sjtuarticle} 和 \cls{sjtureport} 文档类中,不设置可选参数。
%   \end{itemize}
% \end{function}
%
% \begin{function}[updated=2024-12-20]{\tableofcontents,\listoffigures,\listoftables,\listofalgorithms}
%   \begin{syntax}
%     \TNA{tableofcontents}\oarg{标题参数}
%     \TNA{listoffigures}\oarg{标题参数}
%     \TNA{listoftables}\oarg{标题参数}
%     \TNA{listofalgorithms}\oarg{标题参数}
%   \end{syntax}
%   目录、插图、表格和算法等索引命令如表 \ref{tab:list} 所示,将其插入到期望的
%   位置即可,这些命令可以使用可选的 \meta{标题参数} 来调整标题相关设置。
%   \begin{table}[H]
%     \centering\small
%     \begin{threeparttable}
%       \caption{目录和索引表}
%       \label{tab:list}
%       \begin{tabular}{lp{4cm}lp{4cm}}
%         \toprule
%         \strong{用途}     & \strong{命令}         & \strong{用途}     & \strong{命令}          \\
%         \midrule
%         目录              & \tn{tableofcontents}  & 插图索引          & \tn{listoffigures}     \\
%         \midrule
%         表格索引          & \tn{listoftables}     & 算法索引\tnote{a} & \tn{listofalgorithms}  \\
%         \bottomrule
%       \end{tabular}
%       \begin{tablenotes}
%         \item[a] 启用 \pkg{algorithm2e} 或 \pkg{algorithm} 后有效。
%       \end{tablenotes}
%     \end{threeparttable}
%   \end{table}
% \end{function}
%
% \subsection{正文部分}
%
% \begin{function}[rEXP]{\mainmatter}
%   \begin{syntax}
%     \TNB{mainmatter}
%   \end{syntax}
%   声明正文部分开始。正文部分是论文的核心,您可以分章节撰写。
%   如有需求,也可以采用多文件编译的方式。
% \end{function}
%
% \begin{function}[updated=2022-12-03]{\footnote}
%   \begin{syntax}
%     \TNA{footnote}\oarg{脚注编号}\marg{脚注文字}
%   \end{syntax}
%   插入脚注。其中脚注编号参数是可选的,一般不需要输入。
% \end{function}
%
% \begin{function}{assumption,axiom,conjecture,corollary,definition,example,
%   exercise,lemma,problem,proposition,theorem}
%   \sjtutex{} 预定义了一系列数学环境,如表 \ref{tab:theorems} 所示,
%   在启用 \pkg{amsthm} 或 \pkg{ntheorem} 宏包后有效。
%   \begin{table}[H]
%     \centering\small
%     \caption{预定义的数学环境}
%     \label{tab:theorems}
%     \begin{tabular}{*{7}{l}}
%       \toprule
%       \env{assumption} & \env{axiom}   & \env{conjecture} & \env{corollary}   & \env{definition} & \env{example}  & \env{exercise} \\
%       假设             & 公理          & 猜想             & 推论              & 定义             & 例             & 练习           \\
%       \midrule
%       \env{lemma}      & \env{problem} & \env{proof}      & \env{proposition} & \env{remark}     & \env{solution} & \env{theorem}  \\
%       引理             & 问题          & 证明             & 命题              & 注               & 解             & 定理           \\
%       \bottomrule
%     \end{tabular}
%   \end{table}
% \end{function}
%
% \begin{function}[added=2023-10-24]{\setbaselineskip}
%   \begin{syntax}
%     \TNA{setbaselineskip}\marg{长度}
%   \end{syntax}
%   设置当前的基线间距,一般在字号命令之后使用。
% \end{function}
%
% \begin{function}{\appendix}
%   \begin{syntax}
%     \TNA{appendix}
%   \end{syntax}
%   附录由 \tn{appendix} 命令开启,然后像正文一样书写。
% \end{function}
%
% \begin{function}[rEXP,updated=2024-12-20]{nomenclature}
%   \begin{syntax}
%     \BEV{nomenclature}\oarg{标题参数}
%     \  \meta{符号对照表}
%     \EEV{nomenclature}
%   \end{syntax}
%   符号对照表环境。可以使用 \meta{标题参数} 手动设置标题。
%   符号对照表环境仅设置标题,内部实现可由用户自行决定:
%   可以使用 \pkg{longtable},也可以使用 \pkg{nomencl} 宏包。
% \end{function}
%
% \begin{function}[rEXP,updated=2024-12-20]{abbreviation}
%   \begin{syntax}
%     \BEV{abbreviation}\oarg{标题参数}
%     \  \meta{缩略语对照表}
%     \EEV{abbreviation}
%   \end{syntax}
%   缩略语对照表环境。可以使用可选参数手动设置标题。
%   缩略语对照表环境仅设置标题,内部实现可由用户自行决定。
% \end{function}
%
% \subsection{后文部分}
%
% \begin{function}[rEXP]{\backmatter}
%   \begin{syntax}
%     \TNB{backmatter}
%   \end{syntax}
%   声明后文部分开始。 后文部分包含致谢等。
% \end{function}
%
% \begin{function}[rEXP,updated=2024-12-20]{acknowledgements}
%   \begin{syntax}
%     \BEV{acknowledgements}\oarg{标题参数}
%     \  \meta{致谢内容}
%     \EEV{acknowledgements}
%   \end{syntax}
%   致谢环境。盲审模式下致谢将被隐去。可以使用 \meta{标题参数} 手动设置标题。
% \end{function}
%
% \begin{function}[rEXP,updated=2024-12-20]{achievements}
%   \begin{syntax}
%     \BEV{achievements}\oarg{标题参数}
%     \  \meta{获得的科研成果}
%     \EEV{achievements}
%   \end{syntax}
%   科研成果环境,可以使用 \meta{标题参数} 手动设置标题。内部请配合使用下面的附录用文献列表环境
%   \env{bibliolist} 和 \env{bibliolist*}。你可以在该环境中使用带星号的节次命令
%   以分隔不同的类型的成果(比如学术论文、专利等);你也可以使用多个 \env{achievements}
%   环境,配合不同的 \meta{标题参数},展示不同类型的成果。
% \end{function}
%
% \begin{function}[rEXP,updated=2024-12-08]{bibliolist,bibliolist*}
%   \begin{syntax}
%     \BEV{bibliolist}\oarg{可选参数}\marg{最长条目编号}
%     \  \TNA{item} \meta{文献条目}
%     \EEV{bibliolist}
%     \BEV{bibliolist*}\oarg{可选参数}\marg{最长条目编号}
%     \  \TNA{item} \meta{文献条目(隐去姓名)}
%     \EEV{bibliolist*}
%   \end{syntax}
%   用于展示科研成果的文献列表,比如学术论文、专利等。
%   需要指定最长条目的编号作为参数,比如 \opt{99};
%   如果将该强制参数被指定为空,将不显示编号,每条以悬挂缩进做区分。
%   可以使用可选参数 \opt{resume} 设置连续编号。
%   环境内部使用 \tn{item} 来分隔各条目。
%   普通模式下显示 \env{bibliolist} 中的内容,盲审模式下显示 \env{bibliolist*} 中的内容。
% \end{function}
%
% \begin{function}[rEXP,updated=2024-12-20]{resume}
%   \begin{syntax}
%     \BEV{resume}\oarg{标题参数}
%     \  \meta{简历内容}
%     \EEV{resume}
%   \end{syntax}
%   简历环境。盲审模式下简历将被隐去。可以使用 \meta{标题参数} 手动设置标题。
% \end{function}
%
% \begin{function}[rEXP,updated=2024-12-20]{digest}
%   \begin{syntax}
%     \BEV{digest}\oarg{lang=<zh|(en)|de|ja>,标题参数}
%     \  \meta{大摘要}
%     \EEV{digest}
%   \end{syntax}
%   学士论文大摘要,可以设置可选参数,指定大摘要的局部语言会同步更改小标题
%   为对应语言设置,默认为 \opt{lang=en}。
% \end{function}
%
% \section{宏包依赖情况}
%
% 使用不同编译方式、指定不同选项,会导致宏包依赖情况有所不同。
% 具体如下:
% \begin{itemize}
%   \item 在任何情况下,文档类都会\emph{显式}调用以下宏包
%     (或文档类):
%     \begin{itemize}
%       \item \cls{ctexbook}、\cls{ctexrep} 和 \cls{ctexart},
%         提供中文排版的通用框架。属于 \CTeX{} 宏集 \cite{CTeX}。
%       \item \pkg{array},提供表格环境的增强功能。
%       \item \pkg{mathtools},对 \LaTeX{} 的数学排版功能进行了全面扩展。
%         是 \pkg{amsmath} 的扩充。
%       \item \pkg{geometry},用于调整页面尺寸。
%       \item \pkg{fancyhdr},处理页眉页脚。
%       \item \pkg{titletoc},设置目录格式。
%       \item \pkg{caption}、\cls{bicaption} 和 \cls{subcaption},用于设置题注。
%       \item \pkg{graphicx},提供图形插入的接口。
%       \item \pkg{enumitem},设置列表环境格式。
%     \end{itemize}
%   \item 部分西文与数学字体预设会调用相关的字体宏包,具体调用情况请参见对应的
%     字体预设文件。
%   \item 部分数字字体预设会调用 \pkg{unicode-math} 处理 Unicode 编码的
%     OpenType 数学字体。在未启用 \pkg{unicode-math} 的情况下,会调用 \pkg{bm}
%     来选择粗体数学符号。
%   \item 当需要显示总页码数时,会调用 \pkg{pageslts} 宏包。
% \end{itemize}
%
% 这里只列出了本模板直接调用的宏包。这些宏包自身的调用情况,
% 此处不再具体展开。如有需要,请参阅相关文档。
%
% \begin{thebibliography}{99}
%
% \addcontentsline{toc}{section}{\refname}
%
% \newcommand\urlprefix{\newline\hspace*{\fill}}
% \let\OldUrl=\url
% \renewcommand\url[2][]{{\small\textit{#1}~\OldUrl{#2}}}
% \newcommand\CTANurl[2][]{{^^A
%   \small\textit{#1}~\href{https://mirror.ctan.org/#2}{\texttt{CTAN://#2}}}}
%
% \subsection*{图书}
%
% \bibitem{Knuth1986}
% \textsc{Knuth D E}.
% \newblock \textit{The \TeX book: Computers \& Typesetting, volume A}\allowbreak[M].
% \newblock Boston: Addison--Wesley Publishing Company, 1986.
% \urlprefix \CTANurl[源代码^^A
%   \footnote{此代码只可作为学习之用。未经 Knuth 本人同意,您不应当编译此文档。}:]^^A
%   {systems/knuth/dist/tex/texbook.tex}
%
% \bibitem{MittelbachF2023}
% \textsc{Mittelbach F} and \textsc{Fischer U}.
% \newblock \textit{The \LaTeX{} Companion}\allowbreak[M].
% \newblock 3rd ed.
% \newblock Boston: Addison--Wesley Publishing Company, 2023.
%
% \bibitem{LiuHY2013}
% 刘海洋.
% \newblock \textit{\LaTeX{} 入门}\allowbreak[M].
% \newblock 北京: 电子工业出版社, 2013.
%
% \subsection*{标准、规范}
%
% \bibitem{gb-t-7713.1-2006}
% 国务院学位委员会办公室, 全国信息与文献标准化技术委员会.
% \newblock \textit{学位论文编写规则: GB/T 7713.1--2006}\allowbreak[S].
% \newblock 北京: 中国标准出版社, 2007.
%
% \bibitem{cy-t-35-2001}
% 全国信息与文献标准化技术委员会第七分委员会, 中华人民共和国新闻出版总署.
% \newblock \textit{科技文献的章节编号方法: CY/T 35--2001}\allowbreak[S].
% \newblock [S.l. : s.n.], 2001.
%
% \bibitem{SJTUGS2023}
% 上海交通大学研究生院.
% \newblock \textit{上海交通大学博士、硕士学位论文撰写指南}\allowbreak[EB/OL].
% \newblock (2023-11-03)[2023-12-04].
% \urlprefix\url{https://www.gs.sjtu.edu.cn/post/detail/Z3MxNDc=}
%
% \subsection*{宏包、模版}
%
% \bibitem{source2e}
% \textsc{Braams J}, \textsc{Carlisle D}, \textsc{Jeffrey A}, et~al.
% \newblock \textit{The \LaTeXe{} Sources}\allowbreak[CP/OL].
% \newblock (2024-11-01).
% \urlprefix\url{https://ctan.org/pkg/latex}
% \urlprefix\CTANurl[源代码:]{macros/latex/base/source2e.pdf}
%
% \bibitem{interface3}
% \textsc{The \LaTeX{} Project}.
% \newblock \textit{The \LaTeXiii{} Interfaces}\allowbreak[EB/OL].
% \newblock (2024-11-02).
% \urlprefix\url{https://ctan.org/pkg/l3kernel}
% \urlprefix\CTANurl[文档:]{macros/latex/l3kernel/interface3.pdf}
%
% \bibitem{CTeX}
% \textsc{CTEX.ORG}.
% \newblock \textit{\CTeX{} 宏集手册}\allowbreak[EB/OL].
% \newblock version 2.5.10,
% \newblock (2022-07-14).
% \urlprefix\url{https://ctan.org/pkg/ctex}
% \urlprefix\CTANurl[文档及源代码:]{language/chinese/ctex/ctex.pdf}
%
% \bibitem{zhlineskip}
% 张瑞熹.
% \newblock \textit{zhlineskip 宏包}\allowbreak[EB/OL].
% \newblock version 1.0e,
% \newblock (2019-05-15).
% \urlprefix\url{https://ctan.org/pkg/zhlineskip}
% \urlprefix\CTANurl[文档:]{language/chinese/zhlineskip/zhlineskip.pdf}
%
% \bibitem{fduthesis}
% 曾祥东.
% \newblock \textit{fduthesis: 复旦大学论文模板}\allowbreak[EB/OL].
% \newblock version 0.9a,
% \newblock (2023-05-27).
% \urlprefix\url{https://ctan.org/pkg/fduthesis}
% \urlprefix\CTANurl[文档及源代码:]{macros/latex/contrib/fduthesis/fduthesis-code.pdf}
%
% \bibitem{thuthesis}
% 清华大学 TUNA 协会.
% \newblock \textit{\textsc{ThuThesis}:清华大学学位论文模板}\allowbreak[EB/OL].
% \newblock version 7.5.2,
% \newblock (2025-07-01).
% \urlprefix\url{https://ctan.org/pkg/thuthesis}
% \urlprefix\CTANurl[文档及源代码:]{macros/latex/contrib/thuthesis/thuthesis.pdf}
%
% \end{thebibliography}
%
% \end{documentation}
%
% \begin{implementation}
%
% \clearpage
% \section{代码实现}
%
% \changes{v2.1}{2023/12/05}{同步 \LaTeX{} 2020-10-01,无需显式调用
% \pkg{expl3} 和 \pkg{xparse} 宏包。}
% \changes{v2.2}{2024/11/22}{同步 \LaTeX{} 2024-06-01,无需显式调用
% \pkg{xtemplate} 宏包。}
% \changes{v2.2}{2025/02/10}{重命名视觉形象系统文件名称。}
% 本模板使用 \LaTeXiii{} 语法编写,依赖 \pkg{expl3} 环境,
% 并需调用 \pkg{l3packages} 中的相关宏包。
%
% \subsection{前置准备}
%
% 目前 \pkg{sjtutex} 文档类仅支持 \XeLaTeX{}、\LuaLaTeX{} 和 \pdfLaTeX{} 编译。
%    \begin{macrocode}
%<@@=sjtu>
%<*class>
\msg_new:nnn { sjtutex } { unsupported-engine }
  {
    The~sjtutex~classes~does~NOT~support~'#1'~engine. \\\\
    Please~set~your~typesetting~program~to~either~
    'xelatex'~or~'lualatex'.
  }
\bool_lazy_any:nF
  {
    { \sys_if_engine_xetex_p:  }
    { \sys_if_engine_luatex_p: }
    { \sys_if_engine_pdftex_p: }
  }
  { \msg_fatal:nnV { sjtutex } { unsupported-engine } \c_sys_engine_str }
%    \end{macrocode}
%
% \subsection{内部定义}
%
% \begin{variable}{\l_@@_tmp_tl,\l_@@_tmp_clist,\l_@@_tmp_box}
% 临时变量。
%    \begin{macrocode}
\tl_new:N \l_@@_tmp_tl
\clist_new:N \l_@@_tmp_clist
%<thesis>\box_new:N \l_@@_tmp_box
%    \end{macrocode}
% \end{variable}
%
% \begin{macro}[int]{\clist_use:Nv,\clist_use:cv,
% \exp_args:NNnv,\exp_last_unbraced:ce}
% \LaTeX3{} 函数变体。
%    \begin{macrocode}
\cs_generate_variant:Nn \clist_use:Nn { Nv, cv }
\exp_args_generate:n { Nnv }
\cs_generate_variant:Nn \exp_last_unbraced:Ne { ce }
%    \end{macrocode}
% \end{macro}
%
% \begin{macro}[int]{\@@_conditional_const:nn}
%    \begin{macrocode}
\cs_new_protected:Npn \@@_conditional_const:nn #1#2
  {
    \bool_if:nTF {#2}
      {
        \cs_new_eq:cN { @@_if_ #1 :T  } \use:n
        \cs_new_eq:cN { @@_if_ #1 :F  } \use_none:n
        \cs_new_eq:cN { @@_if_ #1 :TF } \use_i:nn
        \cs_new_eq:cN { @@_if_ #1 _p: } \c_true_bool
      }
      {
        \cs_new_eq:cN { @@_if_ #1 :T  } \use_none:n
        \cs_new_eq:cN { @@_if_ #1 :F  } \use:n
        \cs_new_eq:cN { @@_if_ #1 :TF } \use_ii:nn
        \cs_new_eq:cN { @@_if_ #1 _p: } \c_false_bool
      }
  }
%    \end{macrocode}
% \end{macro}
%
% \begin{macro}[int,pTF]{\@@_if_engine_opentype:}
%    \begin{macrocode}
\@@_conditional_const:nn { engine_opentype }
  {
    \bool_lazy_or_p:nn
      { \sys_if_engine_xetex_p:  }
      { \sys_if_engine_luatex_p: }
  }
%    \end{macrocode}
% \end{macro}
%
% \begin{macro}[int]{\@@_disable_package_load:n}
%    \begin{macrocode}
\msg_new:nnn { sjtutex } { package-disabled }
  { The~loading~of~package~'#1'~is~disabled. }
\cs_new_protected:Npn \@@_disable_package_load:n #1
  {
    \disable@package@load {#1}
      { \msg_warning:nnn { sjtutex } { package-disabled } {#1} }
  }
%    \end{macrocode}
% \end{macro}
%
% \changes{v2.0.2}{2023/03/31}{区分 \texttt{dim} 与 \texttt{skip} 类型变量。}
% \begin{macro}[int]{\@@_dim_set_to_wd:Nn}
% 操作长度变量的辅助函数。
%    \begin{macrocode}
%<*thesis>
\cs_new:Npn \@@_dim_set_to_wd:Nn #1#2
  {
    \hbox_set:Nn \l_@@_tmp_box {#2}
    \dim_set:Nn #1 { \box_wd:N \l_@@_tmp_box }
  }
\cs_generate_variant:Nn \@@_dim_set_to_wd:Nn { Nv }
%</thesis>
%    \end{macrocode}
% \end{macro}
%
% \begin{macro}[int]{\@@_cs_provide_eq:NN}
%    \begin{macrocode}
\cs_new_protected:Npn \@@_cs_provide_eq:NN #1#2
  { \cs_if_exist:NF #1 { \cs_set_eq:NN #1 #2 } }
\cs_generate_variant:Nn \@@_cs_provide_eq:NN { cc }
%    \end{macrocode}
% \end{macro}
%
% \begin{macro}[int]{\@@_name_set:nnn,\@@_name_const:nnn,
% \@@_name_const_from_clist:nnnn}
% 定义默认名称的辅助函数。
%    \begin{macrocode}
\cs_new_protected:Npn \@@_name_set:nnn #1#2#3
  { \tl_set:cn   { l_@@_name_ #2 _ #1 _tl } {#3} }
\cs_new_protected:Npn \@@_name_const:nnn #1#2#3
  { \tl_const:cn { c_@@_name_ #2 _ #1 _tl } {#3} }
%<*thesis>
\cs_new_protected:Npn \@@_name_const_from_clist:nnnn #1#2#3#4
  { \tl_const:ce { c_@@_name_ #2 _ #1 _tl } { \clist_item:nn {#4} {#3} } }
%</thesis>
%    \end{macrocode}
% \end{macro}
%
% \subsection{文档选项}
%
% \begin{variable}{\g_@@_thesis_type_int}
% 论文类型。
%    \begin{macrocode}
%<thesis>\int_new:N \g_@@_thesis_type_int
%    \end{macrocode}
% \end{variable}
%
% \begin{variable}{\g_@@_lang_tl,\g_@@_lang_clist}
% 论文主要语言。
%    \begin{macrocode}
\tl_new:N \g_@@_lang_tl
%<thesis>\clist_set:Nn \g_@@_lang_clist { zh, en }
%    \end{macrocode}
% \end{variable}
%
% \begin{variable}{\g_@@_baseline_skip_dim,\g_@@_fixed_baselineskip_bool}
% 固定行距。
%    \begin{macrocode}
\dim_new:N \g_@@_baseline_skip_dim
\bool_new:N \g_@@_fixed_baselineskip_bool
%    \end{macrocode}
% \end{variable}
%
% \begin{variable}{\g_@@_text_font_tl,\g_@@_math_font_tl,\g_@@_cjk_font_tl}
% 字体配置。
%    \begin{macrocode}
\tl_new:N \g_@@_text_font_tl
\tl_new:N \g_@@_math_font_tl
\tl_new:N \g_@@_cjk_font_tl
%    \end{macrocode}
% \end{variable}
%
% \begin{variable}{\g_@@_slanted_uppercase_greek_bool}
% 大写希腊字母的正/斜体。
%    \begin{macrocode}
\bool_new:N \g_@@_slanted_uppercase_greek_bool
%    \end{macrocode}
% \end{variable}
%
% \begin{variable}{\g_@@_upright_integral_bool}
% 积分号的正/斜体。
%    \begin{macrocode}
\bool_new:N \g_@@_upright_integral_bool
%    \end{macrocode}
% \end{variable}
%
% \begin{variable}{\g_@@_integral_limits_bool}
% 积分号上下限的位置。
%    \begin{macrocode}
\bool_new:N \g_@@_integral_limits_bool
%    \end{macrocode}
% \end{variable}
%
% \begin{variable}{\g_@@_math_font_options_clist}
% 传入数学字体宏包的选项列表。
%    \begin{macrocode}
\clist_new:N \g_@@_math_font_options_clist
%    \end{macrocode}
% \end{variable}
%
% \begin{variable}{\g_@@_review_bool}
% 盲审模式。
%    \begin{macrocode}
%<thesis>\bool_new:N \g_@@_review_bool
%    \end{macrocode}
% \end{variable}
%
% \begin{variable}{\g_@@_options_to_ctex_class_clist}
% 保存由 \pkg{sjtutex} 传入 \pkg{ctex} 文档类的选项列表。
% 默认 \pkg{ctex} 文档类的选项:
% 使用 UTF8 编码,不调整基础类的版式以及不载入 \pkg{ctex} 字体预设配置。
% \cls{sjtuthesis} 和 \cls{sjtureport} 文档类默认字号为小四号。
% \cls{sjtureport} 和 \cls{sjtuarticle} 文档类默认使用 1.3 行距倍数。
%    \begin{macrocode}
\clist_set:Nn \g_@@_options_to_ctex_class_clist
  {
    UTF8,
%<!article>    zihao = -4,
%<!thesis>    linespread = 1.3,
    scheme = plain,
    fontset = none
  }
%    \end{macrocode}
% \end{variable}
%
% \begin{variable}{\g_@@_options_to_packages_clist}
% 保存由传入其他宏包的选项列表。
%    \begin{macrocode}
\clist_set:Nn \g_@@_options_to_packages_clist
  {
    { no-math           } { fontspec     } ,
    { list = off        } { bicaption    } ,
    { warnings-off =
      {
        mathtools-overbracket,
        mathtools-colon
      }
    }                     { unicode-math } ,
%<!article>    { chapter           } { algorithm    } ,
%<!article>    { algochapter       } { algorithm2e  } ,
    { amsmath, thmmarks } { ntheorem     }
  }
%    \end{macrocode}
% \end{variable}
%
% \begin{macro}[int]{\@@_set_deprecated_option:n}
% 对过时选项给出警告。
%    \begin{macrocode}
\msg_new:nnn { sjtutex } { deprecated-option }
  { Option~'\l_keys_key_str'~is~deprecated! \\ #1 }
\cs_new_protected:Npn \@@_set_deprecated_option:n #1
  {
    \msg_warning:nnn { sjtutex } { deprecated-option } { Option~'#1'~is~set. }
    \keys_set:nn { sjtu / option } {#1}
  }
%    \end{macrocode}
% \end{macro}
%
%    \begin{macrocode}
\msg_new:nnn { sjtutex } { unsupported-option }
  { Option~'\l_keys_key_str'~is~not~supported! \\ Will~be~ignored. }
%    \end{macrocode}
%
% 定义 |sjtu/option| 键值类。
%    \begin{macrocode}
\keys_define:nn { sjtu / option }
  {
%    \end{macrocode}
%
% \changes{v2.0}{2022/10/25}{\cls{sjtuthesis} 类型选项移除 \opt{course},
% 不再支持课程论文。}
% \begin{macro}{type}
% 论文类型。
%    \begin{macrocode}
%<*thesis>
    type .choice: ,
    type .value_required:n = true ,
    type .choices:nn =
      { bachelor, master, doctor }
      { \int_gset_eq:NN \g_@@_thesis_type_int \l_keys_choice_int } ,
    type .initial:n = { master } ,
%</thesis>
%    \end{macrocode}
% \end{macro}
%
% \changes{v2.0}{2023/03/19}{语言选项新增 \opt{ja},添加日文模板。}
% \changes{v2.0}{2023/03/23}{语言选项新增 \opt{de},添加德文模板。}
% \begin{macro}{lang}
% 论文主要语言。
%    \begin{macrocode}
    lang .choice: ,
    lang .value_required:n = true ,
    lang .choices:nn =
      { zh, en, de, ja }
      {
        \tl_gset_eq:NN \g_@@_lang_tl \l_keys_choice_tl
%<*thesis>
        \clist_if_in:NnF \g_@@_lang_clist {#1}
          { \clist_gput_right:Nn \g_@@_lang_clist {#1} }
%</thesis>
      } ,
    lang .initial:n = { zh } ,
%    \end{macrocode}
% \end{macro}
%
% \changes{v2.0}{2022/12/03}{添加 \opt{lineskip} 文档类选项。}
% \changes{v2.1}{2023/10/24}{添加 \opt{baselineskip} 文档类选项,
% 替换原 \opt{lineskip} 选项。}
% \begin{macro}{baselineskip}
% 正文基线间距。
%    \begin{macrocode}
    baselineskip .choice: ,
    baselineskip .value_required:n = true ,
    baselineskip / false   .code:n =
      { \bool_gset_false:N \g_@@_fixed_baselineskip_bool } ,
    baselineskip / unknown .code:n =
      {
        \bool_gset_true:N  \g_@@_fixed_baselineskip_bool
        \dim_gset:Nn \g_@@_baseline_skip_dim {#1}
      } ,
%<thesis>    baselineskip .initial:n = { 20 bp } ,
%<!thesis>    baselineskip .initial:n = { false } ,
%    \end{macrocode}
% \end{macro}
%
% \begin{macro}{text-font,math-font,cjk-font}
% 字体配置。
%    \begin{macrocode}
    text-font .tl_gset:N = \g_@@_text_font_tl ,
    text-font .initial:n = { newtx } ,
    math-font .tl_gset:N = \g_@@_math_font_tl ,
    cjk-font  .tl_gset:N = \g_@@_cjk_font_tl ,
%    \end{macrocode}
% \end{macro}
%
% \opt{fontset} 等是过时选项。
%    \begin{macrocode}
    fontset    .code:n = { \@@_set_deprecated_option:n { cjk-font = #1 } } ,
    nofonts    .code:n = { \@@_set_deprecated_option:n { cjk-font = none } } ,
    adobefonts .code:n = { \@@_set_deprecated_option:n { cjk-font = adobe } } ,
    winfonts   .code:n = { \@@_set_deprecated_option:n { cjk-font = windows } } ,
%    \end{macrocode}
%
% \changes{v2.0}{2023/01/05}{添加 \opt{math-style} 文档类选项,默认值为 \opt{ISO}。}
% \begin{macro}{math-style}
% 数学符号样式。
%    \begin{macrocode}
    math-style .choice: ,
    math-style .value_required:n = true,
    math-style / TeX .code:n =
      {
        \bool_gset_false:N \g_@@_slanted_uppercase_greek_bool
        \bool_gset_false:N \g_@@_upright_integral_bool
        \bool_gset_false:N \g_@@_integral_limits_bool
      } ,
    math-style / ISO .code:n =
      {
        \bool_gset_true:N \g_@@_slanted_uppercase_greek_bool
        \bool_gset_true:N \g_@@_upright_integral_bool
        \bool_gset_true:N \g_@@_integral_limits_bool
      } ,
    math-style .initial:n = { ISO } ,
%    \end{macrocode}
% \end{macro}
%
% \begin{macro}{uppercase-greek}
% 大写希腊字母的正/斜体。
%    \begin{macrocode}
    uppercase-greek .choice: ,
    uppercase-greek .value_required:n = true ,
    uppercase-greek / slanted .code:n =
      { \bool_gset_true:N  \g_@@_slanted_uppercase_greek_bool } ,
    uppercase-greek / upright .code:n =
      { \bool_gset_false:N \g_@@_slanted_uppercase_greek_bool } ,
%    \end{macrocode}
% \end{macro}
%
% \begin{macro}{integral}
% 积分号的正/斜体。
%    \begin{macrocode}
    integral .choice: ,
    integral .value_required:n = true ,
    integral / slanted .code:n =
      { \bool_gset_false:N \g_@@_upright_integral_bool } ,
    integral / upright .code:n =
      { \bool_gset_true:N  \g_@@_upright_integral_bool } ,
%    \end{macrocode}
% \end{macro}
%
% \begin{macro}{integral-limits}
% 积分号上下限的位置。
%    \begin{macrocode}
    integral-limits .choice: ,
    integral-limits .value_required:n = true ,
    integral-limits / false .code:n =
      { \bool_gset_false:N \g_@@_integral_limits_bool } ,
    integral-limits / true  .code:n =
      { \bool_gset_true:N  \g_@@_integral_limits_bool } ,
%    \end{macrocode}
% \end{macro}
%
% \begin{macro}{review}
% 盲审模式。
%    \begin{macrocode}
%<thesis>    review .bool_gset:N = \g_@@_review_bool ,
%<thesis>    review .initial:n = false ,
%    \end{macrocode}
% \end{macro}
%
% 将未知选项传给标准文档类。
%    \begin{macrocode}
    unknown .code:n =
      { \clist_gput_right:No \g_@@_options_to_ctex_class_clist { \CurrentOption } }
  }
%    \end{macrocode}
%
% 屏蔽不支持的 \pkg{ctex} 选项。
%    \begin{macrocode}
\clist_map_inline:nn
  { GBK, scheme, cap, nocap }
  {
    \keys_define:nn { sjtu / option }
      {
        #1 .code:n =
          { \msg_warning:nn { sjtutex } { unsupported-option } }
      }
  }
%    \end{macrocode}
%
% 将文档类选项传给 |sjtu/option|。
%    \begin{macrocode}
\ProcessKeyOptions [ sjtu / option ]
%    \end{macrocode}
%
% \begin{macro}[int,pTF]{\@@_if_main_lang_de:,\@@_if_main_lang_ja:}
% 快捷判断主要语言。
%    \begin{macrocode}
\@@_conditional_const:nn { main_lang_de }
  { \str_if_eq_p:Vn \g_@@_lang_tl { de } }
\@@_conditional_const:nn { main_lang_ja }
  { \str_if_eq_p:Vn \g_@@_lang_tl { ja } }
%    \end{macrocode}
% \end{macro}
%
% 数字字体宏包选项。
%    \begin{macrocode}
\bool_if:NT \g_@@_slanted_uppercase_greek_bool
  { \clist_put_right:Nn \g_@@_math_font_options_clist { slantedGreek } }
\bool_if:NT \g_@@_upright_integral_bool
  { \clist_put_right:Nn \g_@@_math_font_options_clist { upint } }
%    \end{macrocode}
%
% 追加选项。
%    \begin{macrocode}
\clist_put_left:Nn \@classoptionslist { a4paper }
\@@_if_main_lang_de:T
  { \clist_put_right:Nn \@classoptionslist { german, ngerman } }
\bool_if:NTF \g_@@_integral_limits_bool
  {
    \clist_put_right:Nn \@classoptionslist { intlimits }
    \clist_put_right:Nn \g_@@_options_to_packages_clist
      { { displaylimits } { cmupint } }
  }
  {
    \clist_put_right:Nn \g_@@_options_to_packages_clist
      { { nolimits      } { cmupint } }
  }
\clist_concat:NNN \@classoptionslist
  \@classoptionslist \g_@@_math_font_options_clist
%    \end{macrocode}
%
% 内部设置接口。
%    \begin{macrocode}
%<*thesis>
\cs_set_protected:Npe \@@_setup:n
  {
    \bool_if:NTF \g_@@_review_bool
      { \exp_not:N \keys_set_exclude_groups:nnn { sjtu } { sensitive } }
      { \exp_not:N \keys_set:nn { sjtu } }
  }
%</thesis>
%    \end{macrocode}
%
%
% \subsection{载入宏包、文档类}
%
% 将选项传入 \pkg{ctex} 文档类。
%    \begin{macrocode}
\exp_args:No \PassOptionsToClass
  { \g_@@_options_to_ctex_class_clist }
%<thesis>  { ctexbook }
%<report>  { ctexrep }
%<article>  { ctexart }
%    \end{macrocode}
%
% 传入各宏包选项。
%    \begin{macrocode}
\clist_map_inline:Nn \g_@@_options_to_packages_clist
  { \PassOptionsToPackage #1 }
%    \end{macrocode}
%
% 载入 \pkg{ctex} 文档类。
% 在使用 \XeLaTeX{} 编译时,\pkg{ctex} 的底层将调用 \pkg{xeCJK}
% 宏包;而在使用 \LuaLaTeX{} 编译时,则将调用 \pkg{LuaTeX-ja} 宏包。
% 两种情况下 \pkg{ctex} 均会调用 \pkg{fontspec} 宏包。
%    \begin{macrocode}
%<thesis>\LoadClass { ctexbook }
%<report>\LoadClass { ctexrep }
%<article>\LoadClass { ctexart }
%    \end{macrocode}
%
% \changes{v2.2.1}{2025/03/27}{不再自动载入 \pkg{xcolor} 宏包。}
% 载入各宏包。
%    \begin{macrocode}
\RequirePackage
  {
    array,
    mathtools,
    geometry,
    fancyhdr,
    titletoc,
    caption,
    bicaption,
    subcaption,
    graphicx,
    enumitem
  }
%    \end{macrocode}
%
% \subsection{文档格式}
%
% \subsubsection{字号行距}
%
% \begin{macro}[int]{\SJTU@orig@normalsize}
% 保存原始 \tn{normalsize} 字号大小。
%    \begin{macrocode}
\normalsize
\cs_set_protected:Npe \SJTU@orig@normalsize
  { \exp_not:N \fontsize { \f@size } { \f@baselineskip } \exp_not:N \selectfont }
%    \end{macrocode}
% \end{macro}
%
% \begin{macro}[int]{\@@_set_font_size:nnNn}
% \begin{macro}{\normalsize}
% 重定义 \tn{normalsize},设置正文的基线间距。
%    \begin{macrocode}
\msg_new:nnn { sjtutex } { baselineskip-invalid }
  { Invalid~baselineskip~value~'#1'! \\ Using~default~value~instead. }
\cs_new_protected:Npn \@@_set_font_size:nnNn #1#2#3#4
  { \cs_set_protected:Npn #3 { \@setfontsize #3 {#1} {#2} #4 } }
\bool_if:NT \g_@@_fixed_baselineskip_bool
  {
    \dim_compare:nNnTF { \f@size pt } > { \g_@@_baseline_skip_dim }
      {
        \msg_warning:nne { sjtutex } { baselineskip-invalid }
          { \dim_to_decimal:n { \g_@@_baseline_skip_dim } pt }
      }
      {
        \tl_set:Ne \l_@@_font_size_tl
          { { \f@size } { \dim_to_decimal:n { \g_@@_baseline_skip_dim } } }
        \int_case:nn { \@ptsize }
          {
            { 0 }
              {
                \exp_after:wN \@@_set_font_size:nnNn \l_@@_font_size_tl
                  \normalsize
                  {
                    \abovedisplayskip 10\p@ \@plus2\p@ \@minus5\p@
                    \abovedisplayshortskip \z@ \@plus3\p@
                    \belowdisplayshortskip 6\p@ \@plus3\p@ \@minus3\p@
                    \belowdisplayskip \abovedisplayskip
                    \let\@listi\@listI
                  }
              }
            { 1 }
              {
                \exp_after:wN \@@_set_font_size:nnNn \l_@@_font_size_tl
                  \normalsize
                  {
                    \abovedisplayskip 11\p@ \@plus3\p@ \@minus6\p@
                    \abovedisplayshortskip \z@ \@plus3\p@
                    \belowdisplayshortskip 6.5\p@ \@plus3.5\p@ \@minus3\p@
                    \belowdisplayskip \abovedisplayskip
                    \let\@listi\@listI
                  }
              }
            { 2 }
              {
                \exp_after:wN \@@_set_font_size:nnNn \l_@@_font_size_tl
                  \normalsize
                  {
                    \abovedisplayskip 12\p@ \@plus3\p@ \@minus7\p@
                    \abovedisplayshortskip \z@ \@plus3\p@
                    \belowdisplayshortskip 6.5\p@ \@plus3.5\p@ \@minus3\p@
                    \belowdisplayskip \abovedisplayskip
                    \let\@listi\@listI
                  }
              }
          }
        \normalsize
      }
  }
%    \end{macrocode}
% \end{macro}
% \end{macro}
%
% \begin{macro}{\setbaselineskip}
% 设置基线间距,在字号命令之后使用。
%    \begin{macrocode}
\NewDocumentCommand \setbaselineskip { m }
  { \fontsize { \f@size } {#1} \selectfont }
%    \end{macrocode}
% \end{macro}
%
% \subsubsection{字体配置}
%
% \begin{macro}[int]{\@@_fontset_error:nn}
% 字库不可用时给出紧急错误信息,停止读取定义文件。
%    \begin{macrocode}
\msg_new:nnn { sjtutex } { font-unavailable }
  { '#1-font~=~#2'~is~unavailable~in~current~mode. }
\cs_new_protected:Npn \@@_fontset_error:nn #1#2
  { \msg_critical:nnnn { sjtutex } { font-unavailable } {#1} {#2} }
\cs_set_protected:Npn \ctex_fontset_error:n #1
  { \msg_critical:nnnn { sjtutex } { font-unavailable } { cjk } {#1} }
%    \end{macrocode}
% \end{macro}
%
% \begin{macro}[int]{\@@_declare_math_symbol:nnNn}
%    \begin{macrocode}
\cs_new_protected:Nn \@@_declare_math_symbol:nnNn
  {
    \cs_undefine:N #3
    \DeclareMathSymbol {#3} {#1} {#2} {#4}
  }
%    \end{macrocode}
% \end{macro}
%
% \begin{macro}[int]{\@@_set_slanted_greek:}
%    \begin{macrocode}
\cs_new_protected:Nn \@@_set_slanted_greek:
  {
    \clist_const:Nn \c_@@_uppercase_greek_clist
      { Gamma, Delta, Theta, Lambda, Xi, Pi, Sigma, Upsilon, Phi, Psi, Omega }
    \clist_map_inline:Nn \c_@@_uppercase_greek_clist
      {
        \cs_set_eq:cc { up ##1 } {     ##1 }
        \cs_set_eq:cc { it ##1 } { var ##1 }
      }
    \bool_if:NT \g_@@_slanted_uppercase_greek_bool
      {
        \clist_map_inline:Nn \c_@@_uppercase_greek_clist
          { \cs_set_eq:cc { ##1 } { it ##1 } }
      }
  }
%    \end{macrocode}
% \end{macro}
%
% \begin{macro}[int]{\@@_set_unimath_symbol:}
%    \begin{macrocode}
\cs_new_protected:Nn \@@_set_unimath_symbol:
  {
    \clist_map_inline:nn
      {
        { increment } { upDelta     } ,
        { QED       } { blacksquare }
      }
      { \@@_cs_provide_eq:cc ##1 }
  }
%    \end{macrocode}
% \end{macro}
%
% 如果没有指定数学字体,则根据西文字体设置匹配的数字字体。
%    \begin{macrocode}
\tl_if_empty:NT \g_@@_math_font_tl
  { \tl_gset_eq:NN \g_@@_math_font_tl \g_@@_text_font_tl }
%    \end{macrocode}
%
% 根据操作系统判断默认 CJK 字体配置。
%    \begin{macrocode}
\tl_if_empty:NT \g_@@_cjk_font_tl
  {
    \sys_if_platform_windows:TF
      { \tl_gset:Nn \g_@@_cjk_font_tl { windows } }
      {
        \ctex_if_platform_macos:TF
          { \tl_gset:Nn \g_@@_cjk_font_tl { mac    } }
          { \tl_gset:Nn \g_@@_cjk_font_tl { fandol } }
      }
  }
%    \end{macrocode}
%
% \begin{macro}[int]{\@@_load_font:nn,\@@_load_fontset:}
% 如果字体配置文件不存在,则载入默认值,并给出警告。
%    \begin{macrocode}
\msg_new:nnn { sjtutex } { invalid-font }
  {
    Invalid~value~'#1-font~=~\tl_use:c { g_@@_ #1 _font_tl }'! \\
    Using~'#2'~instead.
  }
\cs_new_protected:Npn \@@_load_font:nn #1#2
  {
    \str_if_eq:vnF { g_@@_ #1 _font_tl } { none }
      {
        \file_if_exist:nF
          { sjtu- #1 -font- \tl_use:c { g_@@_ #1 _font_tl } .def }
          {
            \msg_warning:nnnn { sjtutex } { invalid-font } {#1} {#2}
            \tl_gset:cn { g_@@_ #1 _font_tl } {#2}
          }
        \ctex_file_input:n
          { sjtu- #1 -font- \tl_use:c { g_@@_ #1 _font_tl } .def }
      }
  }
\cs_new_protected:Nn \@@_load_fontset:
  {
    \clist_map_inline:nn
      {
        { math } { newtx  },
        { text } { newtx  },
        { cjk  } { fandol }
      }
      { \@@_load_font:nn ##1 }
  }
\@onlypreamble \@@_load_font:nn
\@onlypreamble \@@_load_fontset:
%</class>
%    \end{macrocode}
% \end{macro}
%
% \paragraph{西文与数学字体}
%
% \changes{v2.0.3}{2023/09/25}{新增 \opt{libertinus} 字体配置。}
%    \begin{macrocode}
%<*font&(math|text)>
%<*math&type1>
%<*newtx|newpx>
\tl_set_eq:NN \l_@@_save_encodingdefault_tl \encodingdefault
\tl_set_eq:NN \l_@@_save_rmdefault_tl \rmdefault
\tl_set_eq:NN \l_@@_save_sfdefault_tl \sfdefault
\tl_set_eq:NN \l_@@_save_ttdefault_tl \ttdefault
\tl_set:Nn \encodingdefault { OT1 }
%<newtx>\tl_set:Nn \rmdefault { ntxtlf }
%<newpx>\tl_set:Nn \rmdefault { zplTLF }
\tl_set:Nn \qhv@scale { 0.94 }
\tl_set:Nn \sfdefault { qhv }
\tl_set:Nn \ttdefault { qcr }
%<newtx>\RequirePackage { newtxmath }
%<newpx>\RequirePackage { newpxmath }
\tl_set_eq:NN \encodingdefault \l_@@_save_encodingdefault_tl
\tl_set_eq:NN \rmdefault \l_@@_save_rmdefault_tl
\tl_set_eq:NN \sfdefault \l_@@_save_sfdefault_tl
\tl_set_eq:NN \ttdefault \l_@@_save_ttdefault_tl
%</newtx|newpx>
%<times>\PassOptionsToPackage { Symbol } { upgreek }
%<lm|times>\RequirePackage { amssymb, upgreek }
%<*lm>
\SetSymbolFont { operators    } { normal } { OT1 } { lmr  } { m  } { n  }
\SetSymbolFont { letters      } { normal } { OML } { lmm  } { m  } { it }
\SetSymbolFont { symbols      } { normal } { OMS } { lmsy } { m  } { n  }
\SetSymbolFont { largesymbols } { normal } { OMX } { lmex } { m  } { n  }
\SetSymbolFont { operators    } { bold   } { OT1 } { lmr  } { bx } { n  }
\SetSymbolFont { letters      } { bold   } { OML } { lmm  } { b  } { it }
\SetSymbolFont { symbols      } { bold   } { OMS } { lmsy } { b  } { n  }
\SetSymbolFont { largesymbols } { bold   } { OMX } { lmex } { m  } { n  }
\SetMathAlphabet { \mathbf } { normal } { OT1 } { lmr  } { bx } { n  }
\SetMathAlphabet { \mathsf } { normal } { OT1 } { lmss } { m  } { n  }
\SetMathAlphabet { \mathit } { normal } { OT1 } { lmr  } { m  } { it }
\SetMathAlphabet { \mathtt } { normal } { OT1 } { lmtt } { m  } { n  }
\SetMathAlphabet { \mathbf } { bold   } { OT1 } { lmr  } { bx } { n  }
\SetMathAlphabet { \mathsf } { bold   } { OT1 } { lmss } { bx } { n  }
\SetMathAlphabet { \mathit } { bold   } { OT1 } { lmr  } { bx } { it }
\SetMathAlphabet { \mathtt } { bold   } { OT1 } { lmtt } { m  } { n  }
\@@_set_slanted_greek:
%</lm>
%<*times>
\tl_set_eq:NN \l_@@_save_rmdefault_tl \rmdefault
  \RequirePackage { mathptmx }
\tl_set_eq:NN \rmdefault \l_@@_save_rmdefault_tl
\tl_set:Nn \Hv@scale { 0.94 }
\DeclareMathAlphabet { \mathsf } { OT1 } { phv } { m } { n }
\DeclareMathAlphabet { \mathtt } { OT1 } { pcr } { m } { n }
\SetMathAlphabet { \mathsf } { bold } { OT1 } { phv } { b } { n }
\SetMathAlphabet { \mathtt } { bold } { OT1 } { pcr } { b } { n }
\DeclareSymbolFont { SJTU@ptm } { OML } { ptmcm } { m } { it }
\@@_declare_math_symbol:nnNn { \mathord } { SJTU@ptm } \upvarsigma { "26 }
%</times>
%<lm|times>\bool_if:NT \g_@@_upright_integral_bool
%<lm|times>  { \RequirePackage { cmupint } }
\@@_set_unimath_symbol:
%</math&type1>
%<*!(math&type1)>
%<lm>\@@_if_engine_opentype:F
%<*!lm>
\@@_if_engine_opentype:TF
  {
%<*math>
    \RequirePackage { unicode-math }
%<*stixtwo>
    \bool_if:NTF \g_@@_upright_integral_bool
      {
        \setmathfont { STIXTwoMath-Regular.otf }
          [ StylisticSet = 8 ]
      }
      { \setmathfont { STIXTwoMath-Regular.otf } }
    \setmathfont { STIXTwoMath-Regular.otf }
      [
        range        = { scr, bfscr },
        StylisticSet = 1
      ]
%</stixtwo>
%<*libertinus>
    \bool_if:NTF \g_@@_upright_integral_bool
      { \setmathfont { LibertinusMath-Regular.otf } }
      {
        \setmathfont { LibertinusMath-Regular.otf }
          [ StylisticSet = 8 ]
      }
    \setmathfont { latinmodern-math.otf } [ range = \checkmark ]
%</libertinus>
%<cambria>    \setmathfont { Cambria~Math }
%<*newcm>
    \bool_if:NTF \g_@@_upright_integral_bool
      {
        \setmathfont { NewCMMath-Book.otf }
          [ StylisticSet = 2 ]
      }
      { \setmathfont { NewCMMath-Book.otf } }
    \setmathfont { NewCMMath-Book.otf }
      [
        range        = { scr, bfscr },
        StylisticSet = 1
      ]
%</newcm>
%<*xits>
    \bool_if:NTF \g_@@_upright_integral_bool
      {
        \setmathfont { XITSMath-Regular }
          [
            Extension    = .otf,
            BoldFont     = XITSMath-Bold,
            StylisticSet = 8
          ]
      }
      {
        \setmathfont { XITSMath-Regular }
          [
            Extension    = .otf,
            BoldFont     = XITSMath-Bold,
          ]
      }
    \setmathfont { XITSMath-Regular.otf }
      [
        range        = { cal, bfcal },
        StylisticSet = 1
      ]
%</xits>
%</math>
%<*newtx|newpx|stixtwo|xits>
%<math>    \setmathrm
%<text>    \setmainfont
%<newtx>      { TeXGyreTermesX }
%<newpx>      { TeXGyrePagellaX }
%<stixtwo>      { STIXTwoText }
%<xits>      { XITS }
      [
        Extension      = .otf,
        UprightFont    = *-Regular,
        BoldFont       = *-Bold,
        ItalicFont     = *-Italic,
        BoldItalicFont = *-BoldItalic
      ]
%<math>    \setmathsf
%<text>    \setsansfont
      { texgyreheros }
      [
        Extension      = .otf,
        UprightFont    = *-regular,
        BoldFont       = *-bold,
        ItalicFont     = *-italic,
        BoldItalicFont = *-bolditalic,
        Scale          = 0.94
      ]
%<math>    \setmathtt
%<text>    \setmonofont
      { texgyrecursor }
      [
        Extension      = .otf,
        UprightFont    = *-regular,
        BoldFont       = *-bold,
        ItalicFont     = *-italic,
        BoldItalicFont = *-bolditalic,
        Ligatures      = CommonOff
      ]
%</newtx|newpx|stixtwo|xits>
%<*text&times>
    \setmainfont { Times~New~Roman } [ Ligatures = Rare ]
    \setsansfont { Arial } [ Scale = 0.94 ]
    \setmonofont { Courier~New }
%</text&times>
%<*libertinus>
%<math>    \setmathrm
%<text>    \setmainfont
      { LibertinusSerif }
      [
        Extension           = .otf,
        UprightFont         = *-Regular,
        BoldFont            = *-Bold,
        ItalicFont          = *-Italic,
        BoldItalicFont      = *-BoldItalic,
        SlantedFont         = *-Regular,
        BoldSlantedFont     = *-Bold,
        SlantedFeatures     = { FakeSlant = 0.2 },
        BoldSlantedFeatures = { FakeSlant = 0.2 }
      ]
%<math>    \setmathsf
%<text>    \setsansfont
      { LibertinusSans }
      [
        Extension           = .otf,
        UprightFont         = *-Regular,
        BoldFont            = *-Bold,
        ItalicFont          = *-Italic,
        BoldItalicFont      = *-Italic,
        BoldItalicFeatures  = { FakeBold  = 3   },
        SlantedFont         = *-Regular,
        BoldSlantedFont     = *-Bold,
        SlantedFeatures     = { FakeSlant = 0.2 },
        BoldSlantedFeatures = { FakeSlant = 0.2 }
      ]
%</libertinus>
%<*cambria>
%<*math>
    \setmathrm { Cambria }
    \setmathsf { Calibri }
    \setmathtt { Consolas } [ Scale = 0.95 ]
%</math>
%<*text>
    \setmainfont { Cambria }
    \setsansfont { Calibri }
    \setmonofont { Consolas } [ Scale = 0.95 ]
%</text>
%</cambria>
%<*newcm>
%<math>    \setmathrm
%<text>    \setmainfont
      { NewCM10 }
      [
        Extension    = .otf,
        SizeFeatures =
          {
            {
              Size        = -9,
              Font        = NewCM08-Book,
              ItalicFont  = NewCM08-BookItalic,
              SlantedFont = NewCM08-Book,
            },
            { Size        = 9- }
          },
        UprightFont         = *-Book,
        BoldFont            = *-Bold,
        ItalicFont          = *-BookItalic,
        BoldItalicFont      = *-BoldItalic,
        SlantedFont         = *-Book,
        SlantedFeatures     = { FakeSlant = 0.25 },
        BoldSlantedFont     = *-Bold,
        BoldSlantedFeatures = { FakeSlant = 0.25 }
      ]
%<math>    \setmathsf
%<text>    \setsansfont
      { NewCMSans10 }
      [
        Extension    = .otf,
        SizeFeatures =
          {
            {
              Size       = -9,
              Font       = NewCMSans08-Book,
              ItalicFont = NewCMSans08-BookOblique,
            },
            { Size       = 9- }
          },
        UprightFont    = *-Book,
        BoldFont       = *-Bold,
        ItalicFont     = *-BookOblique,
        BoldItalicFont = *-BoldOblique
      ]
%<math>    \setmathtt
%<text>    \setmonofont
      { NewCMMono10 }
      [
        Extension           = .otf,
        UprightFont         = *-Book,
        BoldFont            = *-Bold,
        ItalicFont          = *-BookItalic,
        BoldItalicFont      = *-BoldOblique,
        SlantedFont         = *-Book,
        SlantedFeatures     = { FakeSlant = 0.25 },
        BoldSlantedFont     = *-Bold,
        BoldSlantedFeatures = { FakeSlant = 0.25 }
      ]
%</newcm>
  }
%</!lm>
%<*text&!otf>
  {
    \tl_set:Nn \encodingdefault { T1 }
%<lm>    \tl_set:Nn \rmdefault { lmr  }
%<lm>    \tl_set:Nn \sfdefault { lmss }
%<libertinus>    \tl_set:Nn \rmdefault { LibertinusSerif-TLF }
%<libertinus>    \tl_set:Nn \sfdefault { LibertinusSans-TLF  }
%<lm|libertinus>    \tl_set:Nn \ttdefault { lmtt }
%<newtx>    \PassOptionsToPackage { nohelv, nott } { newtxtext }
%<newpx>    \PassOptionsToPackage { nohelv, nott } { newpxtext }
%<newtx>    \RequirePackage { newtxtext }
%<newpx>    \RequirePackage { newpxtext }
%<*stixtwo>
    \DeclareEncodingSubset { TS1 } { ? } { 0 }
    \UndeclareTextCommand { \textpertenthousand } { T1 }
    \DeclareTextSymbolDefault { \textpertenthousand } { TS1 }
    \tl_set:Nn \rmdefault { stix2 }
%</stixtwo>
%<newtx|newpx|stixtwo>    \tl_set:Nn \qhv@scale { 0.94 }
%<newtx|newpx|stixtwo>    \tl_set:Nn \sfdefault { qhv }
%<newtx|newpx|stixtwo>    \tl_set:Nn \ttdefault { qcr }
%<*times>
    \tl_set:Nn \rmdefault { ptm }
    \tl_set:Nn \Hv@scale { 0.94 }
    \tl_set:Nn \sfdefault { phv }
    \tl_set:Nn \ttdefault { pcr }
%</times>
  }
%</text&!otf>
%<*text&otf>
%<cambria>  { \@@_fontset_error:nn { text } { cambria } }
%<newcm>  { \@@_fontset_error:nn { text } { newcm } }
%<xits>  { \@@_fontset_error:nn { text } { xits } }
%</text&otf>
%<*math&!otf>
  {
%<*libertinus>
    \exp_args:No \PassOptionsToPackage
      { \g_@@_math_font_options_clist } { libertinust1math }
    \RequirePackage { libertinust1math }
%</libertinus>
%<*stixtwo>
    \DeclareSizeFunction { sub } { \sub@sfcnt \@font@info }
    \PassOptionsToPackage { notext } { stix2 }
    \RequirePackage { stix2 }
    \clist_map_inline:nn
      {
        \upalpha      { "0B } ,
        \upbeta       { "0C } ,
        \upgamma      { "0D } ,
        \updelta      { "0E } ,
        \upepsilon    { "0F } ,
        \upzeta       { "10 } ,
        \upeta        { "11 } ,
        \uptheta      { "12 } ,
        \upiota       { "13 } ,
        \upkappa      { "14 } ,
        \uplambda     { "15 } ,
        \upmu         { "16 } ,
        \upnu         { "17 } ,
        \upxi         { "18 } ,
        \uppi         { "19 } ,
        \uprho        { "1A } ,
        \upsigma      { "1B } ,
        \uptau        { "1C } ,
        \upupsilon    { "1D } ,
        \upphi        { "1E } ,
        \upchi        { "1F } ,
        \uppsi        { "20 } ,
        \upomega      { "21 } ,
        \upvarepsilon { "22 } ,
        \upvartheta   { "23 } ,
        \upvarpi      { "24 } ,
        \upvarrho     { "25 } ,
        \upvarsigma   { "26 } ,
        \upvarphi     { "27 }
      }
      { \@@_declare_math_symbol:nnNn { \stix@lcgc } { operators } #1 }
    \@@_set_slanted_greek:
%</stixtwo>
  }
%</math&!otf>
%<*math&otf>
%<cambria>  { \@@_fontset_error:nn { math } { cambria } }
%<newcm>  { \@@_fontset_error:nn { math } { newcm } }
%<xits>  { \@@_fontset_error:nn { math } { xits } }
%</math&otf>
%</!(math&type1)>
%</font&(math|text)>
%    \end{macrocode}
%
% \pkg{unicode-math} 宏包设置。
%    \begin{macrocode}
%<*class>
\ctex_at_end_package:nn { unicode-math }
  {
    \DeclareDocumentCommand \bm { m }
      { { \symbf {#1} } }
    \DeclareDocumentCommand \boldsymbol { m }
      { { \symbf {#1} } }
    \bool_if:NTF \g_@@_slanted_uppercase_greek_bool
      { \keys_set:nn { unicode-math } { math-style = ISO } }
      { \keys_set:nn { unicode-math } { math-style = TeX } }
    \bool_if:NTF \g_@@_integral_limits_bool
      { \removenolimits } { \addnolimits }
      {
        \int\iint\iiint\iiiint\oint\oiint\oiiint
        \intclockwise\varointclockwise\ointctrclockwise\sumint
        \intbar\intBar\fint\cirfnint\awint\rppolint
        \scpolint\npolint\pointint\sqint\intlarhk\intx
        \intcap\intcup\upint\lowint
      }
    \clist_map_inline:nn
      { amsfonts, amssymb, bm, upgreek }
      { \@@_disable_package_load:n {#1} }
  }
%    \end{macrocode}
%
% 若未使用 \pkg{unicode-math} 配置数学字体,则自动调用 \pkg{bm},
% 并在未定义 \tn{uppi} 和 \tn{checkmark} 时分别调用 \pkg{upgreek} 和 \pkg{amsfonts}。
%    \begin{macrocode}
\ctex_at_end_preamble:n
  {
    \IfPackageLoadedF { unicode-math }
      {
        \RequirePackage { bm }
        \cs_if_exist:NF \uppi
          { \RequirePackage { upgreek  } }
        \cs_if_exist:NF \checkmark
          { \RequirePackage { amsfonts } }
      }
  }
%</class>
%    \end{macrocode}
%
% \paragraph{CJK 字体}
%
% \changes{v2.2}{2024/11/28}{新增 \opt{hanyi} 字体配置。}
% 在字体未提供对应粗体的情况下,允许使用伪粗。
%    \begin{macrocode}
%<*font&cjk>
\@@_if_engine_opentype:TF
  {
    \@@_if_main_lang_ja:TF
      {
%<*windows>
        \setCJKmainfont { MS~Mincho } [ AutoFakeBold = 3 ]
        \setCJKsansfont { MS~Gothic } [ AutoFakeBold = 3 ]
        \setCJKmonofont { MS~Mincho }
        \setCJKfamilyfont { jamin  } { MS~Mincho } [ AutoFakeBold = 3 ]
        \setCJKfamilyfont { jagoth } { MS~Gothic } [ AutoFakeBold = 3 ]
%</windows>
%<*mac>
        \setCJKmainfont { HiraMinProN  }
          [
            UprightFont    = *-W3 ,
            BoldFont       = *-W6
          ]
        \setCJKsansfont { HiraKakuProN }
          [
            UprightFont    = *-W3 ,
            BoldFont       = *-W6
          ]
        \setCJKmonofont { HiraMinProN-W3 }
        \setCJKfamilyfont { jamin  } { HiraMinProN  }
          [
            UprightFont    = *-W3 ,
            BoldFont       = *-W6
          ]
        \setCJKfamilyfont { jagoth } { HiraKakuProN }
          [
            UprightFont    = *-W3 ,
            BoldFont       = *-W6
          ]
%</mac>
%<*ubuntu>
        \setCJKmainfont { Noto~Serif~CJK~JP }
          [
            UprightFont = *~Light ,
            BoldFont    = *~Bold
          ]
        \setCJKsansfont { Noto~Sans~CJK~JP  }
          [
            UprightFont = *~Medium ,
            BoldFont    = *~Bold
          ]
        \setCJKmonofont { Noto~Serif~CJK~JP }
          [
            UprightFont = *~Light ,
            BoldFont    = *~Bold
          ]
        \setCJKfamilyfont { jamin  } { Noto~Serif~CJK~JP }
          [
            UprightFont = *~Light ,
            BoldFont    = *~Bold
          ]
        \setCJKfamilyfont { jagoth } { Noto~Sans~CJK~JP  }
          [
            UprightFont = *~Medium ,
            BoldFont    = *~Bold
          ]
%</ubuntu>
%<*adobe>
        \setCJKmainfont { KozMinPr6N }
          [
            UprightFont = *-Light ,
            BoldFont    = *-Bold
          ]
        \setCJKsansfont { KozGoPr6N  }
          [
            UprightFont = *-Medium ,
            BoldFont    = *-Bold
          ]
        \setCJKmonofont { KozMinPr6N-Light }
        \setCJKfamilyfont { jamin  } { KozMinPr6N }
          [
            UprightFont = *-Light ,
            BoldFont    = *-Bold
          ]
        \setCJKfamilyfont { jagoth } { KozGoPr6N  }
          [
            UprightFont = *-Medium ,
            BoldFont    = *-Bold
          ]
%</adobe>
%<*fandol|hanyi>
        \setCJKmainfont { HaranoAjiMincho }
          [
            Extension   = .otf ,
            UprightFont = *-Regular ,
%<fandol>            BoldFont    = *-Bold
%<hanyi>            BoldFont    = *-Medium
          ]
        \setCJKsansfont { HaranoAjiGothic }
          [
            Extension   = .otf ,
            UprightFont = *-Medium ,
            BoldFont    = *-Bold
          ]
        \setCJKmonofont { HaranoAjiGothic }
          [
            Extension   = .otf ,
            UprightFont = *-Regular
          ]
        \setCJKfamilyfont { jamin  } { HaranoAjiMincho }
          [
            Extension   = .otf ,
%<fandol>            BoldFont    = *-Bold
%<hanyi>            BoldFont    = *-Medium
          ]
        \setCJKfamilyfont { jagoth } { HaranoAjiGothic }
          [
            Extension   = .otf ,
            UprightFont = *-Medium ,
            BoldFont    = *-Bold
          ]
%</fandol|hanyi>
%<*founder>
        \setCJKmainfont { ipam.ttf } [ AutoFakeBold = 3 ]
        \setCJKsansfont { ipag.ttf } [ AutoFakeBold = 3 ]
        \setCJKmonofont { ipag.ttf }
        \setCJKfamilyfont { jamin  } { ipam.ttf } [ AutoFakeBold = 3 ]
        \setCJKfamilyfont { jagoth } { ipag.ttf } [ AutoFakeBold = 3 ]
%</founder>
        \NewDocumentCommand \mincho   { } { \CJKfamily { jamin   } }
        \NewDocumentCommand \gothic   { } { \CJKfamily { jagoth  } }
      }
%<*windows>
      {
        \setCJKmainfont { SimSun   }
          [ AutoFakeBold = 3 , ItalicFont = KaiTi ]
        \setCJKsansfont { SimHei   } [ AutoFakeBold = 3 ]
        \setCJKmonofont { FangSong }
      }
    \setCJKfamilyfont { zhsong } { SimSun   }
      [ AutoFakeBold = 3 , ItalicFont = KaiTi ]
    \setCJKfamilyfont { zhhei  } { SimHei   } [ AutoFakeBold = 3 ]
    \setCJKfamilyfont { zhkai  } { KaiTi    }
    \setCJKfamilyfont { zhfs   } { FangSong }
%</windows>
%<*mac>
      {
        \setCJKmainfont { Songti~SC  }
          [
            UprightFont    = *~Light ,
            BoldFont       = *~Bold ,
            ItalicFont     = Kaiti~SC~Regular ,
            BoldItalicFont = Kaiti~SC~Bold
          ]
        \setCJKsansfont { Heiti~SC   }
          [
            UprightFont    = *~Medium ,
            AutoFakeBold   = 3
          ]
        \setCJKmonofont { STFangsong }
      }
    \setCJKfamilyfont { zhsong } { Songti~SC  }
      [
        UprightFont    = *~Light ,
        BoldFont       = *~Bold ,
        ItalicFont     = Kaiti~SC~Regular ,
        BoldItalicFont = Kaiti~SC~Bold
      ]
    \setCJKfamilyfont { zhhei  } { Heiti~SC   }
      [
        UprightFont    = *~Medium ,
        AutoFakeBold   = 3
      ]
    \setCJKfamilyfont { zhfs   } { STFangsong }
    \setCJKfamilyfont { zhkai  } { Kaiti~SC   }
      [
        UprightFont    = *~Regular ,
        BoldFont       = *~Bold
        ]
%</mac>
%<*ubuntu>
      {
        \setCJKmainfont { Noto~Serif~CJK~SC }
          [
            UprightFont = *~Light ,
            BoldFont    = *~Bold ,
            ItalicFont  = AR~PL~KaitiM~GB
          ]
        \setCJKsansfont { Noto~Sans~CJK~SC  }
          [
            UprightFont = *~Medium ,
            BoldFont    = *~Bold
          ]
        \setCJKmonofont { Noto~Serif~CJK~SC }
          [
            UprightFont = *~Light ,
            BoldFont    = *~Bold
          ]
      }
    \setCJKfamilyfont { zhsong } { Noto~Serif~CJK~SC }
      [
        UprightFont = *~Light ,
        BoldFont    = *~Bold ,
        ItalicFont  = AR~PL~KaitiM~GB
      ]
    \setCJKfamilyfont { zhhei  } { Noto~Sans~CJK~SC  }
      [
        UprightFont = *~Medium ,
        BoldFont    = *~Bold
      ]
    \setCJKfamilyfont { zhkai  } { AR~PL~KaitiM~GB   }
%</ubuntu>
%<*adobe>
      {
        \setCJKmainfont { AdobeSongStd-Light       }
          [ AutoFakeBold = 3 , ItalicFont = AdobeKaitiStd-Regular ]
        \setCJKsansfont { AdobeHeitiStd-Regular    } [ AutoFakeBold = 3 ]
        \setCJKmonofont { AdobeFangsongStd-Regular }
      }
    \setCJKfamilyfont { zhsong } { AdobeSongStd-Light       }
      [ AutoFakeBold = 3 , ItalicFont = AdobeKaitiStd-Regular ]
    \setCJKfamilyfont { zhhei  } { AdobeHeitiStd-Regular    } [ AutoFakeBold = 3 ]
    \setCJKfamilyfont { zhfs   } { AdobeFangsongStd-Regular }
    \setCJKfamilyfont { zhkai  } { AdobeKaitiStd-Regular    }
%</adobe>
%<*fandol>
      {
        \setCJKmainfont { FandolSong }
          [
            Extension   = .otf ,
            UprightFont = *-Regular ,
            BoldFont    = *-Bold ,
            ItalicFont  = FandolKai-Regular
          ]
        \setCJKsansfont { FandolHei  }
          [
            Extension   = .otf ,
            UprightFont = *-Regular ,
            BoldFont    = *-Bold
          ]
        \setCJKmonofont { FandolFang }
          [
            Extension   = .otf ,
            UprightFont = *-Regular
          ]
      }
    \setCJKfamilyfont { zhsong } { FandolSong }
      [
        Extension   = .otf ,
        UprightFont = *-Regular ,
        BoldFont    = *-Bold ,
        ItalicFont  = FandolKai-Regular
      ]
    \setCJKfamilyfont { zhhei  } { FandolHei  }
      [
        Extension   = .otf ,
        UprightFont = *-Regular,
        BoldFont    = *-Bold
      ]
    \setCJKfamilyfont { zhfs   } { FandolFang }
      [
        Extension   = .otf ,
        UprightFont = *-Regular
      ]
    \setCJKfamilyfont { zhkai  } { FandolKai  }
      [
        Extension   = .otf ,
        UprightFont = *-Regular
      ]
%</fandol>
%<*founder>
      {
        \setCJKmainfont { FZShuSong-Z01  }
          [ AutoFakeBold = 3 , ItalicFont = FZKai-Z03 ]
        \setCJKsansfont { FZHei-B01      } [ AutoFakeBold = 3 ]
        \setCJKmonofont { FZFangSong-Z02 }
      }
    \setCJKfamilyfont { zhsong } { FZShuSong-Z01  }
      [ AutoFakeBold = 3 , ItalicFont = FZKai-Z03 ]
    \setCJKfamilyfont { zhhei  } { FZHei-B01      } [ AutoFakeBold = 3 ]
    \setCJKfamilyfont { zhkai  } { FZKai-Z03      }
    \setCJKfamilyfont { zhfs   } { FZFangSong-Z02 }
%</founder>
%<*hanyi>
      {
        \setCJKmainfont { HYShuSongEr~S }
          [ BoldFont = HYZhongSong~S, ItalicFont = HYKaiTi~S ]
        \setCJKsansfont { HYZhongHei~S  } [ BoldFont = HYDaHei~S ]
        \setCJKmonofont { HYFangSong~S  }
      }
    \setCJKfamilyfont { zhsong } { HYShuSongEr~S }
      [ BoldFont = HYZhongSong~S ]
    \setCJKfamilyfont { zhhei  } { HYZhongHei~S  }
      [ BoldFont = HYDaHei~S     ]
    \setCJKfamilyfont { zhkai  } { HYKaiTi~S     }
    \setCJKfamilyfont { zhfs   } { HYFangSong~S  }
%</hanyi>
    \NewDocumentCommand \songti   { } { \CJKfamily { zhsong  } }
    \NewDocumentCommand \heiti    { } { \CJKfamily { zhhei   } }
%<!ubuntu>    \NewDocumentCommand \fangsong { } { \CJKfamily { zhfs    } }
    \NewDocumentCommand \kaishu   { } { \CJKfamily { zhkai   } }
  }
  {
    \@@_if_main_lang_ja:TF
%<windows>      { \@@_fontset_error:nn { cjk } { windows } }
%<windows>      { \ctex_file_input:n { ctex-fontset-windows.def } }
%<mac>      { \@@_fontset_error:nn { cjk } { mac } }
%<mac>      { \ctex_file_input:n { ctex-fontset-mac.def } }
%<ubuntu>      { \@@_fontset_error:nn { cjk } { ubuntu } }
%<ubuntu>      { \ctex_file_input:n { ctex-fontset-ubuntu.def } }
%<adobe>      { \@@_fontset_error:nn { cjk } { adobe } }
%<adobe>      { \ctex_file_input:n { ctex-fontset-adobe.def } }
%<fandol>      { \@@_fontset_error:nn { cjk } { fandol } }
%<fandol>      { \ctex_file_input:n { ctex-fontset-fandol.def } }
%<founder>      { \@@_fontset_error:nn { cjk } { founder } }
%<founder>      { \ctex_file_input:n { ctex-fontset-founder.def } }
%<hanyi>      { \@@_fontset_error:nn { cjk } { hanyi } }
%<hanyi>      { \ctex_file_input:n { ctex-fontset-hanyi.def } }
  }
%</font&cjk>
%    \end{macrocode}
%
% \begin{macro}{\CJKrmfamily,\CJKsffamily,\CJKttfamily}
% 只改变 CJK 字体族的命令。
%    \begin{macrocode}
%<*class>
\NewDocumentCommand \CJKrmfamily { } { \CJKfamily { \CJKrmdefault } }
\NewDocumentCommand \CJKsffamily { } { \CJKfamily { \CJKsfdefault } }
\NewDocumentCommand \CJKttfamily { } { \CJKfamily { \CJKttdefault } }
%    \end{macrocode}
% \end{macro}
%
% 带圈数字使用 CJK 字体。
%    \begin{macrocode}
\sys_if_engine_xetex:T
  {
    \xeCJK_declare_char_class:nn { CJK }
      { "24EA, "2460->"2473, "3251->"32BF, "25A1 }
  }
\sys_if_engine_luatex:T
  {
    \ltjdefcharrange { 99 }
      { "24EA, "2460- "2473, "3251- "32BF, "25A1 }
    \ltjsetparameter { jacharrange = { +99 } }
  }
%    \end{macrocode}
%
% \begin{macro}[int]{\@@_unicode_char:n}
% 由 Unicode 码位直接调用字符的辅助函数。
%    \begin{macrocode}
\sys_if_engine_pdftex:TF
  {
    \cs_new:Npn \@@_unicode_char:n #1
      {
        \exp_not:N \Unicode
          { \int_div_truncate:nn {#1} { 256 } }
          { \int_mod:nn          {#1} { 256 } }
      }
  }
  { \cs_new:Npn \@@_unicode_char:n #1 { \tex_Uchar:D #1 \scan_stop: } }
%    \end{macrocode}
% \end{macro}
%
% \begin{macro}[int]{\@@_symbol_const:nn}
% 定义符号的辅助函数。
%    \begin{macrocode}
\cs_new_protected:Npn \@@_symbol_const:nn #1#2
  { \tl_const:ce { c_@@_symbol_ #1 _tl } { \@@_unicode_char:n {#2} } }
%    \end{macrocode}
% \end{macro}
%
% \changes{v2.0.2}{2023/04/01}{延迟载入字体配置,修复 \pkg{unicode-math}
% 设置不生效的问题。}
% 载入字体配置。
%    \begin{macrocode}
\@@_load_fontset:
%    \end{macrocode}
%
% \subsubsection{名称选项}
%
% 定义 |sjtu/name| 键值类。
%
% 设置标准文档类中已定义的名称。
%    \begin{macrocode}
\keys_define:nn { sjtu / name }
  {
    contents        .meta:nn = { ctex } { contentsname   = {#1} } ,
    listfigure      .meta:nn = { ctex } { listfigurename = {#1} } ,
    listtable       .meta:nn = { ctex } { listtablename  = {#1} } ,
    figure          .meta:nn = { ctex } { figurename     = {#1} } ,
    table           .meta:nn = { ctex } { tablename      = {#1} } ,
%<!thesis>    abstract        .meta:nn = { ctex } { abstractname   = {#1} } ,
    index           .meta:nn = { ctex } { indexname      = {#1} } ,
    appendix        .meta:nn = { ctex } { appendixname   = {#1} } ,
    proof           .meta:nn = { ctex } { proofname      = {#1} } ,
    bib             .meta:nn = { ctex } { bibname        = {#1} } ,
    part           .tl_set:N = \partname ,
%<!article>    chapter        .tl_set:N = \chaptername ,
%    \end{macrocode}
%
% \changes{v2.0}{2022/12/28}{简化 \opt{sjtu/name} 中键的名称:\opt{abbr},
% \opt{nom}, \opt{ack}, \opt{achv}。}
% 标准文档类中未定义的名称。
%    \begin{macrocode}
    figure*        .tl_set:N = \SJTU@figurename@bi@second ,
    figure*       .initial:n = { 图 } ,
    table*         .tl_set:N = \SJTU@tablename@bi@second ,
    table*        .initial:n = { 表 } ,
    algorithm      .tl_set:N = \SJTU@algorithmname ,
    algorithm     .initial:n = { Algorithm } ,
    listalgorithm  .tl_set:N = \SJTU@listalgorithmname ,
    listalgorithm .initial:n = { List~of~Algorithms } ,
%<*thesis>
    abbr           .tl_set:N = \SJTU@abbrname ,
    abbr          .initial:n = { Abbreviations } ,
    nom            .tl_set:N = \SJTU@nomname ,
    nom           .initial:n = { Nomenclature } ,
    ack            .tl_set:N = \SJTU@ackname ,
    ack           .initial:n = { Acknowledgements } ,
    resume         .tl_set:N = \SJTU@resumename ,
    resume        .initial:n = { Resume } ,
    digest         .tl_set:N = \SJTU@digestname ,
    digest        .initial:n = { Digest } ,
    achv           .tl_set:N = \SJTU@achvname ,
    achv          .initial:n = { List~of~Research~Achievements },
%</thesis>
  }
%    \end{macrocode}
%
% \subsubsection{页面设置}
%
% 利用 \pkg{geometry} 宏包设置页面边距以及页眉高度。
%    \begin{macrocode}
\geometry
  {
    vmargin       = { 3.5 cm , 4.0 cm } ,
    hmargin       = 2.5 cm ,
%<thesis>    bindingoffset = 0.5 cm ,
    headheight    = 1.5 cm ,
    headsep       = 0.5 cm ,
    footskip      = 1.0 cm
  }
%    \end{macrocode}
%
% \changes{v2.0.3}{2023/09/24}{学位论文页面纵向顶部对齐。}
% 学位论文页面纵向顶部对齐。
%    \begin{macrocode}
%<*thesis>
\AtEndOfClass { \raggedbottom }
%</thesis>
%    \end{macrocode}
%
% \subsubsection{页眉页脚}
%
% \pkg{ctex} 宏包使用 \opt{heading} 选项后,会把页面格式设置为 |headings|。
% 因此必须在 \pkg{ctex} 调用之后重新设置 \tn{pagestyle} 为 |fancy|。
%    \begin{macrocode}
\pagestyle { fancy }
%    \end{macrocode}
%
% 清除所有页眉页脚。
%    \begin{macrocode}
\fancyhf { }
%    \end{macrocode}
%
% \begin{macro}{style/header-font,style/footer-font}
% 设置页眉页脚字体。
%    \begin{macrocode}
\keys_define:nn { sjtu / style }
  {
    header-font  .tl_set:N = \l_@@_style_header_font_tl ,
%<thesis>    header-font .initial:n = \zihao { -5 } \setbaselineskip { 12 bp } ,
%<!thesis>    header-font .initial:n = \zihao { -5 } \sffamily ,
    footer-font  .tl_set:N = \l_@@_style_footer_font_tl ,
%<thesis>    footer-font .initial:n = \zihao { -5 } \setbaselineskip { 12 bp }
%<!thesis>    footer-font .initial:n = \zihao { -5 }
  }
\fancyheadinit { \l_@@_style_header_font_tl }
\fancyfootinit { \l_@@_style_footer_font_tl }
%    \end{macrocode}
% \end{macro}
%
% \begin{macro}{style/header-uppercase}
% \begin{macro}[int]{\@@_nouppercase:n}
% 页眉西文是否大写。
%    \begin{macrocode}
\keys_define:nn { sjtu / style }
  {
    header-uppercase       .choice: ,
    header-uppercase / true  .code:n =
      { \cs_set_eq:NN \@@_nouppercase:n \use:n                 } ,
    header-uppercase / false .code:n =
      { \cs_set:Nn    \@@_nouppercase:n { \nouppercase {##1} } } ,
    header-uppercase      .default:n = { true  } ,
    header-uppercase      .initial:n = { false }
  }
\cs_generate_variant:Nn \@@_nouppercase:n { V }
%    \end{macrocode}
% \end{macro}
% \end{macro}
%
% \begin{macro}{style/page-number}
% \begin{macro}[int]{\@@_page:nn}
% 页脚页码格式。
%    \begin{macrocode}
\tl_set:Nn \@@_the_page_tl { \thepage }
\tl_new:N \@@_the_last_page_tl
\msg_new:nnn { sjtutex } { require-pageslts }
  {
    Add~'\token_to_str:N \usepackage{pageslts}'~in~your~preamble \\
    to~enable~the~page~number~of~the~last~page.
  }
\keys_define:nn { sjtu / style }
  {
    page-number    .code:n =
      {
        \regex_match:nnT { \#2 } { #1 }
          {
            \ctex_if_preamble:TF
              { \RequirePackage { pageslts } }
              {
                \IfPackageLoadedF { pageslts }
                  { \msg_warning:nn { sjtutex } { require-pageslts } }
              }
          }
        \cs_set:Nn \@@_page:nn { #1 }
      } ,
    page-number .initial:n = { {#1} }
  }
%    \end{macrocode}
% \end{macro}
% \end{macro}
%
% 设置页眉内容。
%    \begin{macrocode}
\tl_set:Nn \l_@@_header_tl
  {
%<thesis>    { \songti \l_@@_info_subject_zh_tl }
%<*!thesis>
    \includegraphics [ height = 1.2 cm ]
      { sjtu-vi-logo-std-h-cor-red.pdf }
%</!thesis>
  }
%<thesis>\tl_set:Nn \l_@@_leftmark_tl  { \leftmark }
%<!thesis>\tl_set:Nn \l_@@_leftmark_tl  { \l_@@_info_subject_tl }
\tl_set:Nn \l_@@_rightmark_tl { \leftmark }
%    \end{macrocode}
%
% 设置页眉页脚。
%    \begin{macrocode}
\legacy_if:nTF { @twoside }
  {
    \fancyhead [ LO, RE ] { \l_@@_header_tl }
    \fancyhead [ LE ]     { \@@_nouppercase:V \l_@@_leftmark_tl  }
    \fancyhead [ RO ]     { \@@_nouppercase:V \l_@@_rightmark_tl }
  }
  {
    \fancyhead [ L ] { \l_@@_header_tl }
    \fancyhead [ R ] { \@@_nouppercase:V \l_@@_rightmark_tl }
  }
\fancyfoot [ C ]
  { \@@_page:nn { \@@_the_page_tl } { \@@_the_last_page_tl } }
%    \end{macrocode}
%
% \begin{macro}{\headrule}
% \cls{sjtuthesis} 的页眉线。
%    \begin{macrocode}
%<*thesis>
\tl_set:Nn \headrule
  {
    \hrule height 2.25 pt width \headwidth
    \skip_vertical:n {  0.75 pt }
    \hrule height 0.75 pt width \headwidth
    \skip_vertical:n { -3.75 pt }
  }
%</thesis>
%    \end{macrocode}
% \end{macro}
%
% \changes{v2.2}{2024/11/08}{更新 \cls{sjtureport} 和 \cls{sjtuarticle} 的
% \texttt{plain} 页面样式。}
% 重定义 |plain| 样式。
%    \begin{macrocode}
%<*!thesis>
\legacy_if:nTF { @twoside }
  {
    \fancypagestyle { plain }
      {
        \fancyhead { }
        \fancyhead [ LO, RE ] { \l_@@_header_tl }
        \tl_set:Nn \headrulewidth { 0 pt }
      }
  }
  {
    \fancypagestyle { plain }
      {
        \fancyhead { }
        \fancyhead [ L ] { \l_@@_header_tl }
        \tl_set:Nn \headrulewidth { 0 pt }
      }
  }
%</!thesis>
%    \end{macrocode}
%
% |SJTU@null| 样式,不对当前页面样式做任何修改。
%    \begin{macrocode}
\cs_new_eq:NN \ps@SJTU@null \prg_do_nothing:
%    \end{macrocode}
%
% \begin{macro}{\cleardoublepage}
% 空白页清空页眉页脚。
%    \begin{macrocode}
\RenewDocumentCommand \cleardoublepage { }
  {
    \clearpage
    \legacy_if:nT { @twoside }
      {
        \int_if_odd:nF \c@page
          { \hbox:n { } \thispagestyle { empty } \newpage }
      }
  }
%    \end{macrocode}
% \end{macro}
%
% \subsubsection{页码设置}
%
% 文档初始页码编码设置。
%    \begin{macrocode}
%<thesis>\pagenumbering { Alph }
%    \end{macrocode}
%
% \begin{macro}{\frontmatter}
% 前置部分使用大写罗马数字编码。
%    \begin{macrocode}
%<*thesis>
\RenewDocumentCommand \frontmatter { }
  {
    \cleardoublepage
    \@mainmatterfalse
    \pagenumbering { Roman }
  }
%</thesis>
%    \end{macrocode}
% \end{macro}
%
% \subsubsection{章节标题}
%
% 设置章节标题样式。
%    \begin{macrocode}
%<*!article>
\ctex_set:nn { chapter }
  {
    pagestyle   = SJTU@null ,
    fixskip     = true ,
%<*thesis>
    beforeskip  = 27 bp ,
    afterskip   = 27 bp ,
    format      = \zihao { 3 } \setbaselineskip{ 20 bp } \bfseries
                  \CJKsffamily \centering ,
%</thesis>
%<*report>
    beforeskip  = 30 pt ,
    afterskip   = 24 pt ,
    format      = \Large \bfseries \CJKsffamily \centering ,
%</report>
    nameformat  = ,
    titleformat = ,
    lofskip     = \c_zero_skip ,
    lotskip     = \c_zero_skip ,
    aftername   = \quad
  }
%</!article>
%</class>
%<*scheme&(zh|ja)>
\keys_set_known:nn { ctex / chapter }
  { name = { 第 \space , \space 章 } }
%</scheme&(zh|ja)>
%<*class>
\ctex_set:nn { section }
%<*thesis>
  {
    beforeskip  = 24 bp ,
    afterskip   =  6 bp ,
    format      = \zihao { 4 } \setbaselineskip { 18 bp } \bfseries
                  \CJKsffamily
  }
%</thesis>
%<!thesis>  { format = \large \bfseries \CJKsffamily }
\ctex_set:nn { subsection }
%<*thesis>
  {
    beforeskip  = 12 bp ,
    afterskip   =  6 bp ,
    format      = \zihao { -4 } \setbaselineskip { 16 bp } \bfseries
                  \CJKsffamily
  }
%</thesis>
%<!thesis>  { format = \SJTU@orig@normalsize \bfseries \CJKsffamily }
\ctex_set:nn { subsubsection }
%<*thesis>
  {
    beforeskip  =  6 bp ,
    afterskip   =  6 bp ,
    format      = \zihao { -4 } \setbaselineskip { 16 bp } \normalfont
  }
%</thesis>
%<!thesis>  { format = \SJTU@orig@normalsize \normalfont }
\ctex_set:n { secnumdepth = 3 }
%    \end{macrocode}
%
% \changes{v2.1}{2024/01/10}{新增 \opt{style/indent-first} 选项。}
% \begin{macro}{style/indent-first}
% 章节标题后首段是否缩进。
%    \begin{macrocode}
\keys_define:nn { sjtu / style }
  {
    indent-first    .choice: ,
    indent-first   .choices:nn =
      { true, false }
      {
        \clist_map_inline:nn
          {
%<article>            part,
%<!article>            chapter,
            section, subsection, subsubsection,
            paragraph, subparagraph
          }
          { \ctex_set:nn {####1} { afterindent = #1 } }
      } ,
    indent-first   .default:n = { true } ,
    indent-first   .initial:n = { true }
  }
%    \end{macrocode}
% \end{macro}
%
% \begin{macro}[int]{\@@_pdf_bookmark:nn,\@@_phantom_section:}
% 在 \pkg{hyperref} 载入后才有意义。
%    \begin{macrocode}
\cs_new_eq:NN \@@_pdf_bookmark:nn \use_none:nn
\cs_new_eq:NN \@@_phantom_section: \prg_do_nothing:
%</class>
%    \end{macrocode}
% \end{macro}
%
% \subsubsection{段落}
%
% 设置全文首行缩进。
%    \begin{macrocode}
%<*scheme>
\ctex_if_autoindent_touched:F
%<zh>  { \ctex_set:n { autoindent = true } }
%<en|de>  { \ctex_set:n { autoindent = 1.5 em } }
%<ja>  { \ctex_set:n { autoindent = 1 } }
%    \end{macrocode}
%
% \begin{macro}{\verse,\quotation}
% 修改诗歌和引用环境的缩进。
%    \begin{macrocode}
%<zh|ja>\ctex_patch_cmd:Nnn \verse { -1.5em } { -2 \ccwd }
%<zh|ja>\ctex_patch_cmd:Nnn \verse {  1.5em } {  2 \ccwd }
\ctex_patch_cmd:Nnn \quotation { 1.5em } { \parindent }
%</scheme>
%    \end{macrocode}
% \end{macro}
%
% 使用 \pkg{enumitem} 调整默认列表环境的间距。
%    \begin{macrocode}
%<*class>
\setlist { nosep }
%    \end{macrocode}
%
% \subsubsection{数学公式}
%
% \changes{v2.1}{2023/11/30}{新增 \opt{style/equation-font} 选项。}
% \begin{macro}{style/equation-font}
% 设置行间数学公式的字体。
%    \begin{macrocode}
\keys_define:nn { sjtu / style }
  {
    equation-font  .tl_set:N = \SJTU@style@equation@font ,
%<thesis>    equation-font .initial:n = \linespread { } \SJTU@orig@normalsize
%<!thesis>    equation-font .initial:V = \c_empty_tl
  }
%    \end{macrocode}
% \end{macro}
%
% 借用 \pkg{zhlineskip} 宏包的代码设置行间公式字体。
%    \begin{macrocode}
\clist_map_inline:nn
  {
    array, matrix, pmatrix, bmatrix, Bmatrix, vmatrix, Vmatrix,
    matrix*, pmatrix*, bmatrix*, Bmatrix*, vmatrix*, Vmatrix*,
    cases, cases*, dcases, dcases*, rcases, rcases*, drcases, drcases*,
    aligned, alignedat, gathered, multlined, lgathered, rgathered
  }
  { \AtBeginEnvironment {#1} { \SJTU@style@equation@font } }
\clist_map_inline:nn
  { \start@gather, \start@align, \start@multline }
  {
    \ctex_patch_cmd:Nnn #1
      { \collect@body }
      {
        \SJTU@style@equation@font
        \collect@body
      }
  }
\ctex_patch_cmd:Nnn \gather@split
  { \spread@equation }
  {
    \SJTU@style@equation@font
    \spread@equation
  }
%    \end{macrocode}
%
% \subsubsection{数学环境}
%
% 可以选用 \pkg{amsthm} 或 \pkg{ntheorem} 宏包控制数学环境样式,
% 并提供对证明环境 \env{proof} 的支持。
%
% \changes{v2.1.1}{2024/03/21}{新增 \opt{style/theorem-header-font}、
% \opt{style/theorem-body-font} 选项。}
% \begin{macro}{style/theorem-header-font,style/theorem-body-font}
% 预定义的数学环境的定理头(即标题)以及定理内容的字体。
%    \begin{macrocode}
\keys_define:nn { sjtu / style }
  {
    theorem-header-font  .tl_set:N = \SJTU@style@thm@header@font ,
    theorem-header-font .initial:n = \bfseries \CJKsffamily ,
    theorem-body-font    .tl_set:N = \SJTU@style@thm@body@font ,
    theorem-body-font   .initial:n = \normalfont ,
  }
%    \end{macrocode}
% \end{macro}
%
% \subsubsection{浮动体}
%
% 下面这组命令使浮动对象的缺省值稍微宽松一点,从而防止幅度对象占据过多的
% 文本页面,也可以防止在很大空白的浮动页上放置很小的图形。
%    \begin{macrocode}
\tl_set:Nn \textfraction      { 0.15 }
\tl_set:Nn \topfraction       { 0.85 }
\tl_set:Nn \bottomfraction    { 0.65 }
\tl_set:Nn \floatpagefraction { 0.60 }
%    \end{macrocode}
%
% \begin{macro}{style/float-font}
% 设置浮动体内的字体。
%    \begin{macrocode}
\keys_define:nn { sjtu / style }
  {
    float-font  .tl_set:N = \SJTU@style@float@font ,
%<thesis>    float-font .initial:n = \zihao { 5 } \setbaselineskip { 14 bp }
%<!thesis>    float-font .initial:n = \zihao { 5 }
  }
\ctex_patch_cmd:Nnn \@floatboxreset
  { \normalsize } { \SJTU@style@float@font }
%    \end{macrocode}
% \end{macro}
%
% \changes{v2.1}{2023/11/29}{更新题注格式。}
% \begin{macro}{style/caption-font,style/subcaption-font}
% 题注格式。
%    \begin{macrocode}
\keys_define:nn { sjtu / style }
  {
%    \end{macrocode}
%
% \pkg{bicaption} 中双语标题之间的间距受 \pkg{caption} 字体定义 \opt{normalsize}
% 的影响,这里我们直接重定义 \opt{normalsize}。
%    \begin{macrocode}
    caption-font       .code:n =
      { \DeclareCaptionFont { normalsize    } {#1} } ,
%    \end{macrocode}
%
%    \begin{macrocode}
%<thesis>    caption-font    .initial:n = \zihao { 5 } \setbaselineskip { 14 bp }
%<thesis>                                 \bfseries ,
%<!thesis>    caption-font    .initial:n = \zihao { 5 } \bfseries ,
    subcaption-font    .code:n =
      { \DeclareCaptionFont { SJTU@sub@font } {#1} } ,
%<thesis>    subcaption-font .initial:n = \zihao { 5 } \setbaselineskip { 14 bp }
%<thesis>                                 \normalfont
%<!thesis>    subcaption-font .initial:n = \zihao { 5 } \normalfont
  }
\captionsetup
  {
    bi-slc   = off ,
    labelsep = quad ,
%<thesis>    skip     = 6 bp
  }
\captionsetup [ sub ]
  {
    bi-slc   = on ,
    font     = SJTU@sub@font ,
    format   = hang
  }
%    \end{macrocode}
% \end{macro}
%
% 双语题注。
%    \begin{macrocode}
\DeclareCaptionOption { bi-second-names } [ ]
  {
    \tl_set:Nn \figurename { \SJTU@figurename@bi@second }
    \tl_set:Nn \tablename  { \SJTU@tablename@bi@second  }
  }
\captionsetup [ bi-second ] { bi-second-names }
%    \end{macrocode}
%
% \changes{v2.1}{2023/12/02}{新增 \opt{style/num-sep}、
% \opt{style/theorem-num-sep} 选项。}
% \begin{macro}{style/num-sep,style/float-num-sep,
% style/equation-num-sep,style/theorem-num-sep}
% 图、表、公式以及定理编号中的分隔符。
%    \begin{macrocode}
\keys_define:nn { sjtu / style }
  {
    num-sep            .code:n =
      {
        \tl_set:Nn \SJTU@style@fl@num@sep {#1}
        \tl_set:Nn \SJTU@style@eq@num@sep {#1}
        \tl_set:Nn \@thmcountersep        {#1}
      } ,
    num-sep         .initial:n = { . } ,
    float-num-sep    .tl_set:N = \SJTU@style@fl@num@sep ,
    equation-num-sep .tl_set:N = \SJTU@style@eq@num@sep ,
    theorem-num-sep  .tl_set:N = \@thmcountersep
  }
%    \end{macrocode}
% \end{macro}
%
% \begin{macro}[int]{\SJTU@counterwithin}
%    \begin{macrocode}
\NewDocumentCommand \SJTU@counterwithin
  { s O{ \SJTU@style@fl@num@sep } O{ \arabic } m m }
  {
    \@ifbothcounters {#4} {#5}
      {
        \@addtoreset {#4} {#5}
        \IfBooleanF {#1}
          {
            \exp_args:Nco \cs_gset:Npn { the #4 }
              { \cs:w the #5 \cs_end: #2 #3 {#4} }
          }
      }
  }
%    \end{macrocode}
% \end{macro}
%
% 定义图、表、公式的编号格式。
%    \begin{macrocode}
%<*!article>
\SJTU@counterwithin { figure } { chapter }
\SJTU@counterwithin { table  } { chapter }
\SJTU@counterwithin [ \SJTU@style@eq@num@sep ] { equation } { chapter }
%</!article>
%    \end{macrocode}
%
% \begin{variable}{\l_@@_counter_without_chapter_clist}
% 大摘要中不需要随章编号的各计数器。
%    \begin{macrocode}
%<*thesis>
\clist_set:Nn \l_@@_counter_without_chapter_clist
  { section, figure, table, equation }
%</thesis>
%    \end{macrocode}
% \end{variable}
%
% \subsubsection{脚注}
%
% \begin{macro}[int]{\@@_circled_number:N}
% 通过 Unicode 码位调用带圈数字。
%    \begin{macrocode}
\msg_new:nnn { sjtutex } { circled-number-exceed }
  { Circled~number~exceed~50! }
\cs_new:Npn \@@_circled_number:N #1
  {
    \int_compare:nNnTF {#1} < { 21 }
      { \@@_unicode_char:n { \int_eval:n { "2460 - 1 + #1 } } }
      {
        \int_compare:nNnTF {#1} < { 36 }
          { \@@_unicode_char:n { \int_eval:n { "3251 - 21 + #1 } } }
          {
            \int_compare:nNnTF {#1} < { 51 }
              { \@@_unicode_char:n { \int_eval:n { "32B1 - 36 + #1 } } }
              {
                \msg_warning:nn { sjtutex } { circled-number-exceed }
                \int_to_arabic:n {#1}
              }
          }
      }
  }
%    \end{macrocode}
% \end{macro}
%
% \begin{macro}{style/fnmark-font}
% 脚注编号字体。
%    \begin{macrocode}
\keys_define:nn { sjtu / style }
  {
    fnmark-font           .choice: ,
    fnmark-font / haranoaji .code:n =
      {
        \@@_if_engine_opentype:TF
          {
            \tl_set:Nn \l_@@_style_fnmark_font_tl
              {
                \CJKfontspec { HaranoAjiMincho }
                  [
                    Extension   = .otf ,
                    UprightFont = *-Regular ,
                    BoldFont    = *-Bold
                  ]
              }
          }
          { \tl_set_eq:NN \l_@@_style_fnmark_font_tl \c_empty_tl }
      } ,
    fnmark-font / unknown .tl_set:N = \l_@@_style_fnmark_font_tl ,
    fnmark-font          .initial:V = \c_empty_tl
  }
%    \end{macrocode}
% \end{macro}
%
% \begin{macro}{style/fnmark-style}
% 脚注编号样式。
% 重定义内部脚注文字命令,使用带圈数字编号时,脚注不使用上标。
% 见 \url{https://www.zhihu.com/question/53030087}。
%    \begin{macrocode}
\keys_define:nn { sjtu / style }
  {
    fnmark-style           .choice: ,
    fnmark-style / plain   .code:n =
      {
        \cs_set:Npn \SJTU@makefnmark
          { \hbox:n { \@textsuperscript { \normalfont \@thefnmark } } }
        \tl_set:Nn \thefootnote
          { \arabic { footnote } }
        \tl_set:Nn \thempfootnote
          { { \itshape \alph { mpfootnote } } }
      } ,
    fnmark-style / circled .code:n =
      {
        \cs_set:Npn \SJTU@makefnmark
          { \hbox:n { \@thefnmark } }
        \tl_set:Nn \thefootnote
          { { \l_@@_style_fnmark_font_tl \@@_circled_number:N \c@footnote } }
        \tl_set:Nn \thempfootnote
          { { \l_@@_style_fnmark_font_tl \@@_circled_number:N \c@mpfootnote } }
      }
  }
%</class>
%    \end{macrocode}
% \end{macro}
%
%    \begin{macrocode}
%<*scheme>
\keys_set:nn { sjtu / style }
%<zh|ja>  { fnmark-style = circled }
%<en|de>  { fnmark-style = plain   }
%</scheme>
%    \end{macrocode}
%
% 在导言末尾修改 \tn{@makefntext},支持使用 \pkg{footmisc} 修改脚注格式。
%    \begin{macrocode}
%<*class>
\ctex_at_end_preamble:n
  {
    \cs_set_eq:NN \SJTU@orig@makefntext \@makefntext
    \cs_set:Npn \@makefntext #1
      {
        \group_begin:
          \cs_set_eq:NN \@makefnmark \SJTU@makefnmark
          \SJTU@orig@makefntext {#1}
        \group_end:
      }
  }
%    \end{macrocode}
%
% \subsection{多语言支持}
%
% \begin{variable}{\l_@@_lang_tl}
%    \begin{macrocode}
\tl_set_eq:NN \l_@@_lang_tl \g_@@_lang_tl
%    \end{macrocode}
% \end{variable}
%
%    \begin{macrocode}
\msg_new:nnn { sjtutex } { lang-validation }
  { Invalid~language~argument~'#1'! }
\keys_define:nn { sjtu / private }
  {
    lang .choice: ,
    lang .value_required:n = true ,
    lang .groups:n = { lang } ,
    lang / unknown .code:n =
      { \msg_warning:nnn { sjtutex } { lang-validation } {#1} }
  }
%    \end{macrocode}
%
% \begin{macro}[int]{\@@_select_language:n}
% 选择局部语言。
%    \begin{macrocode}
\cs_new_protected:Npn \@@_select_language:n
  { \keys_set_groups:nnn { sjtu / private } { lang } }
%    \end{macrocode}
% \end{macro}
%
% 载入语言配置文件。
%    \begin{macrocode}
%<thesis>\clist_map_inline:Nn \g_@@_lang_clist
%<thesis>  { \file_input:n { sjtu-lang- #1 .def } }
%<!thesis>\file_input:n { sjtu-lang- \g_@@_lang_tl .def }
\file_input:n { sjtu-scheme- \g_@@_lang_tl .def }
%</class>
%    \end{macrocode}
%
% \begin{macro}[int]{\@@_set_cjk_default_zh:,\@@_set_cjk_default_ja:}
% 设置 CJK 默认字体族的辅助命令。
%    \begin{macrocode}
%<*lang>
%<*zh>
\cs_new_protected:Nn \@@_set_cjk_default_zh:
  {
    \tl_set:Nn \CJKrmdefault { zhsong }
    \tl_set:Nn \CJKsfdefault { zhhei  }
    \tl_set:Nn \CJKttdefault { zhfs   }
  }
%</zh>
%<*ja>
\cs_new_protected:Nn \@@_set_cjk_default_ja:
  {
    \tl_set:Nn \CJKrmdefault { jamin  }
    \tl_set:Nn \CJKsfdefault { jagoth }
    \tl_set:Nn \CJKttdefault { jagoth }
  }
%</ja>
%    \end{macrocode}
% \end{macro}
%
% \changes{v2.2.1}{2025/03/27}{新增语言设置钩子。}
% 语言设置钩子。
%    \begin{macrocode}
%<zh>\NewHook { sjtutex / lang / zh }
%<en>\NewHook { sjtutex / lang / en }
%<de>\NewHook { sjtutex / lang / de }
%<ja>\NewHook { sjtutex / lang / ja }
%    \end{macrocode}
%
%    \begin{macrocode}
\keys_define:nn { sjtu / private }
  {
%<zh>    lang / zh .code:n =
%<en>    lang / en .code:n =
%<de>    lang / de .code:n =
%<ja>    lang / ja .code:n =
      {
        \tl_set_eq:NN \l_@@_lang_tl \l_keys_value_tl
%<zh>        \@@_set_cjk_default_zh:
%<ja>        \@@_set_cjk_default_ja:
        \normalfont
%<zh>        \UseHook { sjtutex / lang / zh }
%<en>        \UseHook { sjtutex / lang / en }
%<de>        \UseHook { sjtutex / lang / de }
%<ja>        \UseHook { sjtutex / lang / ja }
      } ,
%<zh>    zh .meta:n = { lang = zh } ,
%<zh>    zh .groups:n = { lang }
%<en>    en .meta:n = { lang = en } ,
%<en>    en .groups:n = { lang }
%<de>    de .meta:n = { lang = de } ,
%<de>    de .groups:n = { lang }
%<ja>    ja .meta:n = { lang = ja } ,
%<ja>    ja .groups:n = { lang }
  }
%    \end{macrocode}
%
%    \begin{macrocode}
%<zh>\AddToHook { sjtutex / lang / zh }
%<en>\AddToHook { sjtutex / lang / en }
%<de>\AddToHook { sjtutex / lang / de }
%<ja>\AddToHook { sjtutex / lang / ja }
  {
%<zh>    \tl_set:Nn \languagename { chinese }
%<en>    \tl_set:Nn \languagename { english }
%<de>    \tl_set:Nn \languagename { ngerman }
%<ja>    \tl_set:Nn \languagename { japanese }
%<zh>    \ctex_set:n { autoindent = true }
%<en|de>    \ctex_set:n { autoindent = 1.5 em }
%<ja>    \ctex_set:n { autoindent = 1 }
  }
%    \end{macrocode}
%
% \begin{macro}[int]{\@@_title_case_aux_zh:n,\@@_title_case_aux_en:n,
% \@@_title_case_aux_de:n,\@@_title_case_aux_ja:n}
% 设置标题大小写转换的辅助命令。
%    \begin{macrocode}
%<zh>\cs_set_eq:NN \@@_title_case_aux_zh:n \use:n
%<en>\cs_set_eq:NN \@@_title_case_aux_en:n \MakeUppercase
%<de>\cs_set_eq:NN \@@_title_case_aux_de:n \MakeUppercase
%<ja>\cs_set_eq:NN \@@_title_case_aux_ja:n \use:n
%    \end{macrocode}
% \end{macro}
%
% 通用名称常量。
%    \begin{macrocode}
\clist_map_inline:nn
  {
%<zh>    { keywords } { 关键词          } ,
%<en>    { keywords } { Key~words       } ,
%<de>    { keywords } { Schlüsselwörter } ,
%<ja>    { keywords } { キーワード      } ,
%<zh|ja>    { info_sep } { : \null        } ,
%<en|de>    { info_sep } { \hbox { :~ }    } ,
%<zh>    { item_sep } { ,              }
%<en|de>    { item_sep } { ,~              }
%<ja>    { item_sep } { \quad           }
  }
%<zh>  { \@@_name_const:nnn { zh } #1 }
%<en>  { \@@_name_const:nnn { en } #1 }
%<de>  { \@@_name_const:nnn { de } #1 }
%<ja>  { \@@_name_const:nnn { ja } #1 }
%</lang>
%    \end{macrocode}
%
% 初始化语言名称。
%    \begin{macrocode}
%<*scheme>
%<zh>\tl_set:Nn \languagename { chinese  }
%<en>\tl_set:Nn \languagename { english  }
%<de>\tl_set:Nn \languagename { ngerman  }
%<ja>\tl_set:Nn \languagename { japanese }
%    \end{macrocode}
%
% 设置名称选项。默认值为英文,只需修改其他语种。
%    \begin{macrocode}
%<*!en>
\keys_set_known:nn { sjtu / name }
  {
%<*zh>
    contents      = { 目 \protect \quad 录   } ,
    listfigure    = { 插 \protect \quad 图   } ,
    listtable     = { 表 \protect \quad 格   } ,
    figure        = { 图                     } ,
    table         = { 表                     } ,
    abstract      = { 摘 \protect \quad 要   } ,
    index         = { 索 \protect \quad 引   } ,
    appendix      = { 附录                   } ,
    proof         = { 证明                   } ,
    bib           = { 参考文献               } ,
    figure*       = { Figure                 } ,
    table*        = { Table                  } ,
    algorithm     = { 算法                   } ,
    listalgorithm = { 算 \protect \quad 法   } ,
    abbr          = { 缩略语对照表           } ,
    nom           = { 符号对照表             } ,
    ack           = { 致 \protect \quad 谢   } ,
    resume        = { 个人简历               } ,
    digest        = { 大摘要                 } ,
    achv          = { 学术论文和科研成果目录 }
%</zh>
%<*de>
    contents      = { Inhaltsverzeichnis     } ,
    listfigure    = { Abbildungsverzeichnis  } ,
    listtable     = { Tabellenverzeichnis    } ,
    figure        = { Abbildung              } ,
    table         = { Tabelle                } ,
    abstract      = { Zusammenfassung        } ,
    index         = { Index                  } ,
    appendix      = { Anhang                 } ,
    proof         = { Beweis                 } ,
    bib           = { Literaturverzeichnis   } ,
    part          = { Teil                   } ,
    chapter       = { Kapitel                } ,
    figure*       = { Figure                 } ,
    table*        = { Table                  } ,
    algorithm     = { Algorithmus            } ,
    listalgorithm = { Algorithmenverzeichnis } ,
    abbr          = { Abkürzungsverzeichnis  } ,
    nom           = { Symbolverzeichnis      } ,
    ack           = { Danksagungen           } ,
    resume        = { Lebenslauf             } ,
    digest        = { Kurzfassung            } ,
    achv          = { Forschungsleistungen   }
%</de>
%<*ja>
    contents      = { 目 \protect \quad 次 } ,
    listfigure    = { 図目次               } ,
    listtable     = { 表目次               } ,
    figure        = { 図                   } ,
    table         = { 表                   } ,
    abstract      = { 概 \protect \quad 要 } ,
    index         = { 索 \protect \quad 引 } ,
    appendix      = { 付録                 } ,
    proof         = { 证明                 } ,
    bib           = { 参考文献             } ,
    figure*       = { Figure               } ,
    table*        = { Table                } ,
    algorithm     = { アルゴリズム         } ,
    listalgorithm = { アルゴリズム目次     } ,
    abbr          = { 略語表               } ,
    nom           = { 記号表               } ,
    ack           = { 謝 \protect \quad 辞 } ,
    resume        = { 履歴書               } ,
    digest        = { 要 \protect \quad 約 } ,
    achv          = { 研究業績書           }
%</ja>
  }
%</!en>
%</scheme>
%    \end{macrocode}
%
% 学位论文名称常量。
%    \begin{macrocode}
%<*thesis-i18n>
%<zh>\@@_symbol_const:nn { white_square } { "25A1 }
%<zh>\@@_name_const_from_clist:nnnn { zh }
%<en>\@@_name_const_from_clist:nnnn { en }
%<de>\@@_name_const_from_clist:nnnn { de }
%<ja>\@@_name_const_from_clist:nnnn { ja }
  { degree_level } { \g_@@_thesis_type_int }
%<zh>  { 学士, 硕士, 博士 }
%<en>  { Bachelor, Master, Doctor }
%<de>  { Bachelor, Master, Doktor }
%<ja>  { 学士, 修士, 博士 }
\clist_map_inline:nn
  {
%<*zh>
    { univ             } { 上海交通大学           } ,
    { address          } { 中国・上海             } ,
    { thesis           } { 学位论文               } ,
    { title_page       } { 题名页                 } ,
    { declaration      } { 原创性声明及使用授权书 } ,
    { orig_decl        } { 原创性声明             } ,
    { auth_decl        } { 使用授权书             } ,
    { decl_author      } { 学位论文作者           } ,
    { decl_supervisor  } { 指导教师               } ,
    { abstract         } { 摘 \protect \quad 要   }
%</zh>
%<*en>
    { univ             } { Shanghai~ Jiao~ Tong~ University } ,
    { address          } { Shanghai,~ P.R.~ China           } ,
    { title_page       } { Title~ Page                      } ,
    { declaration      } { Statutory~ Declaration           } ,
    { abstract         } { Abstract                         }
%</en>
%<*de>
    { univ             } { Shanghai~ Jiao~ Tong~ Universität } ,
    { address          } { Shanghai,~ VR~ China              } ,
    { title_page       } { Titelblatt                        } ,
    { declaration      } { Eidesstattliche~ Erklärung        } ,
    { abstract         } { Abstrakt                          }
%</de>
%<*ja>
    { univ             } { 上海交通大学         } ,
    { address          } { 中国・上海           } ,
    { thesis           } { 学位請求論文         } ,
    { title_page       } { 標題紙               } ,
    { declaration      } { 誓約書・公表許諾書   } ,
    { abstract         } { 要 \protect \quad 旨 }
%</ja>
  }
%<zh>  { \@@_name_const:nnn { zh } #1 }
%<en>  { \@@_name_const:nnn { en } #1 }
%<de>  { \@@_name_const:nnn { de } #1 }
%<ja>  { \@@_name_const:nnn { ja } #1 }
\clist_map_inline:nn
  {
%<*zh>
    { author           } { 姓名              } ,
    { id               } { 学号              } ,
    { supervisor       } { 导师              } ,
    { assoc_supervisor } { 副导师            } ,
    { co_supervisor    } { 联合导师          } ,
    { department       } { 院系              } ,
    { major            } { 学科 \, / \, 专业 } ,
    { degree           } { 申请学位          }
%</zh>
%<*en>
    { author           } { Author             } ,
    { supervisor       } { Supervisor         } ,
    { assoc_supervisor } { Assoc.\ Supervisor } ,
    { co_supervisor    } { Co-supervisor      }
%</en>
%<*de>
    { author           } { Autor/in         } ,
    { supervisor       } { Betreuer/in      } ,
    { assoc_supervisor } { Zweitbetreuer/in } ,
    { co_supervisor    } { Co-Betreuer/in   }
%</de>
%<*ja>
    { author           } { 氏名       } ,
    { supervisor       } { 指導教員   } ,
    { assoc_supervisor } { 副指導教員 } ,
    { co_supervisor    } { 共同指導   }
%</ja>
  }
%<zh>  { \@@_name_set:nnn { zh } #1 }
%<en>  { \@@_name_set:nnn { en } #1 }
%<de>  { \@@_name_set:nnn { de } #1 }
%<ja>  { \@@_name_set:nnn { ja } #1 }
%    \end{macrocode}
%
% \changes{v2.0.3}{2023/04/08}{更新学位论文初始英文主题。}
% 初始化主题。
%    \begin{macrocode}
%<zh>\keys_define:nn { sjtu / info / zh }
%<en>\keys_define:nn { sjtu / info / en }
%<de>\keys_define:nn { sjtu / info / de }
%<ja>\keys_define:nn { sjtu / info / ja }
  {
    subject          .initial:n =
      {
%<*zh>
        \c_@@_name_univ_zh_tl
        \c_@@_name_degree_level_zh_tl
        \c_@@_name_thesis_zh_tl
%</zh>
%<*en>
        A~ Dissertation~ Submitted~ to \\
        { \c_@@_name_univ_en_tl }~ for~
        the~ Degree~ of~ { \c_@@_name_degree_level_en_tl }
%</en>
%<*de>
        Eine~ Dissertation~ Eingereicht~ an \\
        der~ { \c_@@_name_univ_de_tl }~ für~
        { \c_@@_name_degree_level_de_tl } titel
%</de>
%<*ja>
        \c_@@_name_univ_ja_tl
        \c_@@_name_degree_level_ja_tl
        \c_@@_name_thesis_ja_tl
%</ja>
      }
  }
%</thesis-i18n>
%    \end{macrocode}
%
% 将形如 |yyyy-mm-dd| 或 |yyyy-mm| 的 ISO 日期格式字符串转化为日期表示。
%
% 日期常量。
%    \begin{macrocode}
%<*lang>
%<*zh|ja>
\clist_map_inline:nn
  {
    { year  } { 年 } ,
    { month } { 月 } ,
    { day   } { 日 }
  }
%<zh>  { \@@_name_const:nnn { zh } #1 }
%<ja>  { \@@_name_const:nnn { ja } #1 }
%</zh|ja>
%<*en>
\clist_const:Nn \c_@@_name_month_en_clist
  {
    January, February, March, April, May, June,
    July, August, September, October, November, December
  }
%</en>
%<*de>
\clist_const:Nn \c_@@_name_month_de_clist
  {
    Januar, Februar, März, April, Mai, Juni,
    Juli, August, September, Oktober, November, Dezember
  }
%</de>
%    \end{macrocode}
%
% \begin{macro}[int]{\@@_date_aux_zh:nnn,\@@_date_aux_zh:w,
% \@@_date_aux_short_zh:nn,\@@_date_aux_short_zh:w}
% 中文日期。
%    \begin{macrocode}
%<*zh>
\cs_new:Npn \@@_date_aux_zh:nnn #1#2#3
  {
    \int_to_arabic:n {#1} ~ { \exp_not:V \c_@@_name_year_zh_tl  } ~
    \int_to_arabic:n {#2} ~ { \exp_not:V \c_@@_name_month_zh_tl } ~
    \int_to_arabic:n {#3} ~ { \exp_not:V \c_@@_name_day_zh_tl   }
  }
\cs_new:Npn \@@_date_aux_zh:w #1-#2-#3 \q_stop
  { \@@_date_aux_zh:nnn {#1} {#2} {#3} }
\cs_new:Npn \@@_date_aux_short_zh:nn #1#2
  {
    \int_to_arabic:n {#1} ~ { \exp_not:V \c_@@_name_year_zh_tl  } ~
    \int_to_arabic:n {#2} ~ { \exp_not:V \c_@@_name_month_zh_tl }
  }
\cs_new:Npn \@@_date_aux_short_zh:w #1-#2 \q_stop
  { \@@_date_aux_short_zh:nn {#1} {#2} }
%</zh>
%    \end{macrocode}
% \end{macro}
%
% \begin{macro}[int]{\@@_ordinal_en:n}
% 上标形式的序数词。
%    \begin{macrocode}
%<*en>
\cs_new:Npn \@@_ordinal_en:n #1
  {
    \int_to_arabic:n {#1}
    \exp_not:N \textsuperscript
      {
        \int_case:nnF { \int_mod:nn {#1} { 100 } }
          {
            { 11 } { th }
            { 12 } { th }
            { 13 } { th }
          }
          {
            \int_case:nnF { \int_mod:nn {#1} { 10 } }
              {
                { 1 } { st }
                { 2 } { nd }
                { 3 } { rd }
              }
              { th }
          }
      }
  }
%    \end{macrocode}
% \end{macro}
%
% \begin{macro}[int]{\@@_date_aux_en:nnn,\@@_date_aux_en:w,
% \@@_date_aux_short_en:nn,\@@_date_aux_short_en:w}
% 英文日期。
%    \begin{macrocode}
\cs_new:Npn \@@_date_aux_en:nnn #1#2#3
  {
    \clist_item:Nn \c_@@_name_month_en_clist {#2} ~
    \@@_ordinal_en:n {#3} ,~
    \int_to_arabic:n {#1}
  }
\cs_new:Npn \@@_date_aux_en:w #1-#2-#3 \q_stop
  { \@@_date_aux_en:nnn {#1} {#2} {#3} }
\cs_new:Npn \@@_date_aux_short_en:nn #1#2
  {
    \clist_item:Nn \c_@@_name_month_en_clist {#2} ,~
    \int_to_arabic:n {#1}
  }
\cs_new:Npn \@@_date_aux_short_en:w #1-#2 \q_stop
  { \@@_date_aux_short_en:nn {#1} {#2} }
%</en>
%    \end{macrocode}
% \end{macro}
%
% \begin{macro}[int]{\@@_date_aux_de:nnn,\@@_date_aux_de:w,
% \@@_date_aux_short_de:nn,\@@_date_aux_short_de:w}
% 德文日期。
%    \begin{macrocode}
%<*de>
\cs_new:Npn \@@_date_aux_de:nnn #1#2#3
  {
    \clist_item:Nn \c_@@_name_month_de_clist {#2} ~
    {#3} ,~ \int_to_arabic:n {#1}
  }
\cs_new:Npn \@@_date_aux_de:w #1-#2-#3 \q_stop
  { \@@_date_aux_de:nnn {#1} {#2} {#3} }
\cs_new:Npn \@@_date_aux_short_de:nn #1#2
  {
    \clist_item:Nn \c_@@_name_month_de_clist {#2} ,~
    \int_to_arabic:n {#1}
  }
\cs_new:Npn \@@_date_aux_short_de:w #1-#2 \q_stop
  { \@@_date_aux_short_de:nn {#1} {#2} }
%</de>
%    \end{macrocode}
% \end{macro}
%
% \begin{macro}[int]{\@@_date_aux_ja:nnn,\@@_date_aux_ja:w,
% \@@_date_aux_short_ja:nn,\@@_date_aux_short_ja:w}
% 日文日期。
%    \begin{macrocode}
%<*ja>
\cs_new:Npn \@@_date_aux_ja:nnn #1#2#3
  {
    \int_to_arabic:n {#1} ~ { \exp_not:V \c_@@_name_year_ja_tl  } ~
    \int_to_arabic:n {#2} ~ { \exp_not:V \c_@@_name_month_ja_tl } ~
    \int_to_arabic:n {#3} ~ { \exp_not:V \c_@@_name_day_ja_tl   }
  }
\cs_new:Npn \@@_date_aux_ja:w #1-#2-#3 \q_stop
  { \@@_date_aux_ja:nnn {#1} {#2} {#3} }
\cs_new:Npn \@@_date_aux_short_ja:nn #1#2
  {
    \int_to_arabic:n {#1} ~ { \exp_not:V \c_@@_name_year_ja_tl  } ~
    \int_to_arabic:n {#2} ~ { \exp_not:V \c_@@_name_month_ja_tl }
  }
\cs_new:Npn \@@_date_aux_short_ja:w #1-#2 \q_stop
  { \@@_date_aux_short_ja:nn {#1} {#2} }
%</ja>
%</lang>
%    \end{macrocode}
% \end{macro}
%
% \subsection{信息录入}
%
% \changes{v2.0}{2022/12/17}{添加 \opt{display-date} 键。}
% \changes{v2.0}{2022/12/23}{使用 \opt{assoc-supervisor} 键表示副导师,
%                            使用 \opt{co-supervisor} 键表示联合导师。}
% \changes{v2.0}{2022/12/23}{添加 \opt{subject} 键。}
% \changes{v2.0}{2023/03/17}{使用语言代码前缀区别不同语种的 \opt{sjtu/info} 键。}
% \begin{macro}[int]{\@@_info_keys_define:n}
% 定义 \cls{sjtuthesis} 不同语种 |sjtu/info| 键值类的辅助函数。
%    \begin{macrocode}
%<*class>
%<*thesis>
\msg_new:nnn { sjtutex } { key-already-defined }
  { Key~'#1'~is~already~defined~and~redefination~is~being~ignored. }
\msg_new:nnn { sjtutex } { key-invalid }
  { Key~'#1'~is~invalid~and~ignored. }
\msg_new:nnn { sjtutex } { key-needs-two-arguments }
  { Key~'#1'~needs~two~arguments~otherwise~ignored. }
\cs_new_protected:Npn \@@_info_keys_define:n #1
  {
    \clist_map_inline:nn
      {
        title, display_title, subject, date, department, major,
        author, supervisor, assoc_supervisor, co_supervisor, degree
      }
      { \tl_new:c { l_@@_info_ ##1 _ #1 _tl } }
    \clist_map_inline:nn
      { keywords, fund }
      { \clist_new:c { l_@@_info_ ##1 _ #1 _clist } }
    \bool_if:NTF \g_@@_review_bool
      {
        \seq_set_from_clist:cn { l_@@_info_show_ #1 _seq }
          { author, id, supervisor, department, major, degree }
      }
      { \seq_new:c { l_@@_info_show_ #1 _seq } }
    \keys_define:nn { sjtu }
      { info / #1 .meta:nn = { sjtu / info / #1 } {##1} }
    \keys_define:nn { sjtu / info }
      {        #1 .meta:nn = { sjtu / info / #1 } {##1} }
    \keys_define:nn { sjtu / info / #1 }
      {
        display-title       .code:n =
          {
            \tl_set:co { l_@@_info_display_title_ #1 _tl }
              { \cs:w @@_title_case_aux_ #1 :n \cs_end: {##1} }
          } ,
        title               .code:n =
          {
            \tl_set:cn { l_@@_info_title_ #1 _tl } {##1}
            \tl_if_empty:cT { l_@@_info_display_title_ #1 _tl }
              { \keys_set:nn { sjtu / info / #1 } { display-title = {##1} } }
          } ,
        subject           .tl_set:c = { l_@@_info_subject_ #1 _tl } ,
        keywords       .clist_set:c = { l_@@_info_keywords_ #1 _clist } ,
        fund           .clist_set:c = { l_@@_info_fund_ #1 _clist } ,
        fund              .groups:n = { sensitive } ,
        date               .meta:nn = { sjtu / info } { date = {##1} } ,
        display-date      .tl_set:c = { l_@@_info_date_ #1 _tl } ,
        @show+              .code:n =
          {
            \seq_if_in:cnF { l_@@_info_show_ #1 _seq } {##1}
              { \seq_put_right:cn { l_@@_info_show_ #1 _seq } {##1} }
          } ,
        @show+            .groups:n = { sensitive } ,
        custom             .meta:nn = { sjtu / info / #1 / custom } {##1} ,
        custom / unknown    .code:n =
          {
            \regex_match:nVTF { \A [a-z] [a-z0-9\-]* \Z } \l_keys_key_str
              {
                \keys_if_exist:neTF { sjtu / info / #1 } \l_keys_key_str
                  {
                    \msg_warning:nne { sjtutex } { key-already-defined }
                      \l_keys_path_str
                  }
                  {
                    \int_compare:nNnTF { \tl_count:n {##1} } = { 2 }
                      {
                        \tl_set_eq:NN \l_@@_tmp_tl \l_keys_key_str
                        \tl_replace_all:Nnn \l_@@_tmp_tl { - } { _ }
                        \tl_set:co { l_@@_name_ \l_@@_tmp_tl _ #1 _tl }
                          { \use_i:nn  ##1 }
                        \tl_set:co { l_@@_info_ \l_@@_tmp_tl _ #1 _tl }
                          { \use_ii:nn ##1 }
                        \exp_args:Ne \@@_setup:n
                          { info / #1 / @show+ = \l_@@_tmp_tl }
                      }
                      {
                        \msg_warning:nne { sjtutex } { key-needs-two-arguments }
                          \l_keys_path_str
                      }
                  }
              }
              { \msg_warning:nne { sjtutex } { key-invalid } \l_keys_path_str }
          }
      }
    \clist_map_inline:nn
      {
        author, supervisor, assoc-supervisor, co-supervisor,
        department, major, degree
      }
      {
        \tl_set:Nn \l_@@_tmp_tl {##1}
        \tl_replace_all:Nnn \l_@@_tmp_tl { - } { _ }
        \keys_define:ne { sjtu / info / #1 }
          {
            ##1 .code:n =
              {
                \exp_not:N \tl_set:cn
                  { l_@@_info_ \l_@@_tmp_tl _ #1 _tl } {####1}
                \exp_not:N \@@_setup:n
                  { info / #1 / @show+ = \l_@@_tmp_tl }
              }
          }
      }
    \clist_map_inline:nn
      { author, supervisor, assoc-supervisor, co-supervisor }
      { \keys_define:nn { sjtu / info / #1 } { ##1 .groups:n = { sensitive } } }
  }
%    \end{macrocode}
% \end{macro}
%
% \begin{macro}{info/id}
% \begin{variable}{\l_@@_info_id_zh_tl}
% 单独处理学号。
%    \begin{macrocode}
\tl_new:N \l_@@_info_id_zh_tl
\keys_define:nn { sjtu / info }
  {
    id   .code:n =
      {
        \tl_set:Nn \l_@@_info_id_zh_tl {#1}
        \@@_setup:n { info / zh / @show+ = id }
      } ,
    id .groups:n = { sensitive }
  }
%    \end{macrocode}
% \end{variable}
% \end{macro}
%
% 定义 \cls{sjtuthesis} 中不同语种的信息键值类。
%    \begin{macrocode}
\clist_map_inline:Nn \g_@@_lang_clist
  { \@@_info_keys_define:n {#1} }
%</thesis>
%    \end{macrocode}
%
% 定义 \cls{sjtureport} 和 \cls{sjtuarticle} 中的信息键值类。
% \begin{variable}{\l_@@_info_subject_tl,\l_@@_info_keywords_clist}
%    \begin{macrocode}
%<*!thesis>
\tl_new:N \l_@@_info_subject_tl
\clist_new:N \l_@@_info_keywords_clist
\keys_define:nn { sjtu / info }
  {
    title             .tl_set:N = \@title ,
    author            .tl_set:N = \@author ,
    display-date      .tl_set:N = \@date ,
    subject           .tl_set:N = \l_@@_info_subject_tl ,
    keywords       .clist_set:N = \l_@@_info_keywords_clist ,
  }
%</!thesis>
%    \end{macrocode}
% \end{variable}
%
% \begin{macro}{info/date}
% 日期。
%    \begin{macrocode}
\keys_define:nn { sjtu / info }
  {
    date    .code:n =
      {
        \regex_match:nnTF { \A \d+ \- \d+ \- \d+ \Z } {#1}
          {
%<*thesis>
            \clist_map_inline:Nn \g_@@_lang_clist
              {
                \tl_set:ce { l_@@_info_date_ ##1 _tl }
                  { \exp_last_unbraced:ce { @@_date_aux_ ##1 :w } #1 \q_stop }
              }
%</thesis>
%<*!thesis>
            \tl_set:Ne \@date
              {
                \exp_last_unbraced:ce
                  { @@_date_aux_ \g_@@_lang_tl :w } #1 \q_stop
              }
%</!thesis>
          }
          {
            \regex_match:nnT { \A \d+ \- \d+ \Z } {#1}
              {
%<*thesis>
                \clist_map_inline:Nn \g_@@_lang_clist
                  {
                    \tl_set:ce { l_@@_info_date_ ##1 _tl }
                      { \exp_last_unbraced:ce { @@_date_aux_short_ ##1 :w } #1 \q_stop }
                  }
%</thesis>
%<*!thesis>
                \tl_set:Ne \@date
                  {
                    \exp_last_unbraced:ce
                      { @@_date_aux_short_ \g_@@_lang_tl :w } #1 \q_stop
                  }
%</!thesis>
              }
          }
      } ,
    date .initial:e =
      {
        \int_to_arabic:n { \c_sys_year_int  } -
        \int_to_arabic:n { \c_sys_month_int } -
        \int_to_arabic:n { \c_sys_day_int   }
      }
  }
%<thesis>\tl_set_eq:Nc \today { l_@@_info_date_ \g_@@_lang_tl _tl }
%<!thesis>\tl_set_eq:NN \today \@date
%    \end{macrocode}
% \end{macro}
%
% \subsection{特殊页面}
%
% \subsubsection{页面模板}
%
% 使用 \pkg{lttemplates} 构建页面模板,用于绘制标题页与版权页。
%
% 页面元素模板。
%    \begin{macrocode}
%<*thesis>
\NewTemplateType { sjtu / element } { 0 }
\DeclareTemplateInterface { sjtu / element } { plain } { 0 }
  {
    format      : tokenlist = \c_empty_tl ,
    content     : tokenlist = \c_empty_tl ,
    bottom-skip : skip      = \c_zero_skip ,
    align       : choice { left, right, center, normal } = center
  }
\DeclareTemplateCode { sjtu / element } { plain } { 0 }
  {
    format      = \l_@@_element_format_tl ,
    content     = \l_@@_element_content_tl ,
    bottom-skip = \l_@@_element_bottom_skip ,
    align       =
      {
        left    =
          \cs_set_eq:NN \l_@@_element_align: \raggedright ,
        right   =
          \cs_set_eq:NN \l_@@_element_align: \raggedleft ,
        center  =
          \cs_set_eq:NN \l_@@_element_align: \centering ,
        normal  =
          \cs_set_eq:NN \l_@@_element_align: \prg_do_nothing:
      }
  }
  {
    \AssignTemplateKeys
    \group_begin:
      \l_@@_element_align:
      \l_@@_element_format_tl
      \l_@@_element_content_tl
      \par
    \group_end:
    \skip_vertical:N \l_@@_element_bottom_skip
    \skip_vertical:N \c_zero_skip
  }
%    \end{macrocode}
%
% 页面模板。
%    \begin{macrocode}
\NewTemplateType { sjtu / page } { 1 }
\DeclareTemplateInterface { sjtu / page } { title } { 1 }
  {
    bookmark : tokenlist ,
    style    : tokenlist = empty ,
    format   : tokenlist = \linespread { } \selectfont ,
    precode  : tokenlist ,
    prefix   : tokenlist ,
    elements : commalist
  }
\DeclareTemplateCode { sjtu / page } { title } { 1 }
  {
    bookmark = \l_@@_page_bookmark_tl ,
    style    = \l_@@_page_style_tl ,
    format   = \l_@@_page_format_tl ,
    precode  = \l_@@_page_precode_tl ,
    prefix   = \l_@@_page_prefix_tl ,
    elements = \l_@@_page_elements_clist
  }
  {
    \tl_set_eq:NN \l_@@_page_bookmark_tl \c_novalue_tl
    \AssignTemplateKeys
    \legacy_if:nTF { @openright }
      { \cleardoublepage } { \clearpage }
    \l_@@_page_precode_tl
    \exp_args:No \thispagestyle { \l_@@_page_style_tl }
    \group_begin:
      \@@_select_language:n {#1}
      \clist_gset_eq:Nc \l_@@_info_fund_clist
        { l_@@_info_fund_ \l_@@_lang_tl _clist }
      \exp_args:No \tl_if_novalue:nF { \l_@@_page_bookmark_tl }
        { \@@_pdf_bookmark:nn { 0 } { \l_@@_page_bookmark_tl } }
      \l_@@_page_format_tl
      \clist_map_inline:Nn \l_@@_page_elements_clist
        { \UseInstance { sjtu / element } { \l_@@_page_prefix_tl / ##1 } }
      \clearpage
    \group_end:
    \restoregeometry
  }
%    \end{macrocode}
%
% 声明模板实例的辅助函数。
%    \begin{macrocode}
\cs_new_protected:Npn \@@_declare_element_instance:nnn #1#2#3
  { \DeclareInstance { sjtu / element } {#1/#2} { plain } {#3} }
\cs_new_protected:Npn \@@_declare_page_instance:nnn #1#2
  { \DeclareInstance { sjtu / page } {#1} { title } {#2} }
%    \end{macrocode}
%
% \subsubsection{标题页}
%
% \changes{v2.0}{2022/12/13}{启用新版封面。}
% \changes{v2.0.2}{2023/04/01}{标题页日期底部增加空白。}
% \changes{v2.1}{2023/11/30}{标题页信息栏改用表格实现。}
% \changes{v2.1.4}{2024/08/14}{标题页信息栏允许手动换行。}
% \changes{v2.1.5}{2024/11/06}{修复了 \pkg{array} 更新造成的标题页生成失败的问题。}
%
% \begin{variable}{\SJTU@CT@W@width}
%    \begin{macrocode}
\tl_set:Nn \SJTU@CT@W@width { 5 em }
%    \end{macrocode}
% \end{variable}
%
% \begin{macro}[int]{\SJTU@CT@W,\SJTU@CT@R,\SJTU@CT@L}
%    \begin{macrocode}
\newcolumntype { \SJTU@CT@W } [ 1 ]
  {
    w {#1} { \SJTU@CT@W@width }
    @{ \tl_use:c { c_@@_name_info_sep_ \l_@@_lang_tl _tl } }
  }
\newcolumntype { \SJTU@CT@R }
  { r @{ \tl_use:c { c_@@_name_info_sep_ \l_@@_lang_tl _tl } } }
\newcolumntype { \SJTU@CT@L }
  { >{ \linespread { 0.75 } \normalfont } l }
%    \end{macrocode}
% \end{macro}
%
% \begin{macro}[int]{\@@_info_table_row:nn,\@@_info_table_row:vv,
% \@@_title_page_info_table:n}
% 信息输出。
%    \begin{macrocode}
\cs_new:Npn \@@_info_table_row:nn #1#2
  {
    \exp_not:n {#1}
    &
    \exp_not:N \hbox:n
      {
        \exp_not:N \tabular [ t ] { @{} l @{} }
          \exp_not:n {#2}
        \exp_not:N \endtabular
      }
  }
\cs_generate_variant:Nn \@@_info_table_row:nn { vv }
\cs_new_protected:Npn \@@_title_page_info_table:n #1
  {
    \group_begin:
      \clist_clear:N \l_@@_tmp_clist
      \seq_map_inline:cn { l_@@_info_show_ \l_@@_lang_tl _seq }
        {
          \bool_lazy_all:nT
            {
              { \tl_if_exist_p:c { l_@@_name_ ##1 _ \l_@@_lang_tl _tl } }
              { \tl_if_exist_p:c { l_@@_info_ ##1 _ \l_@@_lang_tl _tl } }
            }
            {
              \clist_put_right:Ne \l_@@_tmp_clist
                {
                  \@@_info_table_row:vv
                    { l_@@_name_ ##1 _ \l_@@_lang_tl _tl }
                    { l_@@_info_ ##1 _ \l_@@_lang_tl _tl }
                }
            }
        }
      \tl_set:Nn \arraystretch { 1 }
      \int_set_eq:NN \hbadness \c_max_int
      \tabular {#1}
        \clist_use:Nn \l_@@_tmp_clist { \\ }
      \endtabular
    \group_end:
  }
%    \end{macrocode}
% \end{macro}
%
% 标题页页面样式,页脚添加资助基金信息。
%    \begin{macrocode}
\cs_new:Npn \ps@SJTU@fund
  {
    \ps@empty
    \tl_set:Nn \@oddfoot
      {
        \hfil
        \minipage [ t ] { \textwidth }
          \centering \zihao { - 5 }
          \clist_use:Nn \l_@@_info_fund_clist { \par }
        \endminipage
        \hfil
      }
    \tl_set_eq:NN \@evenfoot \@oddfoot
  }
%    \end{macrocode}
%
% 构建标题页。
%
% 样式 A 标题页元素实例。
%    \begin{macrocode}
\clist_map_inline:nn
  {
    { logo    }
      {
        content     =
          {
            \includegraphics [ width = 3 cm ]
              { sjtu-vi-badge-reg-red.pdf }
          }
      } ,
    { subject }
      {
        format      = \zihao { -2 } \setbaselineskip { 30 bp } ,
        content     = \tl_use:c { l_@@_info_subject_ \l_@@_lang_tl _tl } ,
        bottom-skip = \c_zero_dim plus 1 fill
      } ,
    { title   }
      {
        format      = \zihao { 2 } \setbaselineskip { 36 bp } \bfseries ,
        content     = \tl_use:c { l_@@_info_display_title_ \l_@@_lang_tl _tl } ,
        bottom-skip = 30 bp plus 1 fill
      } ,
    { info    }
      {
        format      = \zihao { 4 } \setbaselineskip { 30 bp } \heiti ,
        content     = \@@_title_page_info_table:n
                        { \SJTU@CT@W { s } \SJTU@CT@L } ,
        bottom-skip = 30 bp
      } ,
    { date    }
      {
        format      = \zihao { 4 } \setbaselineskip { 30 bp } \bfseries ,
        content     = \tl_use:c { l_@@_info_date_ \l_@@_lang_tl _tl } ,
      }
  }
  { \@@_declare_element_instance:nnn { title a } #1 }
%    \end{macrocode}
%
% 样式 A 标题页实例。
%    \begin{macrocode}
\@@_declare_page_instance:nnn { title a }
  {
    bookmark = \tl_use:c { c_@@_name_title_page_ \g_@@_lang_tl _tl } ,
    style    = SJTU@fund ,
    prefix   = title a ,
    elements = { logo, subject, title, info, date }
  }
%    \end{macrocode}
%
% 样式 B 标题页元素实例。
%    \begin{macrocode}
\clist_map_inline:nn
  {
    { subject }
      {
        format      = \zihao { 4 } \setbaselineskip { 24 bp } \bfseries ,
        content     = \tl_use:c { l_@@_info_subject_ \l_@@_lang_tl _tl } ,
        bottom-skip = \c_zero_dim plus 1 fill
      } ,
    { title   }
      {
        format      = \zihao { -2 } \setbaselineskip { 30 bp } \bfseries ,
        content     = \tl_use:c { l_@@_info_display_title_ \l_@@_lang_tl _tl } ,
        bottom-skip = \c_zero_dim plus 1 fill
      } ,
    { info    }
      {
        format      = \zihao { 3 } \setbaselineskip { 30 bp } \bfseries ,
        content     = \@@_title_page_info_table:n
                        { \SJTU@CT@R \SJTU@CT@L } ,
        bottom-skip = 30 bp plus 1 fill
      } ,
    { date    }
      {
        format      = \zihao { 3 } \setbaselineskip { 30 bp } ,
        content     =
          {
            \tl_use:c { l_@@_info_department_ \l_@@_lang_tl _tl }
            \skip_vertical:N \c_zero_skip
            \tl_use:c { c_@@_name_univ_ \l_@@_lang_tl _tl }
            \skip_vertical:N \c_zero_skip
            \tl_use:c { c_@@_name_address_ \l_@@_lang_tl _tl }
            \skip_vertical:N \c_zero_skip
            \tl_use:c { l_@@_info_date_ \l_@@_lang_tl _tl }
          }
      }
  }
  { \@@_declare_element_instance:nnn { title b } #1 }
%    \end{macrocode}
%
% 样式 B 标题页实例。
%    \begin{macrocode}
\@@_declare_page_instance:nnn { title b }
  {
    prefix   = title b ,
    style    = SJTU@fund ,
    elements = { subject, title, info, date }
  }
%    \end{macrocode}
%
% 载入学位论文语言配置文件。
%    \begin{macrocode}
\clist_map_inline:Nn \g_@@_lang_clist
  { \file_input:n { sjtu-thesis- #1 .def } }
%</thesis>
%</class>
%    \end{macrocode}
%
% 中文标题页使用样式 A,其他语种使用样式 B。
%    \begin{macrocode}
%<*thesis-i18n>
\DeclareInstanceCopy { sjtu / page }
%<zh>  { title / zh } { title a }
%<en>  { title / en } { title b }
%<de>  { title / de } { title b }
%<ja>  { title / ja } { title b }
%</thesis-i18n>
%    \end{macrocode}
%
% \begin{macro}{\maketitle}
% 生成标题页。
%    \begin{macrocode}
%<*class&thesis>
\RenewDocumentCommand \maketitle { }
  {
    \clist_map_inline:Nn \g_@@_lang_clist
      { \UseInstance { sjtu / page } { title / ##1 } { ##1 } }
  }
%</class&thesis>
%    \end{macrocode}
% \end{macro}
%
% \subsubsection{原创性声明及使用授权书}
%
% \changes{v2.2.1}{2025/03/22}{添加原创性声明变体。}
%
% 声明及授权书文本。
%    \begin{macrocode}
%<*thesis-i18n&zh>
\tl_const:Nn \c_@@_orig_decl_text_a_zh_tl
  {
    本人郑重声明:所呈交的学位论文,是本人在导师的指导下,独立进行研究工
    作所取得的成果。
    除文中已经注明引用的内容外,本论文不包含任何其他个人或集体已经发表或
    撰写过的作品成果。
    对本文的研究做出重要贡献的个人和集体,均已在文中以明确方式标明。
    本人完全知晓本声明的法律后果由本人承担。
  }
\tl_const:Nn \c_@@_orig_decl_text_b_zh_tl
  {
    本人郑重声明:所呈交的学位论文,是本人在导师的指导下,独立进行研究工
    作所取得的成果。
    除文中已经注明引用的内容外,本论文不包含任何其他个人或集体已经发表或
    撰写过的作品成果。
    对本文的研究做出重要贡献的个人和集体,已在文中以适当方式予以致谢。
    若在论文撰写过程中使用了人工智能工具,本人已遵循《上海交通大学关于在
    教育教学中使用~ AI~ 的规范》,确保人工智能生成内容的应用场景、引用范
    围及标注方式均符合规定,并杜绝学术不端行为。
    本人完全知晓本声明的法律后果由本人承担。
  }
\tl_const:Nn \c_@@_auth_decl_text_zh_tl
  {
    本人同意学校保留并向国家有关部门或机构送交论文的复印件和电子版,允许
    论文被查阅和借阅。 \par
    \vskip 6 bp
    \noindent
    本学位论文属于: \par
    { \c_@@_symbol_white_square_tl } \, \textbf { 公开论文 } \par
    { \c_@@_symbol_white_square_tl } \, \textbf { 内部论文 },
      保密 \, { \c_@@_symbol_white_square_tl } \, 1 \, 年 \, /
           \, { \c_@@_symbol_white_square_tl } \, 2 \, 年 \, /
           \, { \c_@@_symbol_white_square_tl } \, 3 \, 年,
      过保密期后适用本授权书。 \par
    { \c_@@_symbol_white_square_tl } \, \textbf { 秘密论文 },
      保密 \, \underline { \hspace { 2 em } } \, 年(不超过~ 10~ 年),
      过保密期后适用本授权书。 \par
    { \c_@@_symbol_white_square_tl } \, \textbf { 机密论文 },
      保密 \, \underline { \hspace { 2 em } } \, 年(不超过~ 20~ 年),
      过保密期后适用本授权书。 \par
    \hspace { 6 em }(请在以上方框内选择打“ \ensuremath { \checkmark } ”)
  }
%    \end{macrocode}
%
% 签名框。
%    \begin{macrocode}
\tl_const:Nn \c_@@_signature_text_zh_tl
  {
    签名: \\
    日期: \hspace { \stretch { 3 } } 年
           \hspace { \stretch { 2 } } 月
           \hspace { \stretch { 2 } } 日
  }
\cs_new_protected:Npn \@@_signature:N #1
  {
    \parbox [ t ] { 12 em }
      { #1 \c_@@_signature_text_zh_tl }
  }
%</thesis-i18n&zh>
%    \end{macrocode}
%
% 声明及授权书元素实例。
%    \begin{macrocode}
%<*class>
%<*thesis>
\clist_map_inline:nn
  {
    { orig / title }
      {
        format      = \zihao { 3 } \setbaselineskip { 30 bp }
                      \bfseries \heiti ,
        content     =
          {
            \c_@@_name_univ_zh_tl
            \skip_vertical:N \c_zero_skip
            \c_@@_name_thesis_zh_tl
            \c_@@_name_orig_decl_zh_tl
          } ,
        bottom-skip = 12 bp
      },
    { orig / text  }
      {
        format      = \zihao { -4 } \setbaselineskip { 24 bp } ,
        bottom-skip = 24 bp ,
        align       = normal
      },
    { orig / sign  }
      {
        format      = \zihao { 4 } \setbaselineskip { 30 bp } ,
        content     =
          {
            \@@_signature:N \c_@@_name_decl_author_zh_tl
            \skip_horizontal:n { 2 em } \hbox:n { }
          } ,
        bottom-skip = 24 bp plus 1 fill ,
        align       = right
      },
    { auth / title }
      {
        format      = \zihao { 3 } \setbaselineskip { 30 bp }
                      \bfseries \heiti ,
        content     =
          {
            \c_@@_name_univ_zh_tl
            \skip_vertical:N \c_zero_skip
            \c_@@_name_thesis_zh_tl
            \c_@@_name_auth_decl_zh_tl
          } ,
        bottom-skip = 12 bp
      },
    { auth / text  }
      {
        format      = \zihao { -4 } \setbaselineskip { 24 bp } ,
        content     = \c_@@_auth_decl_text_zh_tl ,
        bottom-skip = 24 bp ,
        align       = normal
      },
    { auth / sign  }
      {
        format      = \zihao { 4 } \setbaselineskip { 30 bp } ,
        content     =
          {
            \@@_signature:N \c_@@_name_decl_author_zh_tl
            \hfill
            \@@_signature:N \c_@@_name_decl_supervisor_zh_tl
          } ,
        bottom-skip = \c_zero_dim plus 1 fill ,
        align       = normal
      }
  }
  {
    \@@_declare_element_instance:nnn { copyright } #1
  }
%    \end{macrocode}
%
% 声明及授权书模板实例。
%    \begin{macrocode}
\@@_declare_page_instance:nnn { copyright }
  {
    bookmark  = \tl_use:c { c_@@_name_declaration_ \g_@@_lang_tl _tl } ,
    precode   = \newgeometry
                  {
                    margin        = 3.0 cm ,
                    bindingoffset = 0.5 cm
                  } ,
    prefix    = copyright ,
    elements  =
      {
        orig / title, orig / text, orig / sign,
        auth / title, auth / text, auth / sign
      }
  }
%    \end{macrocode}
%
% \begin{variable}{\l_@@_copyright_page_file_tl}
%    \begin{macrocode}
\tl_new:N \l_@@_copyright_page_file_tl
%    \end{macrocode}
% \end{variable}
%
%    \begin{macrocode}
\keys_define:nn { sjtu / private / copyright }
  {
    file .tl_set:N = \l_@@_copyright_page_file_tl ,
    file .value_required:n = true ,
    filename .meta:n = { file = #1 } ,
    variant .choice: ,
    variant .choices:nn =
      { a, b }
      {
        \EditInstance { sjtu / element } { copyright / orig / text }
          { content = \tl_use:c { c_@@_orig_decl_text_ #1 _zh_tl } }
      } ,
    variant  .initial:e = \int_compare:nNnTF { \g_@@_thesis_type_int } > { 1 }
                            { a } { b }
  }
%    \end{macrocode}
%
% \changes{v2.0}{2022/12/20}{不再自动载入 \pkg{pdfpages} 宏包。}
%    \begin{macrocode}
\msg_new:nnn { sjtutex } { require-pdfpages }
  {
    Add~'\token_to_str:N \usepackage{pdfpages}'~in~your~preamble \\
    before~inserting~pages~of~external~PDF.
  }
%    \end{macrocode}
%
% \begin{macro}{\copyrightpage}
% 生成声明及授权书。
%    \begin{macrocode}
\NewDocumentCommand \copyrightpage { O{ } }
  {
    \bool_if:NF \g_@@_review_bool
      {
        \group_begin:
          \keys_set_known:nnN { sjtu / private / copyright }
            {#1} \l_@@_tmp_tl
          \tl_if_empty:NTF \l_@@_copyright_page_file_tl
            { \UseInstance { sjtu / page } { copyright } { zh } }
            {
              \cs_if_exist:NTF \includepdf
                {
                  \legacy_if:nTF { @openright }
                    { \cleardoublepage } { \clearpage }
                  \exp_args:Nnv \@@_pdf_bookmark:nn
                    { 0 } { c_@@_name_declaration_ \g_@@_lang_tl _tl }
                  \tl_set_rescan:NnV \l_@@_tmp_tl { } \l_@@_tmp_tl
                  \exp_args:NNo \includepdf [ \l_@@_tmp_tl ]
                    { \l_@@_copyright_page_file_tl }
                }
                {
                  \msg_warning:nn { sjtutex } { require-pdfpages }
                  \UseInstance { sjtu / page } { copyright } { zh }
                }
            }
        \group_end:
      }
  }
%</thesis>
%    \end{macrocode}
% \end{macro}
%
% \subsection{文档组成}
%
% \changes{v2.2}{2024/12/20}{内置文档命令与环境修改为使用键值对参数,废弃对应的带星号版本。}
%
% \subsubsection{统一接口}
%
% 定义文档组成部分的统一键值对接口。
%
% \begin{variable}{\l_@@_heading_numbering_bool,\l_@@_heading_in_toc_bool,
% \l_@@_heading_title_tl,\l_@@_heading_marking_tl}
%    \begin{macrocode}
\bool_new:N \l_@@_heading_numbering_bool
\bool_new:N \l_@@_heading_in_toc_bool
\tl_new:N \l_@@_heading_title_tl
\tl_new:N \l_@@_heading_marking_tl
%    \end{macrocode}
% \end{variable}
%
% 默认 \tn{mainmatter} 之后的章节标题加入目录。
%    \begin{macrocode}
%<*thesis>
\AddToHook { cmd / mainmatter / after }
  { \bool_set_true:N \l_@@_heading_in_toc_bool }
%</thesis>
%    \end{macrocode}
%
% \begin{macro}{intoc,notintoc,title,marking}
%    \begin{macrocode}
\keys_define:nn { sjtu / private }
  {
    intoc    .value_forbidden:n = true,
    intoc    .groups:n = { heading } ,
    intoc    .code:n = { \bool_set_true:N  \l_@@_heading_in_toc_bool } ,
    notintoc .value_forbidden:n = true,
    notintoc .groups:n = { heading } ,
    notintoc .code:n = { \bool_set_false:N \l_@@_heading_in_toc_bool } ,
    title    .tl_set:N = \l_@@_heading_title_tl ,
    title    .groups:n = { heading } ,
    marking  .tl_set:N = \l_@@_heading_marking_tl ,
    marking  .groups:n = { heading }
  }
%    \end{macrocode}
% \end{macro}
%
% \begin{macro}[int]{\SJTU@heading,\@@_heading:nn,\@@_heading:VV,
% \@@_make_heading:n}
% 定义一个灵活的章节标题命令专门处理不同的需求。
%    \begin{macrocode}
\NewDocumentCommand \SJTU@heading { O{#2} m }
  {
    \bool_if:nTF
      { \l_@@_heading_numbering_bool && \l_@@_heading_in_toc_bool }
%<!article>      { \chapter [#1] {#2} }
%<article>      { \section [#1] {#2} }
      {
%<!article>        \CTEX@chapter@break
        \bool_if:NTF \l_@@_heading_in_toc_bool
          {
            \@@_phantom_section:
%<!article>            \addcontentsline { toc } { chapter } {#1}
%<article>            \addcontentsline { toc } { section } {#1}
          }
          { \@@_pdf_bookmark:nn { 0 } {#1} }
%<!article>        \chapter* {#2}
%<article>        \section* {#2}
        \CTEX@gettitle {#1}
        \@mkboth { \MakeUppercase {#1} } { \MakeUppercase {#1} }
      }
  }
\cs_new_protected:Npn \@@_heading:nn #1#2
  { \SJTU@heading [#1] {#2} }
\cs_generate_variant:Nn \@@_heading:nn { VV }
\cs_new_protected:Npn \@@_make_heading:n #1
  {
    \tl_set_eq:NN \l_@@_heading_marking_tl \c_novalue_tl
    \keys_set_groups:nnn { sjtu / private } { heading } {#1}
    \@@_heading:VV
      \l_@@_heading_marking_tl \l_@@_heading_title_tl
  }
%    \end{macrocode}
% \end{macro}
%
% \subsubsection{摘要}
%
% \changes{v2.1}{2023/11/30}{新增 \opt{style/keywords-format} 选项。}
% \begin{macro}{style/keywords-format}
% 关键词排版样式。
%    \begin{macrocode}
\keys_define:nn { sjtu / style }
  {
    keywords-format       .choice: ,
    keywords-format / plain .code:n =
      { \cs_set:Nn \@@_keywords_format:n { \noindent  { \bfseries ##1 } } } ,
    keywords-format / hang  .code:n =
      { \cs_set:Nn \@@_keywords_format:n { \@hangfrom { \bfseries ##1 } } } ,
    keywords-format      .initial:n = { plain }
  }
%    \end{macrocode}
% \end{macro}
%
% \changes{v2.0}{2023/03/17}{\env{abstract} 环境新增指定语言的可选参数。}
% \changes{v2.1}{2023/11/29}{\env{abstract} 环境添加目录条目,
% \env{abstract*} 环境对应修改为不添加目录条目。}
% \begin{macro}{abstract}
% 学位论文摘要环境。
%    \begin{macrocode}
%<*thesis>
\DeclareDocumentEnvironment { abstract } { O{ } }
  {
    \@@_select_language:n {#1}
    \tl_set_eq:Nc \l_@@_heading_title_tl
      { c_@@_name_abstract_ \l_@@_lang_tl _tl }
    \@@_make_heading:n {#1}
  }
  {
    \clist_if_empty:cF { l_@@_info_keywords_ \l_@@_lang_tl _clist }
      {
        \par \mode_leave_vertical: \par
        \@@_keywords_format:n
          {
            \tl_use:c { c_@@_name_keywords_ \l_@@_lang_tl _tl }
            \tl_use:c { c_@@_name_info_sep_ \l_@@_lang_tl _tl }
          }
        \clist_use:cv { l_@@_info_keywords_ \l_@@_lang_tl _clist }
          { c_@@_name_item_sep_ \l_@@_lang_tl _tl }
        \par
      }
  }
%</thesis>
%    \end{macrocode}
%
% 修复通用模板摘要段首缩进。
%    \begin{macrocode}
%<*!thesis>
\legacy_if:nT { @titlepage }
  { \AddToHook { cmd / abstract / after } { \par } }
%    \end{macrocode}
%
% 通用模板摘要后添加关键词。
%    \begin{macrocode}
\AddToHook { env / abstract / end }
  {
    \clist_if_empty:NF \l_@@_info_keywords_clist
      {
        \par \mode_leave_vertical: \par
        \@@_keywords_format:n
          {
            \tl_use:c { c_@@_name_keywords_ \g_@@_lang_tl _tl }
            \tl_use:c { c_@@_name_info_sep_ \g_@@_lang_tl _tl }
          }
        \clist_use:Nv \l_@@_info_keywords_clist
          { c_@@_name_item_sep_ \g_@@_lang_tl _tl }
        \par
      }
  }
%</!thesis>
%    \end{macrocode}
% \end{macro}
%
% \subsubsection{目录列表}
%
% \changes{v2.1}{2023/11/29}{\tn{tableofcontents} 添加目录条目,
% \tn{tableofcontents*} 对应修改为不添加目录条目。}
% \begin{macro}{\tableofcontents}
% 目录。
%    \begin{macrocode}
\DeclareDocumentCommand \tableofcontents { O{ } }
  {
    \group_begin:
      \@@_make_heading:n { title = \contentsname, #1 }
      \@starttoc { toc }
    \group_end:
  }
%    \end{macrocode}
% \end{macro}
%
% \changes{v2.1.2}{2024/03/31}{改用 \pkg{titletoc} 设置目录格式。}
% \changes{v2.1.3}{2024/06/28}{更新目录样式。}
%    \begin{macrocode}
\tl_set:Nn \SJTU@leaders { \titlerule* [ 4bp ] { . } }
\contentsmargin [ 2.55 em ] { 0 pt }
%<article>\titlecontents { section }
%<!article>\titlecontents { chapter }
  [ 0 pt ] { \addvspace { 6 bp } \bfseries }
  { \contentspush { \thecontentslabel \enskip } } { }
  { \SJTU@leaders \thecontentspage }
%<article>\titlecontents { subsection }
%<!article>\titlecontents { section }
  [ 2 em ] { }
  { \contentspush { \thecontentslabel \enskip } } { }
  { \SJTU@leaders \thecontentspage }
%<article>\titlecontents { subsubsection }
%<!article>\titlecontents { subsection }
  [ 4 em ] { }
  { \contentspush { \thecontentslabel \enskip } } { }
  { \SJTU@leaders \thecontentspage }
%    \end{macrocode}
%
% \changes{v2.0.1}{2023/03/31}{插图、表格和算法等索引不缩进。}
% \changes{v2.0.1}{2023/03/31}{调整插图、表格和算法等索引编号宽度。}
% \begin{macro}[int]{\SJTU@listof,\@@_new_list_of:Nnnn}
% \begin{macro}{\listoffigures,\listoftables}
% 图表索引。
%    \begin{macrocode}
\NewDocumentCommand \SJTU@listof { m m O{ } }
  {
    \group_begin:
      \@@_make_heading:n { title = #1, #3 }
      \exp_args:Nv \@starttoc { ext@ #2 }
    \group_end:
  }
\cs_new_protected:Npn \@@_new_list_of:Nnnn #1#2#3#4
  {
    \DeclareDocumentCommand #1 { }
      { \SJTU@listof {#4} {#2} }
    \titlecontents {#2}
      [ 0 pt ] { }
      { \contentspush { #3 \space \thecontentslabel \enskip } } { }
      { \SJTU@leaders \thecontentspage }
    \exp_args:Nnv \contentsuse {#2} { ext@ #2 }
  }
\@@_new_list_of:Nnnn \listoffigures { figure }
  { \figurename } { \listfigurename }
\@@_new_list_of:Nnnn \listoftables  { table  }
  { \tablename  } { \listtablename  }
%    \end{macrocode}
% \end{macro}
% \end{macro}
%
% \subsubsection{预定义环境}
%
% \begin{macro}{abbreviation}
% 缩略语对照表。
%    \begin{macrocode}
%<*thesis>
\NewDocumentEnvironment { abbreviation } { O{ } }
  {
    \bool_set_true:N \l_@@_heading_numbering_bool
    \@@_make_heading:n { title = \SJTU@abbrname, #1 }
    \tl_clear:N \SJTU@style@float@font
  } { }
%    \end{macrocode}
% \end{macro}
%
% \begin{macro}{nomenclature}
% 符号对照表。
%    \begin{macrocode}
\NewDocumentEnvironment { nomenclature } { O{ } }
  {
    \bool_set_true:N \l_@@_heading_numbering_bool
    \@@_make_heading:n { title = \SJTU@nomname, #1 }
    \tl_clear:N \SJTU@style@float@font
  } { }
%    \end{macrocode}
% \end{macro}
%
% \changes{v2.0}{2023/03/17}{移除 \env{summary} 环境。}
%
% \begin{macro}{acknowledgements}
% 致谢,盲审模式下隐藏致谢。
%    \begin{macrocode}
\NewDocumentEnvironment { acknowledgements } { O{ } +b }
  {
    \bool_if:NF \g_@@_review_bool
      {
        \@@_select_language:n {#1}
        \@@_make_heading:n { title = \SJTU@ackname, #1 }
        #2
      }
  } { }
%    \end{macrocode}
% \end{macro}
%
% \changes{v2.2}{2024/12/08}{移除 \env{bibliolist} 环境的使用限制,
% 默认重置计数,新增 \opt{resume} 选项。}
% \begin{macro}{achievements,bibliolist,bibliolist*}
% 发表论文与学术成果。
%    \begin{macrocode}
\newcounter { SJTU@bib }
\bool_new:N \l_@@_bibliolist_resume_bool
\keys_define:nn { sjtu / private / bibliolist }
  {
    resume .value_forbidden:n = true,
    resume .code:n =
      { \bool_set_true:N \l_@@_bibliolist_resume_bool }
  }
\msg_new:nnn { sjtutex } { empty-environment }
  { Empty~'#1'~environment. }
\NewDocumentEnvironment { @bibliolist } { m m }
  {
    \keys_set:nn { sjtu / private / bibliolist } {#1}
    \cs_if_exist_use:N \bibfont
    \list
      {
        \tl_if_blank:nTF {#2}
          { \hfill }
          { \@biblabel { \arabic{ SJTU@bib } } }
      }
      {
        \tl_if_blank:nTF {#2}
          {
            \skip_if_exist:NTF \bibhang
              { \dim_set_eq:NN \leftmargin \bibhang }
              { \dim_set:Nn    \leftmargin { 1 em } }
            \dim_set:Nn \itemindent { - \leftmargin }
          }
          {
            \@@_dim_set_to_wd:Nn \labelwidth { \@biblabel {#2} }
            \dim_set_eq:NN \leftmargin \labelwidth
            \dim_add:Nn    \leftmargin { \labelsep }
          }
        \skip_if_exist:NTF \bibitemsep
          {
            \skip_set_eq:NN \itemsep \bibitemsep
            \skip_if_exist:NT \bibparsep
              { \skip_set_eq:NN \parsep \bibparsep }
          }
          {
            \skip_if_exist:NT \bibsep
              {
                \skip_set_eq:NN \itemsep \bibsep
                \skip_zero:N    \parsep
              }
          }
        \@nmbrlisttrue
        \tl_set:Nn \@listctr { SJTU@bib }
        \tl_clear:N \p@SJTU@bib
        \bool_if:NF \l_@@_bibliolist_resume_bool
          { \setcounter { SJTU@bib } { 0 } }
        \tl_set:Nn \theSJTU@bib { \arabic { SJTU@bib } }
      }
      \sloppy
      \int_set:Nn \clubpenalty  { 4000 }
      \int_set_eq:NN \@clubpenalty \clubpenalty
      \int_set:Nn \widowpenalty { 4000 }
      \char_set_sfcode:nn { `\. } { 1000 }
  }
  {
    \tl_set:Nn \@noitemerr
      { \msg_warning:nnn { sjtutex } { empty-environment } { bibliolist } }
    \endlist
  }
\NewDocumentEnvironment { achievements } { O{ } }
  {
    \@@_select_language:n {#1}
    \@@_make_heading:n { title = \SJTU@achvname, #1 }
  } { }
\NewDocumentEnvironment { bibliolist  } { O{ } m +b }
  {
    \bool_if:NF \g_@@_review_bool
      {
        \begin { @bibliolist } {#1} {#2}
          #3
        \end { @bibliolist }
      }
  } { }
\NewDocumentEnvironment { bibliolist* } { O{ } m +b }
  {
    \bool_if:NT \g_@@_review_bool
      {
        \begin { @bibliolist } {#1} {#2}
          #3
        \end { @bibliolist }
      }
  } { }
%    \end{macrocode}
% \end{macro}
%
% \begin{macro}{resume}
% 简历。
%    \begin{macrocode}
\NewDocumentEnvironment { resume } { O{ } +b }
  {
    \bool_if:NF \g_@@_review_bool
      {
        \@@_select_language:n {#1}
        \@@_make_heading:n { title = \SJTU@resumename, #1 }
        #2
      }
  } { }
%    \end{macrocode}
% \end{macro}
%
% \changes{v2.0}{2023/03/17}{\env{digest} 环境新增指定语言的可选参数。}
% \begin{macro}{digest}
% 大摘要。
%    \begin{macrocode}
\NewHook { sjtutex / digest }
\NewDocumentEnvironment { digest } { O{ } +b }
  {
    \AtEndDocument
      {
        \group_begin:
          \@@_select_language:n { en, #1 }
          \legacy_if:nTF { @openright }
            { \cleardoublepage } { \clearpage }
          \pagenumbering { roman }
          \tl_gset:Nn \@@_the_page_tl { \arabic { page } }
          \UseHook { sjtutex / digest }
          \cs_gset_eq:NN \addcontentsline \use_none:nnn
          \clist_map_inline:Nn \l_@@_counter_without_chapter_clist
            {
              \counterwithout {##1} { chapter }
              \setcounter     {##1} { 0 }
            }
          \tl_set_eq:Nc \l_@@_heading_title_tl
            { l_@@_info_display_title_ \l_@@_lang_tl _tl }
          \@@_make_heading:n { marking = \SJTU@digestname, #1, notintoc }
          #2
        \group_end:
      }
  } { }
%</thesis>
%    \end{macrocode}
% \end{macro}
%
% \subsection{设置接口}
%
% \begin{macro}{\sjtusetup}
% 用户设置接口。
% 盲审模式下不会设置敏感信息。
%    \begin{macrocode}
%<!thesis>\NewDocumentCommand \sjtusetup { } { \keys_set:nn { sjtu } }
%<thesis>\NewDocumentCommand \sjtusetup { } { \@@_setup:n }
%    \end{macrocode}
% \end{macro}
%
% \changes{v2.0}{2022/12/28}{重新制定 \opt{sjtu/style} 域中的接口。}
% 定义元(meta)键值对。
%    \begin{macrocode}
\keys_define:nn { sjtu }
  {
    style .meta:nn = { sjtu / style } {#1} ,
    info  .meta:nn = { sjtu / info  } {#1} ,
    name  .meta:nn = { sjtu / name  } {#1}
  }
%    \end{macrocode}
%
% 兼容 \cls{sjtuthesis} 旧接口。
%    \begin{macrocode}
%<*thesis>
\keys_define:nn { sjtu / info }
  {
    title             .meta:n = { zh / title            = {#1} } ,
    title*            .meta:n = { en / title            = {#1} } ,
    display-title     .meta:n = { zh / display-title    = {#1} } ,
    display-title*    .meta:n = { en / display-title    = {#1} } ,
    subject           .meta:n = { zh / subject          = {#1} } ,
    subject*          .meta:n = { en / subject          = {#1} } ,
    keywords          .meta:n = { zh / keywords         = {#1} } ,
    keywords*         .meta:n = { en / keywords         = {#1} } ,
    author            .meta:n = { zh / author           = {#1} } ,
    author*           .meta:n = { en / author           = {#1} } ,
    supervisor        .meta:n = { zh / supervisor       = {#1} } ,
    supervisor*       .meta:n = { en / supervisor       = {#1} } ,
    assoc-supervisor  .meta:n = { zh / assoc-supervisor = {#1} } ,
    assoc-supervisor* .meta:n = { en / assoc-supervisor = {#1} } ,
    co-supervisor     .meta:n = { zh / co-supervisor    = {#1} } ,
    co-supervisor*    .meta:n = { en / co-supervisor    = {#1} } ,
    degree            .meta:n = { zh / degree           = {#1} } ,
    degree*           .meta:n = { en / degree           = {#1} } ,
    department        .meta:n = { zh / department       = {#1} } ,
    department*       .meta:n = { en / department       = {#1} } ,
    major             .meta:n = { zh / major            = {#1} } ,
    major*            .meta:n = { en / major            = {#1} } ,
    fund              .meta:n = { zh / fund             = {#1} } ,
    fund*             .meta:n = { en / fund             = {#1} } ,
    display-date      .meta:n = { zh / date             = {#1} } ,
    display-date*     .meta:n = { en / date             = {#1} }
  }
\keys_define:nn { sjtu / name }
  {
    abbreviation      .meta:n = { abbr  = {#1} } ,
    nomenclature      .meta:n = { nom   = {#1} } ,
    acknowledgements  .meta:n = { ack   = {#1} } ,
    publications      .meta:n = { achv  = {#1} } ,
    achievements      .meta:n = { achv  = {#1} }
  }
%</thesis>
%    \end{macrocode}
%
% \begin{macro}{\subject,\keywords}
% 通用模板新接口。
%    \begin{macrocode}
%<*!thesis>
\NewDocumentCommand \subject  { m }
  { \keys_set:nn { sjtu / info } { subject  = {#1} } }
\NewDocumentCommand \keywords { m }
  { \keys_set:nn { sjtu / info } { keywords = {#1} } }
%</!thesis>
%    \end{macrocode}
% \end{macro}
%
% \subsection{其他宏包的设置}
%
% \changes{v2.0.3}{2023/09/23}{移除 \pkg{listings} 宏包预设。}
% 这些宏包并非格式要求,但是为了方便同学们使用,在这里进行简单设置。
%
% \subsubsection{\pkg{hyperref} 宏包}
%
%    \begin{macrocode}
\ctex_at_end_package:nn { hyperref }
  {
    \hypersetup
      {
        linktoc            = all,
        bookmarksdepth     = 2,
        bookmarksnumbered  = true,
        bookmarksopen      = true,
        bookmarksopenlevel = 1,
        unicode            = true,
        psdextra           = true,
        breaklinks         = true,
        pdfdisplaydoctitle = true
      }
    \int_new:N \g_@@_bookmark_int
    \cs_set_protected:Npn \@@_pdf_bookmark:nn #1#2
      {
        \phantomsection
        \int_gincr:N \g_@@_bookmark_int
        \pdfbookmark [#1] {#2}
          { sjtubookmark. \int_use:N \g_@@_bookmark_int }
      }
    \cs_set_eq:NN \@@_phantom_section: \phantomsection
    \DeclareExpandableDocumentCommand
      { \@@_pdfstring_newline:w } { s o m } {#3}
    \pdfstringdefDisableCommands
      {
        \cs_set_eq:NN \\       \@@_pdfstring_newline:w
        \cs_set_eq:NN \hspace  \use_none:n
        \cs_set_eq:NN \zihao   \use_none:n
        \tl_set_eq:NN \quad    \c_empty_tl
        \tl_set_eq:NN \qquad   \c_empty_tl
      }
    \ctex_after_end_preamble:n
      {
        \hypersetup
          {
%<*thesis>
            pdftitle    = \l_@@_info_title_zh_tl ,
            pdfauthor   = \l_@@_info_author_zh_tl ,
            pdfsubject  = \l_@@_info_subject_zh_tl ,
            pdfkeywords = \l_@@_info_keywords_zh_clist
%</thesis>
%<*!thesis>
            pdftitle    = \@title ,
            pdfauthor   = \@author ,
            pdfsubject  = \l_@@_info_subject_tl ,
            pdfkeywords = \l_@@_info_keywords_clist
%</!thesis>
          }
      }
  }
%    \end{macrocode}
%
% \changes{v2.2}{2024/12/30}{使用 \pkg{pageslts} 宏包获取总页码。}
% \subsubsection{\pkg{pageslts} 宏包}
%
%    \begin{macrocode}
\ctex_at_end_package:nn { pageslts }
  {
%<thesis>    \pagenumbering { Alph }
%<!thesis>    \pagenumbering { arabic }
    \tl_set:Nn \@@_the_last_page_tl
      { \lastpageref { pagesLTS.\pagesLTS@pnc } }
%<*thesis>
    \AddToHook { sjtutex / digest }
      {
        \tl_gset:Nn \@@_the_last_page_tl
          { \lastpageref { pagesLTS.roman.local } }
      }
%</thesis>
  }
%    \end{macrocode}
%
% \subsubsection{\pkg{threeparttable} 宏包}
%
%    \begin{macrocode}
\ctex_at_end_package:nn { threeparttable }
  { \tl_put_right:Nn \TPTnoteSettings { \footnotesize } }
%    \end{macrocode}
%
% \subsubsection{\pkg{longtable} 宏包}
%
%    \begin{macrocode}
\ctex_at_end_package:nn { longtable }
  { \AtBeginEnvironment { longtable } { \SJTU@style@float@font } }
%</class>
%    \end{macrocode}
%
% \changes{v2.1.1}{2024/03/21}{预定义的数学环境声明移至导言区末尾,
% 且不会覆盖重名的已定义环境。}
% \subsubsection{\pkg{amsthm} 宏包和 \pkg{ntheorem} 宏包}
%
% 预定义的数学环境,不包括证明环境 \env{proof}。
%    \begin{macrocode}
%<*scheme>
\clist_map_inline:nn
  {
%<*zh>
    { assumption  } { 假设        } ,
    { axiom       } { 公理        } ,
    { conjecture  } { 猜想        } ,
    { corollary   } { 推论        } ,
    { definition  } { 定义        } ,
    { example     } { 例          } ,
    { exercise    } { 练习        } ,
    { lemma       } { 引理        } ,
    { problem     } { 问题        } ,
    { proposition } { 命题        } ,
    { remark      } { 注          } ,
    { solution    } { 解          } ,
    { theorem     } { 定理        }
%</zh>
%<*en>
    { assumption  } { Assumption  } ,
    { axiom       } { Axiom       } ,
    { conjecture  } { Conjecture  } ,
    { corollary   } { Corollary   } ,
    { definition  } { Definition  } ,
    { example     } { Example     } ,
    { exercise    } { Exercise    } ,
    { lemma       } { Lemma       } ,
    { problem     } { Problem     } ,
    { proposition } { Proposition } ,
    { remark      } { Remark      } ,
    { solution    } { Solution    } ,
    { theorem     } { Theorem     }
%</en>
%<*de>
    { assumption  } { Annahme     } ,
    { axiom       } { Axiom       } ,
    { conjecture  } { Hypothese   } ,
    { corollary   } { Korollar    } ,
    { definition  } { Definition  } ,
    { example     } { Beispiel    } ,
    { exercise    } { Übung       } ,
    { lemma       } { Lemma       } ,
    { problem     } { Problem     } ,
    { proposition } { Proposition } ,
    { remark      } { Anmerkung   } ,
    { solution    } { Lösung      } ,
    { theorem     } { Theorem     }
%</de>
%<*ja>
    { assumption  } { 仮定        } ,
    { axiom       } { 公理        } ,
    { conjecture  } { 予想        } ,
    { corollary   } { 系          } ,
    { definition  } { 定義        } ,
    { example     } { 例          } ,
    { exercise    } { 練習        } ,
    { lemma       } { 補題        } ,
    { problem     } { 問題        } ,
    { proposition } { 命題        } ,
    { remark      } { 注意        } ,
    { solution    } { 解法        } ,
    { theorem     } { 定理        }
%</ja>
  }
%<zh>  { \@@_name_const:nnn { zh } #1 }
%<en>  { \@@_name_const:nnn { en } #1 }
%<de>  { \@@_name_const:nnn { de } #1 }
%<ja>  { \@@_name_const:nnn { ja } #1 }
%</scheme>
%    \end{macrocode}
%
% 定义前会检测环境是否已经存在,避免覆盖用户的定义。
%    \begin{macrocode}
%<*class>
\cs_new_protected:Nn \@@_new_theorems:
  {
    \clist_map_inline:nn
      {
        assumption, axiom, conjecture, corollary, definition, example,
        exercise, lemma, problem, proposition, theorem
      }
      {
        \cs_if_exist:cF {##1}
          {
            \exp_args:Nnv  \newtheorem  {##1}
%<!article>              { c_@@_name_ ##1 _ \g_@@_lang_tl _tl } [ chapter ]
%<article>              { c_@@_name_ ##1 _ \g_@@_lang_tl _tl }
          }
      }
    \clist_map_inline:nn
      { remark, solution }
      {
        \cs_if_exist:cF {##1}
          {
            \exp_args:NNnv \newtheorem* {##1}
              { c_@@_name_ ##1 _ \g_@@_lang_tl _tl }
          }
      }
  }
%    \end{macrocode}
%
% \pkg{amsthm} 会定义 \tn{openbox},为避免与一些宏包冲突,
% 我们先保存 \tn{openbox},然后取消定义。
%    \begin{macrocode}
\ctex_at_begin_package:nn { amsthm }
  {
    \cs_if_exist:NT \openbox
      {
        \cs_new_eq:NN \SJTU@orig@openbox \openbox
        \cs_undefine:N \openbox
      }
  }
\ctex_at_end_package:nn { amsthm }
  {
    \@@_cs_provide_eq:NN \QED \openbox
    \cs_if_exist:NT \SJTU@orig@openbox
      { \cs_set_eq:NN \openbox \SJTU@orig@openbox }
    \tl_set:Nn \qedsymbol { \ensuremath { \QED } }
    \RenewDocumentEnvironment { proof } { O{ \proofname } }
      {
        \par \pushQED { \qed }
        \SJTU@style@thm@body@font \dim_zero:N \topsep
        \trivlist
        \item
          [
            \skip_horizontal:N \labelsep
            \SJTU@style@thm@header@font #1 \@addpunct { \enskip }
          ]
        \ignorespaces
      }
      { \popQED \endtrivlist \@endpefalse }
    \newtheoremstyle { sjtu }
      { } { } { \SJTU@style@thm@body@font } { }
      { \SJTU@style@thm@header@font } { } { \ccwd } { }
    \@@_disable_package_load:n { ntheorem }
    \ctex_at_end_preamble:n
      {
        \theoremstyle { sjtu }
        \@@_new_theorems:
      }
  }
%    \end{macrocode}
%
% \pkg{ntheorem} 宏包。
%    \begin{macrocode}
\ctex_at_end_package:nn { ntheorem }
  {
    \@@_disable_package_load:n { amsthm }
    \ctex_at_end_preamble:n
      {
        \@@_cs_provide_eq:NN \QED \c_empty_tl
        \theoremheaderfont { \SJTU@style@thm@header@font }
        \theorembodyfont   { \SJTU@style@thm@body@font   }
        \theoremseparator  { \enskip }
        \theoremsymbol { \ensuremath { \QED } }
        \qedsymbol     { \ensuremath { \QED } }
        \cs_if_exist:NF \proof
          { \newtheorem* { proof } { \proofname } }
        \theoremsymbol { }
        \@@_new_theorems:
      }
  }
%    \end{macrocode}
%
% \changes{v2.1.1}{2024/03/22}{添加 \pkg{thmtools} 宏包支持。}
% \subsubsection{\pkg{thmtools} 宏包}
%
% 使用 \pkg{titletoc} 包设置 \tn{listoftheorems} 的样式。
%    \begin{macrocode}
\ctex_at_end_package:nn { thmtools }
  {
    \cs_set:Npn \thmtlo@newentry
      {
        \exp_args:NV \titlecontents \thmt@envname
          [ \thmt@listnumwidth ] { }
          { \contentslabel { \thmt@listnumwidth } }
          { \hspace* { - \thmt@listnumwidth } }
          { \SJTU@leaders \thecontentspage }
        \exp_args:NV \contentsuse \thmt@envname { loe }
      }
    \cs_set:Npn \thmtlo@chaptervspacehack { }
    \RenewDocumentCommand \listoftheorems { O{ } }
      {
        \group_begin:
          \tl_set_eq:NN \l_@@_heading_marking_tl \c_novalue_tl
          \keys_set_groups:nnnN { sjtu / private } { heading }
            { title = \listtheoremname, #1 } \l_@@_tmp_tl
          \tl_set_rescan:NnV \l_@@_tmp_tl { } \l_@@_tmp_tl
          \@@_heading:VV
            \l_@@_heading_marking_tl \l_@@_heading_title_tl
          \exp_args:No \setlisttheoremstyle { \l_@@_tmp_tl }
          \cs_set:Npn \contentsline ##1
            { \use:c { thmt@contentsline@ ##1 } {##1} }
          \clist_map_inline:Nn \thmt@allenvs
            {
              \tl_set:Nn \thmt@envname {##1}
              \thmtlo@newentry
            }
          \@fileswfalse
          \AddToHook { enddocument / afterlastpage }
            {
              \if@filesw
                \@ifundefined { tf@loe }
                  {
                    \expandafter\newwrite\csname tf@loe\endcsname
                    \immediate\openout \csname tf@loe\endcsname \jobname.loe\relax
                  } { }
              \fi
            }
          \@starttoc { loe }
        \group_end:
      }
  }
%    \end{macrocode}
%
% \subsubsection{\pkg{algorithm} 宏包和 \pkg{algorithm2e} 宏包}
%
% \pkg{algorithm} 宏包。
%    \begin{macrocode}
\ctex_at_end_package:nn { algorithm }
  {
    \tl_set:Nn \fname@algorithm   { \SJTU@algorithmname     }
    \tl_set:Nn \listalgorithmname { \SJTU@listalgorithmname }
%<!article>    \SJTU@counterwithin { algorithm } { chapter }
%<thesis>    \clist_put_right:Nn \l_@@_counter_without_chapter_clist { algorithm }
    \@@_new_list_of:Nnnn \listofalgorithms { algorithm }
      { \fname@algorithm } { \listalgorithmname }
  }
%    \end{macrocode}
%
% \pkg{algorithm2e} 宏包。
%    \begin{macrocode}
%<!article>\ctex_at_begin_package:nn { algorithm2e }
%<!article>  { \cs_set_eq:NN \SJTU@orig@at@chapter \@chapter }
\ctex_at_end_package:nn { algorithm2e }
  {
%<!article>    \cs_set_eq:NN \@chapter \SJTU@orig@at@chapter
    \SetAlgorithmName { \SJTU@algorithmname     }
                      { \SJTU@algorithmname     }
                      { \SJTU@listalgorithmname }
    \SetAlgoCaptionSeparator { \enskip }
%<!article>    \SJTU@counterwithin { algocf } { chapter }
%<thesis>    \clist_put_right:Nn \l_@@_counter_without_chapter_clist { algocf }
    \@@_new_list_of:Nnnn \listofalgorithms { algocf }
      { \algorithmcfname } { \listalgorithmcfname }
    \ctex_patch_cmd:Nnn \algocf@latexcaption
      { \addcontentsline }
      { \caption@iflist { \addcontentsline } { \@gobblethree } }
  }
%    \end{macrocode}
%
% \subsubsection{\pkg{nomencl} 宏包}
%    \begin{macrocode}
\ctex_at_end_package:nn { nomencl }
  { \tl_set:Nn \nomname { \SJTU@nomname } }
%    \end{macrocode}
%
% \changes{v2.0.3}{2023/09/25}{添加 \pkg{siunitx} 本地化支持。}
% \subsubsection{\pkg{translations} 宏包}
%    \begin{macrocode}
\ctex_at_end_package:nn { translations }
  {
    \DeclareLanguage { chinese }
    \DeclareLanguageAlias { Chinese } { chinese }
  }
%    \end{macrocode}
%
% \subsubsection{\pkg{siunitx} 宏包}
%    \begin{macrocode}
\ctex_at_end_package:nn { siunitx }
  {
    \RequirePackage { translations }
    \DeclareTranslation { Chinese  } { and } { 和 }
    \DeclareTranslation { Japanese } { and } { と }
    \DeclareTranslation { Chinese  }
      { to~(numerical~range) } { \textasciitilde }
    \DeclareTranslation { Japanese }
      { to~(numerical~range) } { \textasciitilde }
    \IfPackageAtLeastTF { siunitx } { 2021/05/17 } { }
      {
        \DeclareTranslation { English } { to~(numerical~range) } { to  }
        \DeclareTranslation { German }  { to~(numerical~range) } { bis }
        \keys_set:nn { siunitx }
          {
            list-final-separator =
              {
                \ifmmode \  \else \space \fi
                \text { \GetTranslation { and } }
                \ifmmode \  \else \space \fi
              } ,
            list-pair-separator  =
              {
                \ifmmode \  \else \space \fi
                \text { \GetTranslation { and } }
                \ifmmode \  \else \space \fi
              } ,
            range-phrase         =
              {
                \ifmmode \  \else \space \fi
                \text { \GetTranslation { to~(numerical~range) } }
                \ifmmode \  \else \space \fi
              }
          }
      }
  }
%</class>
%    \end{macrocode}
%
% \end{implementation}
%
% \Finale
%
\endinput
