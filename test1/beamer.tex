\documentclass{beamer}

\mode<presentation> {
	\usetheme{Madrid}
	\usecolortheme{rose}
	
	%\setbeamertemplate{footline}
	%若要删除所有幻灯片中的页脚,请取消注释此行
	
	%\setbeamertemplate{footline}[页码]
	%若要用简单的幻灯片计数替换所有幻灯片中的页脚,请取消注释此行
	
	%\setbeamertemplate{navigation symbols}{}
	%要删除所有幻灯片底部的导航符号,请取消注释此行
}

\setbeamertemplate{frametitle continuation}{} % 删除自动续页编号

\usepackage{graphicx} % 允许包含图像
\usepackage{booktabs} % 允许在表中使用\toprule、\ midrule和\ bottomrule
\usepackage[UTF8,noindent]{ctexcap}  % 使用中文输入及显示
\usepackage[bookmarks=true]{hyperref} % 允许使用超链接
\everymath{\displaystyle} % 使所有数学公式显示为行间公式

%-----------------------------------
%	以下为正文
%-----------------------------------

\title[第一章]{第一章:弦振动方程、热传导方程、拉普拉斯方程的导出} 

\author{方言} % Your name
\institute[SJTU] % 您的机构将出现在每张幻灯片的底部,可能是节省空间的简写
{
	上海交通大学 \\ % 你所在的机构
	\medskip
	\textit{fangyan\_pj@sjtu.edu.cn} % Your email address
}
\date{\today} % 日期,可以更改为自定义日期

\begin{document}

\begin{frame}[allowframebreaks]
	\titlepage % 将标题页打印为第一张幻灯片
\end{frame}

\begin{frame}[allowframebreaks]%目录页
	\frametitle{Overview} % 目录幻灯片,注释此块以将其删除
	\tableofcontents
\end{frame}

%-----------------------------------
%	开始创建PPT
%-----------------------------------

\section{弦振动方程} % 可以创建章节,以便将您的演讲组织成离散的块,所有章节和小节都会自动打印在目录中,作为演讲的概述
\subsection{弦振动方程的导出} % 可以在一组具有共同主题的幻灯片之前创建一个小节,以进一步将您的演示分解为块

\begin{frame}[allowframebreaks]
	\frametitle{弦振动方程}
	一条弦固定在区间 $[0,L]$ 上,密度为 $\rho$,在平衡处附近做微小的振动,记 $u(t,x)$ 为弦在时刻 $t$ 处于位置 $x$ 的位移,满足以下的微分方程:

	\begin{equation}
		\label{eq:wave}
		u_{tt} -a^2u_{xx}=F(t,x), \quad 0 < x < L, \quad t > 0
	\end{equation}

	\eqref{eq:wave} 式中 $a$ 为待定的弦的波速,若 $F\equiv 0$,则称为其次方程,否则为非齐次方程。

\end{frame}

\begin{frame}[allowframebreaks]
	\frametitle{弦振动方程的证明}
	任取一段弦 $[x,x+\Delta x]$,这段弦的长度为

	$$
		\Delta s = \int_x^{x+\Delta x}\sqrt{1+\left(\frac{\partial u}{\partial x}\right)^2}\,\mathrm{d}x.
	$$

	其中 $\left\vert\frac{\partial u}{\partial x}\right\vert\ll 1$,则 $\Delta s \approx \Delta x$,则其弹力与时间无关,设为 $T(x)$。

\end{frame}

\begin{frame}[allowframebreaks]
	\frametitle{弦振动方程的证明}
	我们分别在水平方向和竖直方向上对这段弦进行受力分析:

	\begin{itemize}
		\item 水平方向上速度为 $0$,则其受力平衡
		      $$
			      T(x+\Delta x)\cos\theta(x+\Delta x) - T(x)\cos\theta(x) = 0,
		      $$
		\item 在竖直方向( $u^-$ )上进行受力分析
		      $$
			      \rho\Delta s \frac{\partial^2u}{\partial x^2}=T(\sin\beta - \sin\alpha) + F(t,x)\Delta s
		      $$

		      其中 $\displaystyle\lim_{\theta\to 0}\sin\theta = \lim_{\theta\to 0}\tan\theta$,再将 $\Delta s = \Delta x$ 代入上式,得到:

		      $$
			      \rho\Delta x \frac{\partial^2u}{\partial t^2}=T(\frac{\partial u}{\partial x}(x+\Delta x) - \frac{\partial u}{\partial x}(x)) + F(t,x)\Delta x
		      $$

		      \newpage

		      再令 $\Delta x \to 0$,得到:

		      $$
			      \rho \frac{\partial^2u}{\partial t^2} = T\frac{\partial^2u}{\partial x^2} + F(t,x)
		      $$

		      我们令 $a^2 = \frac{T}{\rho},F(t,x)=\frac{f(t,x)}{\rho}$,则得到弦振动方程:

		      $$
			      u_{tt} -a^2u_{xx}=F(t,x), \quad 0 < x < L, \quad t > 0
		      $$
	\end{itemize}
\end{frame}

%-----------------------------------

\subsection{弦振动方程的定解问题}

\begin{frame}[allowframebreaks]
	\frametitle{初值条件}
	当一条弦足够长并且需要研究远离两端的点的振动情况时,可以假设弦无限长。

	\begin{equation}
		\label{eq:wave_ivp}
		\left\{
		\begin{aligned}
			 & u_{tt}-a^2u_{xx}=F(t,x), \\
			 & u(0,x)=\varphi(x),       \\
			 & u_t(0,x)=\psi(x).
		\end{aligned}
		\right.
	\end{equation}

	\eqref{eq:wave_ivp} 式的三个条件分别表示了方程本身、初始位移和初始速度。\newline

	这个问题被称为弦振动方程的\textbf{初值问题}或者\textbf{Cauchy 问题}。
\end{frame}

\begin{frame}[allowframebreaks]
	\frametitle{边界条件}
	当我们要考虑的弦在一个有限的区间 $I=[0,L]$ 上的振动情况时,需要给出边界条件,即弦在端点处的条件。\newline

	我们以一个端点 $x=a$ 为例,常见的边界条件有以下三种:

	\begin{enumerate}
		\item Dirichlet 边界条件:

		      $$
			      u(t,a)=\varphi(t),\forall t> 0
		      $$
		\item Neumann 边界条件:

		      $$
			      u_x(t,a)=\mu(t),\forall t> 0
		      $$

		      这表示了端点处的力。\newline

		      特别的,当 $\mu(t)= 0$ 时,表示端点处没有力的作用,称为自由条件。

		\item Robin 边界条件:

		      $$
			      \begin{aligned}
				                  & T\frac{\partial u}{\partial x}(t,a)-ku(t,a)=0                \\
				      \Rightarrow & \left(\frac{\partial u}{\partial x}-\sigma u\right)|_{x=a}=0
			      \end{aligned}
		      $$

		      其中 $\sigma = \frac{k}{T}$ 表示弦的一端与一线性弹簧连接。\newline

		      若弹簧为非线性的,则有:

		      $$
			      \left(\frac{\partial u}{\partial x}-\sigma (u)\right)|_{x=a}=0
		      $$
	\end{enumerate}
\end{frame}

%-----------------------------------

\section{热传导方程}

\subsection{热传导方程的导出}


\end{document}
