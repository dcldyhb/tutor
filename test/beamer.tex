%-----------------------------------
%-----------------------------------
% Beamer template made by Chenxiao
% 2023.11.04 at Qingdao,China
%-----------------------------------
%-----------------------------------

\documentclass{beamer}

\mode<presentation> {
	\usetheme{Madrid}
	\usecolortheme{rose}
	
	%\setbeamertemplate{footline}
	%若要删除所有幻灯片中的页脚,请取消注释此行
	
	%\setbeamertemplate{footline}[页码]
	%若要用简单的幻灯片计数替换所有幻灯片中的页脚,请取消注释此行
	
	%\setbeamertemplate{navigation symbols}{}
	%要删除所有幻灯片底部的导航符号,请取消注释此行
}

\usepackage{graphicx} % 允许包含图像
\usepackage{booktabs} % 允许在表中使用\toprule、\ midrule和\ bottomrule
\usepackage[UTF8,noindent]{ctexcap}  % 使用中文输入及显示
\usepackage[bookmarks=true]{hyperref}

%-----------------------------------
%	以下为正文
%-----------------------------------

\title[Test]{Beamer Test} 
% 简短标题显示在每张幻灯片的底部,完整标题仅在标题页上

\author{Fangyan} % Your name
\institute[SJTU] % 您的机构将出现在每张幻灯片的底部,可能是节省空间的简写
{
	Shanghai Jiao Tong University \\ % 你所在的机构
	\medskip
	\textit{fangyan\_pj@sjtu.edu.cn} % Your email address
}
\date{\today} % 日期,可以更改为自定义日期

\begin{document}
	
	\begin{frame}
		\titlepage % 将标题页打印为第一张幻灯片
	\end{frame}
	
	\begin{frame}
		\frametitle{Overview} % 目录幻灯片,注释此块以将其删除
		\tableofcontents % 在整个演示过程中,如果您选择使用\ section{}和\ submission{}命令,这些命令将自动打印在此幻灯片上,作为演示的概述
	\end{frame}
	
	%-----------------------------------
	%	开始创建PPT
	%-----------------------------------
	
	\section{First Section} % 可以创建章节,以便将您的演讲组织成离散的块,所有章节和小节都会自动打印在目录中,作为演讲的概述
	\subsection{Subsection Example 1} % 可以在一组具有共同主题的幻灯片之前创建一个小节,以进一步将您的演示分解为块
	
	\begin{frame}
		\frametitle{Paragraphs of Text}
		\textit{We have not succeeded in answering all our problems. The answers we have found only serve to raise a whole set of new questions. In some ways we feel we are as confused as
		ever, but we believe we are confused on a higher level and about more important
		things.}
		
		\rightline{Posted outside the mathematics reading room,Tromso University}
	\end{frame}
	
	%------------------------------------------------
	\subsection{Subsection Example 2}
	
	\begin{frame}
		\frametitle{Bullet Points}
		\begin{itemize}
			\item What we do may be small, but it has a certain character of permanence.
			\item Euclid geometry was as dazzling as first love.
			\item Talk is cheap, solve the PDE.
		\end{itemize}
	\end{frame}
	
	%------------------------------------------------
	
	\begin{frame}
		\frametitle{Blocks of Highlighted Text}
		\begin{block}{Block 1}
			Certainly the best times were when I was alone with mathematics: free of ambition and pretense, and indifferent to the world.
		\end{block}
		
		\begin{block}{Block 2}
			Wir müssen wissen, Wir werden wissen.
		\end{block}
		
		\begin{block}{Block 3}
			If people do not believe that mathematics is simple, it is only because they do not realize how complicated life is.	
		\end{block}
	\end{frame}
	
	%------------------------------------------------
	
	\begin{frame}
		\frametitle{Multiple Columns}
		\begin{columns}[c] % The "c" option specifies centered vertical alignment while the "t" option is used for top vertical alignment
			
			\column{.45\textwidth} % Left column and width
			\textbf{Heading}
			\begin{enumerate}
				\item Statement
				\item Explanation
				\item Example
			\end{enumerate}
			
			\column{.5\textwidth} % Right column and width
			Lorem ipsum dolor sit amet, consectetur adipiscing elit. Integer lectus nisl, ultricies in feugiat rutrum, porttitor sit amet augue. Aliquam ut tortor mauris. Sed volutpat ante purus, quis accumsan dolor.
			
		\end{columns}
	\end{frame}
	
	%------------------------------------------------
	\section{Second Section}
	%------------------------------------------------
	
	\begin{frame}
		\frametitle{Table}
		\begin{table}
			\begin{tabular}{l l l}
				\toprule
				\textbf{Treatments} & \textbf{Response 1} & \textbf{Response 2}\\
				\midrule
				Treatment 1 & 0.0003262 & 0.562 \\
				Treatment 2 & 0.0015681 & 0.910 \\
				Treatment 3 & 0.0009271 & 0.296 \\
				\bottomrule
			\end{tabular}
			\caption{Table caption}
		\end{table}
	\end{frame}
	
	%------------------------------------------------
	
	\begin{frame}
		\frametitle{Theorem}
		\begin{theorem}[Mass--energy equivalence]
		\centerline{$E = mc^2$}
		\end{theorem}
	\end{frame}
	
	%------------------------------------------------
	
	\begin{frame}[fragile] % Need to use the fragile option when verbatim is used in the slide
		\frametitle{Verbatim}
		\begin{example}[Theorem Slide Code]
			\begin{verbatim}
\begin{frame}
\frametitle{Theorem}
\begin{theorem}[Mass--energy equivalence]
$E = mc^2$
\end{theorem}
\end{frame}\end{verbatim}
		\end{example}
	\end{frame}
	
	%------------------------------------------------
	
	\begin{frame}
		\frametitle{Figure}
		includegraphics 命令用于插入图片,你提供了图片文件的名称 test,但确保在LaTeX源文件所在的目录中有名为 test 的图片文件,并且该文件的格式(如 .png、.jpg、.pdf 等)是TeX支持的格式。如果图片文件不在相同的目录中,你需要提供正确的路径。
%		\begin{figure}
%		\includegraphics[width=0.8\linewidth]{test}
%		\end{figure}
	\end{frame}
	
	
	\begin{frame}
		\Huge{\centerline{Thank you all!}}
		\centering{{\normalsize 我们是孩子,但我们精力充沛,勇往直前!}}
	\end{frame}
	
\end{document} 
