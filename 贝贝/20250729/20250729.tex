\documentclass[a4paper , final]{ctexart}
\pagestyle{plain}
\everymath{\displaystyle} % 使所有数学公式默认使用行间公式格式

\usepackage{amsmath}
\usepackage[left=2.5cm, right=2.5cm, top=2.5cm, bottom=2.5cm]{geometry}
\usepackage{ifplatform}
\ifwindows
  \setCJKmainfont{SimSun} % 注意:请确保您的系统中已安装宋体 (SimSun) 字体
\else
  \ifmacosx
    \setCJKmainfont{Songti SC} % macOS 上的宋体字体
\fi
\fi
%\usepackage{bookmark}
\usepackage[hidelinks]{hyperref}
\usepackage{graphicx}
\usepackage{subcaption}

\usepackage{enumitem} % 用于自定义列表

% --- 标题格式设置 ---
\usepackage{titlesec}
\titlespacing*{\section}{0pt}{3.5ex plus 1ex minus .2ex}{2.3ex plus .2ex}

\graphicspath{{figures/}} % 图片路径

% --- 核心修改:定义并统一题目环境 ---
\newlist{problems}{enumerate}{1}
\setlist[problems,1]{label=\arabic*., leftmargin=*}

% 环境1:常规题目 (不含图片)
% 使用 \newenvironment 定义新的环境
\newenvironment{problem}[1]{%
  \item #1
  \par
  \vspace{8cm}
}{}

% 环境2:带图片的题目 (已升级)
% #1 (可选): 图片底部距离, 默认 2cm
% #2: 题目文本
% #3: 图片命令
\newenvironment{problemwithfig}[3][2cm]{%
  \item #2
  \par\noindent
  \begin{minipage}[t][8cm][b]{\linewidth}
    \vfill
    \hfill #3
    \par\vspace{#1} % 图片底部的间距由 #1 控制
  \end{minipage}
}{}
% ---------------------------

\title{不等式综合 II}
\date{\today}

\begin{document}
\maketitle


\section*{不等式的应用}
 
\begin{problems}
  \begin{problem}
  {(根的存在性定理)
  $x_1$ 与 $x_2$ 分别是实系数一元二次方程 $ax^2 + bx + c = 0$ 和 $-ax^2 + bx + c = 0$ 的一个根, 且 $x_1 \ne x_2, x_1 \ne 0, x_2 \ne 0$.证明:方程 $\frac{a}{2}x^2 + bx + c = 0$ 有且仅有一个根介于 $x_1$ 与 $x_2$ 之间.
  }
  \end{problem}

  \begin{problem}
  {(和差化积、均值不等式)
  已知 $x, y > 0$, 求函数 $f(x,y) = \sin x + \sin y - \sin(x+y)$ 的最大值.
  }
  \end{problem}

  \begin{problem}
    {(以函数值表示系数)
      设正系数一元二次方程 $ax^2 + bx + c = 0$ 有实根,证明: $\min\{a,b,c\}\leq\frac{1}{4}(a+b+c)$.
    }
  \end{problem}

  \begin{problem}
    {(绝对值不等式)
      设函数 $f(x) = x^2+ax+b(a,b\in\mathbf{R})$. 计 $M(a,b)$ 为 $\vert f(x)\vert$ 在区间 $[-1,1]$ 上的最大值.
      \begin{enumerate}[label=(\arabic*)]
        \item 证明: 当 $\vert a\vert\ge 2$ 时, $M(a,b) \geq 2$.
        \item 当 $a,b$ 满足 $M(a,b)\leq 2$ 时, 求 $\vert a\vert + \vert b\vert$ 的最大值.
      \end{enumerate} 
    }
  \end{problem}
\end{problems}
\end{document}