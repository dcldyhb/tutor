\documentclass[a4paper , final]{ctexart}
\pagestyle{plain}
\everymath{\displaystyle} % 使所有数学公式默认使用行间公式格式

\usepackage{amsmath}
\usepackage[left=2.5cm, right=2.5cm, top=2.5cm, bottom=2.5cm]{geometry}
\usepackage{ifplatform}
\ifwindows
  \setCJKmainfont{SimSun} % 注意:请确保您的系统中已安装宋体 (SimSun) 字体
\else
  \ifmacosx
    \setCJKmainfont{Songti SC} % macOS 上的宋体字体
\fi
\fi
%\usepackage{bookmark}
%\usepackage[hidelinks]{hyperref}
\usepackage{graphicx}
%\usepackage{subcaption}

\usepackage{enumitem} % 用于自定义列表

% --- 标题格式设置 ---
\usepackage{titlesec}
\titlespacing*{\section}{0pt}{3.5ex plus 1ex minus .2ex}{2.3ex plus .2ex}

\graphicspath{{figures/}} % 图片路径

% --- 核心修改:定义并统一题目环境 ---
\newlist{problems}{enumerate}{1}
\setlist[problems,1]{label=\arabic*., leftmargin=*}

% 环境1:常规题目 (不含图片)
% 使用 \newenvironment 定义新的环境
\newenvironment{problem}[1]{%
  \item #1
  \par
  \vspace{8cm}
}{}

% 环境2:带图片的题目 (已升级)
% #1 (可选): 图片底部距离, 默认 2cm
% #2: 题目文本
% #3: 图片命令
\newenvironment{problemwithfig}[3][2cm]{%
  \item #2
  \par\noindent
  \begin{minipage}[t][8cm][b]{\linewidth}
    \vfill
    \hfill #3
    \par\vspace{#1} % 图片底部的间距由 #1 控制
  \end{minipage}
}{}
% ---------------------------

\title{不等式综合 II}
\date{2025年7月29日}

\begin{document}
\maketitle


\section*{不等式的应用}
 
\begin{problems}
  \begin{problem}
  {(根的存在性定理)
  $x_1$ 与 $x_2$ 分别是实系数一元二次方程 $ax^2 + bx + c = 0$ 和 $-ax^2 + bx + c = 0$ 的一个根, 且 $x_1 \ne x_2, x_1 \ne 0, x_2 \ne 0$.证明:方程 $\frac{a}{2}x^2 + bx + c = 0$ 有且仅有一个根介于 $x_1$ 与 $x_2$ 之间.
  }
  \end{problem}

  \begin{problem}
  {(和差化积、均值不等式)
  已知 $x, y > 0$, 求函数 $f(x,y) = \sin x + \sin y - \sin(x+y)$ 的最大值.
  }
  \end{problem}

  \begin{problem}
    {(以函数值表示系数)
      设正系数一元二次方程 $ax^2 + bx + c = 0$ 有实根,证明: $\min\{a,b,c\}\leq\frac{1}{4}(a+b+c)$.
    }
  \end{problem}

  \begin{problem}
    {(绝对值不等式)
      设函数 $f(x) = x^2+ax+b(a,b\in\mathbf{R})$. 计 $M(a,b)$ 为 $\vert f(x)\vert$ 在区间 $[-1,1]$ 上的最大值.
      \begin{enumerate}[label=(\arabic*)]
        \item 证明: 当 $\vert a\vert\ge 2$ 时, $M(a,b) \geq 2$.
        \item 当 $a,b$ 满足 $M(a,b)\leq 2$ 时, 求 $\vert a\vert + \vert b\vert$ 的最大值.
      \end{enumerate} 
    }
  \end{problem}

  \newpage
  \begin{problem}
    {(基本不等式)
      记 $F(x,y)=x+y-a(x+2\sqrt{2xy}),x,y\in\mathbf{R}^+$
      \begin{enumerate}[label=(\arabic*)]
        \item 是否存在 $x_0\in\mathbf{R}^+$,使 $F(x_0,2)=2$?
        \item 若对任意的 $x,y\in\mathbf{R}^+$,都有 $F(x,y)\geq 0$, 求 $a$ 的取值范围.
      \end{enumerate}
    }
  \end{problem}

  \begin{problem}
    {(基本不等式,三角不等式)
      已知正实数 $x,y$ 满足 $a=x+y,b=\sqrt{x^2+7xy+y^2}$.
      \begin{enumerate}[label=(\arabic*)]
        \item 当 $y=1$ 时,求 $\frac{b}{a}$ 的最大值;
        \item 若 $c^2=kxy$ ,若对于任意的正数 $x,y$,以 $a,b,c$ 为长度的线段恒能构成三角形,求 $k$ 的取值范围.
      \end{enumerate}
    }
  \end{problem}

  \begin{problem}
    {
      设 $f(x)=x^2-2ax+2$,当 $x\in[-1,+\infty)$ 时, $f(x)\geq a$ 恒成立,则实数 $a$ 的取值范围是 \underline{\hspace{3cm}}.
    }
  \end{problem}

  %\newpage
  \begin{problem}
    {
      对于一切实数 $a$,二次函数 $f(x)=ax^2+bx+c(a<b)$ 的值恒为非负实数,求 $M=\frac{a+b+c}{b-a}$ 的最小值.   
    }
  \end{problem}
  
  \begin{problem}
    {
      大圆酒杯的轴截口为函数 $y=x^4$ 的图像,往酒杯里面放一半径为 $r$ 的小球,当半径 $r$ 最大为多少时,小球可接触到杯底的最低点?
    }
  \end{problem}

  \begin{problem}
    {
      设实数 $a,b,c,d$ 满足 $ab=c^2+d^2=1$, 则 $(a-c)^2+(b-d)^2$ 的最小值为 \underline{\hspace{3cm}}.
    }
  \end{problem}
\end{problems}

\section*{几个重要的不等式}

\subsection*{平均值不等式}


若 $a_i \ge 0,i=1,2,\ldots,n$, 计 $H_n =\frac{n}{\frac{1}{a_1}+\frac{1}{a_2}+\cdots+\frac{1}{a_n}}$, $G_n = \sqrt[n]{a_1 a_2 \cdots a_n}$, $A_n = \frac{a_1+a_2+\cdots+a_n}{n}$,$Q_n = \sqrt{a_1^2+a_2^2+\cdots+a_n^2}$ 分别为这 $n$ 个数的调和平均、几何平均、算术平均和平方平均,则有:

\begin{equation*}
  H_n \leq G_n \leq A_n \leq Q_n
\end{equation*}

\subsection*{Cauchy 不等式}

设 $a_1,a_2,\ldots,a_n,b_1,b_2,\ldots,b_n$ 是实数,则有

\begin{equation*}
  (a_1^2+a_2^2+\cdots+a_n^2)(b_1^2+b_2^2+\cdots+b_n^2) \geq (a_1b_1+a_2b_2+\cdots+a_nb_n)^2
\end{equation*}

当且仅当 $\frac{a_1}{b_1}=\frac{a_2}{b_2}=\cdots=\frac{a_n}{b_n}$ 时,等号成立.

\subsection*{排序不等式}

设两个实数列 $a_1 \leq a_2 \leq \cdots \leq a_n$ 和 $b_1 \leq b_2 \leq \cdots \leq b_n$, 则有

\begin{equation*}
  a_1b_1 + a_2b_2 + \cdots + a_nb_n\text{(顺序和)} \geq a_1b_{i_1} + a_2b_{i_2} + \cdots + a_nb_{i_n}\text{(乱序和)} \geq a_1b_n + a_2b_{n-1} + \cdots + a_nb_1\text{(逆序和)}
\end{equation*}

其中 $i_1,i_2,\ldots,i_n$ 是 $1,2,\ldots,n$ 的一个排列.

\subsection*{琴生不等式}

设 $f(x)$ 是一个在区间 $I$ 上连续的函数,若对任意的 $x_1,x_2\in I$, 有不等式 $2f\left(\frac{x_1+x_2}{2}\right) \geq f(x_1) + f(x_2)$ 成立(或 $f(x)$ 存在二阶导数且 $f''(x)<0$),则称 $f(x)$ 是一个上凸函数.

若函数 $f(x)$ 为区间 $I$ 上的上凸函数,则对任意的 $x_1,x_2,\ldots,x_n\in I$, 有:

\begin{equation*}
  f(\frac{x_1+x_2+\cdots x_n}{n})\geq \frac{1}{n}(f(x_1)+f(x_2)+\cdots+f(x_n))
\end{equation*}

\subsection*{幂平均不等式}

设 $p>q$, 则对任意的非负实数 $a_1,a_2,\ldots,a_n$ 有:

\begin{equation*}
  \left(\frac{a_1^p+a_2^p+\cdots+a_n^p}{n}\right)^{\frac{1}{p}} \geq \left(\frac{a_1^q+a_2^q+\cdots+a_n^q}{n}\right)^{\frac{1}{q}}
\end{equation*}

\begin{problems}
  \begin{problem}
    {(幂平均不等式或均值不等式)
      已知 $x,y,z>0,x+y+z=1$,证明: $x^3+y^3+z^3\geq \frac{1}{9}$
    }
  \end{problem}

  \begin{problem}
    {
      设 $x,y,z\in\mathbf{R}^+$, 且 $x+y+z\ge xyz$, 求 $\frac{x^2+y^2+z^2}{xyz}$ 的最小值.
    }
  \end{problem}

  \begin{problem}
    {(均值不等式)
      证明: $\left(1+\frac{1}{n}\right)^n < \left(1+\frac{1}{n+1}\right)^{n+1}$.
    }
  \end{problem}

  \begin{problem}
    {
      试求正实数 $A$ 的最大值,使得对于任意实数 $x,y,z$, 不等式 $x^4+y^4+z^4+x^2yz+xy^2z+xyz^2-A(xy+yz+xy)^2\ge 0$ 成立.
    }
  \end{problem}

  \begin{problem}
    {
      求最小的正实数 $k$, 使得不等式 $ab+bc+ca+k\left(\frac{1}{a}+\frac{1}{b}+\frac{1}{c}\right)\geq 9$ 对任意正实数 $a,b,c$ 成立.
    }
  \end{problem}

  \begin{problem}
    {
      已知 $a,b,c\in \mathbf{R}^+$, 且 $abc=1$, 则 $\frac{1}{a^3(b+c)}+\frac{1}{b^3(c+a)}+\frac{1}{c^3(a+b)}$ 的最小值为\underline{\hspace{3cm}}.
    }
  \end{problem}

  \begin{problem}
    {
      设 $a,b,c\ge 0$, 且 $a^2+b^2+c^2=3$, 则 $a\sqrt{b}+b\sqrt{c}+c\sqrt{a}$ 的最大值为 \underline{\hspace{3cm}}.
    }
  \end{problem}

  \begin{problem}
    {(Cauchy 不等式)
      求出所有实数 $a$ 使得存在非负实数 $x_1,x_2,x_3,x_4,x_5$ 满足下列关系: $\sum_{k=1}^5 kx_k=a,\sum_{k=1}^5 k^3x_k =a^2,\sum_{k=1}^5 k^5x_k =a^3$.
    }
  \end{problem}

  \begin{problem}
    {(均值不等式)
      证明: $\sum_{k=0}^{n-1} \frac{1}{n+k}>n(\sqrt[n]{2}-1)$.
    }
  \end{problem}

  \begin{problem}
    {(排序不等式)
      正实数 $a_1,a_2,\ldots,a_n$ 的任一排列为 ${a'}_1,{a'}_2\ldots {a'}_n$, 证明: $\frac{a_1}{{a'}_1}+\frac{a_2}{{a'}_2}+\cdots+\frac{a_n}{{a'}_n}\geq n$.
    }
  \end{problem}

  \begin{problem}
    {
      设三角形的三条边长分别为 $a,b,c$, 求 $\frac{a+b+c}{\sqrt{ab+bc+ac}}$ 的取值范围.
    }
  \end{problem}

  \begin{problem}
    {
      由排序不等式证明均值不等式.
    }
  \end{problem}

  \begin{problem}
    {(琴生不等式)
      设 $x_1,x_2,\ldots,x_n(n\ge 2)$ 是正实数, 满足 $x_1+x_2+\ldots+x_n=1$, 求 $\frac{x_1}{1-x_1}+\frac{x_2}{1-x_2}+\cdots+\frac{x_n}{1-x_n}$ 的最小值.
    }
  \end{problem}

  \begin{problem}
    {(琴生不等式)
      设三角形的三个内角分别为 $A,B,C$, 求 $\sin{\frac{A}{2}}+\sin{\frac{B}{2}}+\sin{\frac{C}{2}}$ 的最大值.
    }
  \end{problem}

  \begin{problem}
    {(琴生不等式,需要仔细挑选原函数)
      设实数 $r_1,r_2,\ldots,r_n\ge 1$ ,证明 $\frac{1}{r_1+1}+\frac{1}{r_2+1}+\cdots+\frac{1}{r_n+1}\ge \frac{n}{\sqrt[n]{r_1r_2\cdots r_n}+1}$.
    }
  \end{problem}
\end{problems}

\end{document}