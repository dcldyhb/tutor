\documentclass[a4paper , final]{ctexart}
\pagestyle{plain}
\everymath{\displaystyle} % 使所有数学公式默认使用行间公式格式

\usepackage{amsmath}
\usepackage[left=2.5cm, right=2.5cm, top=2.5cm, bottom=2.5cm]{geometry}
\usepackage{ifplatform}
\ifwindows
  \setCJKmainfont{SimSun} % 注意:请确保您的系统中已安装宋体 (SimSun) 字体
\else
  \ifmacosx
    \setCJKmainfont{Songti SC} % macOS 上的宋体字体
\fi
\fi
%\usepackage{bookmark}
\usepackage[hidelinks]{hyperref}
\usepackage{graphicx}
\usepackage{subcaption}

\usepackage{enumitem} % 用于自定义列表

% --- 标题格式设置 ---
\usepackage{titlesec}
\titlespacing*{\section}{0pt}{3.5ex plus 1ex minus .2ex}{2.3ex plus .2ex}

\graphicspath{{figures/}} % 图片路径

% --- 核心修改:定义并统一题目环境 ---
\newlist{problems}{enumerate}{1}
\setlist[problems,1]{label=\arabic*., leftmargin=*}

% 环境1:常规题目 (不含图片)
% 使用 \newenvironment 定义新的环境
\newenvironment{problem}[1]{%
  \item #1
  \par
  \vspace{8cm}
}{}

% 环境2:带图片的题目 (已升级)
% #1 (可选): 图片底部距离, 默认 2cm
% #2: 题目文本
% #3: 图片命令
\newenvironment{problemwithfig}[3][2cm]{%
  \item #2
  \par\noindent
  \begin{minipage}[t][8cm][b]{\linewidth}
    \vfill
    \hfill #3
    \par\vspace{#1} % 图片底部的间距由 #1 控制
  \end{minipage}
}{}
% ---------------------------

\title{不等式综合}
\date{2025年7月24日}

\begin{document}
\maketitle

\section*{三角换元}

三角换元的技巧
\begin{enumerate}
  \item 出现形如 $x^2+y^2$ 的式子时,可以考虑使用三角换元;
  \item 出现形如 $\sqrt{1+x}$ 的式子时,可以考虑使用 $\tan\theta$ 换元;
  \item 出现形如 $\sqrt{1-x}$ 的式子时,可以考虑使用 $\sin\theta$ 或 $\frac{1}{\cos\theta}$ 换元;
  \item  注意,三角函数具有有界性.
\end{enumerate}

\begin{problems}
  \begin{problem}
    {
      设 $a\in\mathbf{R}$,若 $a\sqrt{x}+\sqrt{1+x}\leq 1$ 对 $x>0$ 恒成立,求 $a$ 的取值范围.
    }
  \end{problem}

  \begin{problem}
    {
      已知 $x^2+y^2-\sqrt{3}xy = 1(x,y\in\mathbf{R})$,求 $x+y$ 的取值范围
    }
  \end{problem}
\newpage
  \begin{problem}
    {
      已知实数 $a,b,c$ 满足 $a^2+b^2+c^2=1$,求 $ab+c$ 的最小值.
    }
  \end{problem}

  \begin{problem}
    {
      已知 $x^2+y^2\leq 1$,求 $x^2+xy-y^2$ 的最值
    }
  \end{problem}

  \begin{problem}
    {
      已知实数 $x,y$,满足 $x^2+2xy=1$,求 $x^2+y^2$ 的最小值.
    }
  \end{problem}
\end{problems}

\end{document}